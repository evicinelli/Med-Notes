\input{$HOME/pCloud\ Drive/Modelli/base.tex}

\usepackage[a4paper, left=1.5cm, right=6cm, top=2cm, bottom=2cm, marginpar=4.4cm]{geometry}
\usepackage{pdflscape}

% Font
\usepackage{libertine,libertinust1math}
\renewcommand{\familydefault}{\sfdefault}
% \renewcommand*\ttdefault{txtt}

% Figure a margine
\newcommand{\marginfig}[1]{\marginpar{{\textsf{Vedi \textbf{Figura \ref{#1}}}}}}
\newcommand{\marginnote}[1]{\marginpar{\footnotesize← \emph{#1}}}
% \newcommand{\asidefigure}[2]{\marginpar{\phantom{Img:}\newline\includegraphics{#1}\\\footnotesize\emph{#2}}}
\newcommand{\asidefigure}[2]{\marginpar{\phantom{Img:}\vspace{-1cm}\newline\includegraphics{#1}\\\footnotesize\emph{#2}}}
\newcommand{\marginqr}[2]{\marginpar{\vspace{-1cm}\href{#1}{\qrcode{#1}}\\\vspace{0.1cm}\\\footnotesize\emph{↑ #2}}}

% Note a margine
\let\oldfootnote\footnote
\newcounter{mgncount}
\renewcommand\themgncount{\arabic{mgncount} }
\renewcommand\footnote[1]{\refstepcounter{mgncount}\marginpar{{\textsuperscript{\themgncount}}\footnotesize{\emph{#1}}}\footnotemark~}

% Redefine quote envirnoment so that's boxed:
\let\origquote\quote
\let\endorigquote\endquote
\renewenvironment{quote}{
\begin{tcolorbox}
}
{\end{tcolorbox}\endorigquote}

% Anatomia
\definecolor{ossa}{HTML}{BEAE84}
\newcommand{\mus}[1]{\colorbox{Salmon}{\textcolor{white}{\textsc{#1}}}}
\newcommand{\oss}[1]{\colorbox{ossa}{\textcolor{white}{\textsc{#1}}}}
\newcommand{\ven}[1]{\colorbox{RoyalBlue}{\textcolor{white}{\textsc{#1}}}}
\newcommand{\art}[1]{\colorbox{RedOrange}{\textcolor{white}{\textsc{#1}}}}
\newcommand{\tol}[1]{\colorbox{Aquamarine}{\textcolor{white}{\textsc{#1}}}}
\newcommand{\ner}[1]{\colorbox{Dandelion}{\textcolor{white}{\textsc{#1}}}}
\newcommand{\lin}[1]{\colorbox{PineGreen}{\textcolor{white}{\textsc{#1}}}}
\newcommand{\far}[1]{ \fbox{\textsc{#1}} } % Farmaco (principio attivo)
\newcommand{\farf}[1]{\fbox{\fbox{\textsc{#1}}} } % Famiglia di farmaci
\newcommand{\pat}[1]{\colorbox{black}{\textcolor{white}{\textsc{#1}}}}
\renewcommand{\a}[1]{\underline{\textsc{#1}}}

% Sistema nervoso
\newcommand{\nere}[1]{\colorbox{Dandelion}{\textcolor{Maroon}{\textsc{#1}}}} % fibre nervose efferenti
\newcommand{\nera}[1]{\colorbox{Dandelion}{\textcolor{NavyBlue}{\textsc{#1}}}} % fibre nervose afferenti
\newcommand{\nerm}[1]{\colorbox{Dandelion}{\textcolor{Purple}{\textsc{#1}}}} % fibre nervose miste
\newcommand{\nerdisc}[1]{\colorbox{Dandelion}{\textcolor{Maroon}{\textsc{#1}}}} % fibre nervose efferenti
\newcommand{\nerasc}[1]{\colorbox{Dandelion}{\textcolor{NavyBlue}{\textsc{#1}}}} % fibre nervose afferenti
\newcommand{\nermist}[1]{\colorbox{Dandelion}{\textcolor{Purple}{\textsc{#1}}}} % fibre nervose miste
\newcommand{\nerorto}[1]{\colorbox{Dandelion}{\textcolor{Red}{\textsc{#1}}}} % SN ortosimpatico o simpatico
\newcommand{\nerpara}[1]{\colorbox{Dandelion}{\textcolor{ForestGreen}{\textsc{#1}}}} % SN parasimpatico
\newcommand{\nerent}[1]{\colorbox{Dandelion}{\textcolor{Blue}{\textsc{#1}}}} % SN neurotenterico

% Riferimenti a libri
\newcommand{\gray}[1]{\textsf{ADG, pag. #1}}
\newcommand{\adg}[1]{\textsf{ADG, pag. #1}}
\newcommand{\prom}[1]{\textcolor{NavyBlue}{\textsf{Prometheus, pag. #1}}}
\newcommand{\netter}[1]{ \fbox{\textsf{Netter (2014), tav. #1}} }
\newcommand{\nnetter}[1]{ \fbox{\textsf{Netter (2018), plate #1}} }
\newcommand{\neuronetter}[1]{ \fbox{\textsf{Neuronetter (2010), plate #1}} }

% Tcolorbox
\tcbuselibrary{breakable}
\newcommand{\normalbox}[2]{\begin{tcolorbox}[title=#1]#2\end{tcolorbox}} % Box normale
\newcommand{\simplebox}[2]{\begin{tcolorbox}[title=#1]#2\end{tcolorbox}} % Box normale
\newcommand{\greenbox}[2]{\begin{tcolorbox}[title=#1,colback=green!5,colframe=green!35!black]#2\end{tcolorbox}} % Risvolti utili nella pratica clinica
\newcommand{\warningbox}[2]{\begin{tcolorbox}[title=#1,colback=yellow!5,colframe=yellow!75!red, coltitle=black]#2\end{tcolorbox}} % Risvolti utili nella pratica clinica: importante
\newcommand{\yellowbox}[2]{\begin{tcolorbox}[title=#1,colback=yellow!5,colframe=yellow!75!red, coltitle=black]#2\end{tcolorbox}} % Risvolti utili nella pratica clinica importante
\newcommand{\redbox}[2]{\begin{tcolorbox}[title=#1,colback=red!5,colframe=red!75!black]#2\end{tcolorbox}} % Risvolti utili nella pratica clinica: molta attenzione!
\newcommand{\casoclinico}[3]{\begin{tcolorbox}[title=Caso clinico: {#1},colback=cyan!5,colframe=cyan!75!cyan, coltitle=black]{#2} \tcblower {#3} \end{tcolorbox}} % Caso clinico: presentazione e management
\newcommand{\bluebox}[3]{\begin{tcolorbox}[title={#1},colback=blue!5,colframe=NavyBlue!90!black, coltitle=white]{#2} \tcblower {#3} \end{tcolorbox}} % Rimborsabilità dei farmaci e note aifa
\newcommand{\cyanbox}[2]{\begin{tcolorbox}[title=#1,colback=cyan!5,colframe=cyan!75!cyan, coltitle=black]#2\end{tcolorbox}} % Da usare quando si espone un caso clinico non preciso

% Altra roba
\newcommand{\att}[0]{ $\oplus$ }                                        % Attivazione o regolazione positiva
\newcommand{\down}{$\downarrow$}
\newcommand{\fig}[1]{\textsf{\textbf{Figura \ref{#1}}}}
\newcommand{\goldstandard}{\textcircled{$\star$} }                      % Gold standard (*)
\newcommand{\ini}[0]{ $\otimes$ }                                       % Inibizione o regolazione negativa
\newcommand{\q}[1]{\textcolor{blue}{\textbf{\textsf{Q: #1?}}}}          % Domandona
\newcommand{\s}[1]{\textsc{\textbf{#1}}}                                % Segno o sintomo
\newcommand{\up}{$\uparrow$}
\newcommand{\TODO}[1]{\textcolor{red}{\textsf{\footnotesize{TODO #1}}}} % TODO
\newcommand{\pp}{$\times$} % Per

% Itemize and enumerate
\setlistdepth{9}
\renewlist{itemize}{itemize}{9}
\setlist[itemize,1]{leftmargin=1.2cm, label=\textbullet}
\setlist[itemize,2]{leftmargin=1.2cm, label=\textendash}
\setlist[itemize,3]{leftmargin=1.2cm, label=\textbullet}
\setlist[itemize,4]{leftmargin=1.2cm, label=\textendash}
\setlist[itemize,5]{leftmargin=1.2cm, label=\textasteriskcentered}
\setlist[itemize,6]{leftmargin=1.2cm, label=\textperiodcentered}
\setlist[itemize,7]{leftmargin=1.2cm, label=\textperiodcentered}
\setlist[itemize,8]{leftmargin=1.2cm, label=\textperiodcentered}
\setlist[itemize,9]{leftmargin=1.2cm, label=\textperiodcentered}
\renewlist{enumerate}{enumerate}{9}
\setlist[enumerate,1]{leftmargin=1.2cm, label=$\arabic*.$}
\setlist[enumerate,2]{leftmargin=1.2cm, label=$\alph*.$}
\setlist[enumerate,3]{leftmargin=1.2cm, label=$\roman*.$}
\setlist[enumerate,4]{leftmargin=1.2cm, label=$\arabic*.$}
\setlist[enumerate,5]{leftmargin=1.2cm, label=$\alpha*$}
\setlist[enumerate,6]{leftmargin=1.2cm, label=$\roman*.$}
\setlist[enumerate,7]{leftmargin=1.2cm, label=$\arabic*.$}
\setlist[enumerate,8]{leftmargin=1.2cm, label=$\alph*.$}
\setlist[enumerate,9]{leftmargin=1.2cm, label=$\roman*.$}
