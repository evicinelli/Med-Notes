\PassOptionsToPackage{unicode=true}{hyperref} % options for packages loaded elsewhere
\PassOptionsToPackage{hyphens}{url}
\PassOptionsToPackage{dvipsnames,svgnames*,x11names*}{xcolor}
%
\documentclass[italian,]{article}
\usepackage{lmodern}
\usepackage{amssymb,amsmath}
\usepackage{ifxetex,ifluatex}
\usepackage{fixltx2e} % provides \textsubscript
\ifnum 0\ifxetex 1\fi\ifluatex 1\fi=0 % if pdftex
  \usepackage[T1]{fontenc}
  \usepackage[utf8]{inputenc}
  \usepackage{textcomp} % provides euro and other symbols
\else % if luatex or xelatex
  \usepackage{unicode-math}
  \defaultfontfeatures{Ligatures=TeX,Scale=MatchLowercase}
\fi
% use upquote if available, for straight quotes in verbatim environments
\IfFileExists{upquote.sty}{\usepackage{upquote}}{}
% use microtype if available
\IfFileExists{microtype.sty}{%
\usepackage[]{microtype}
\UseMicrotypeSet[protrusion]{basicmath} % disable protrusion for tt fonts
}{}
\IfFileExists{parskip.sty}{%
\usepackage{parskip}
}{% else
\setlength{\parindent}{0pt}
\setlength{\parskip}{6pt plus 2pt minus 1pt}
}
\usepackage{xcolor}
\usepackage{hyperref}
\hypersetup{
            colorlinks=true,
            linkcolor=CadetBlue,
            filecolor=Maroon,
            citecolor=Blue,
            urlcolor=CadetBlue,
            breaklinks=true}
\urlstyle{same}  % don't use monospace font for urls
\usepackage{longtable,booktabs}
% Fix footnotes in tables (requires footnote package)
\IfFileExists{footnote.sty}{\usepackage{footnote}\makesavenoteenv{longtable}}{}
\usepackage{graphicx,grffile}
\makeatletter
\def\maxwidth{\ifdim\Gin@nat@width>\linewidth\linewidth\else\Gin@nat@width\fi}
\def\maxheight{\ifdim\Gin@nat@height>\textheight\textheight\else\Gin@nat@height\fi}
\makeatother
% Scale images if necessary, so that they will not overflow the page
% margins by default, and it is still possible to overwrite the defaults
% using explicit options in \includegraphics[width, height, ...]{}
\setkeys{Gin}{width=\maxwidth,height=\maxheight,keepaspectratio}
\setlength{\emergencystretch}{3em}  % prevent overfull lines
\providecommand{\tightlist}{%
  \setlength{\itemsep}{0pt}\setlength{\parskip}{0pt}}
\setcounter{secnumdepth}{0}

% set default figure placement to htbp
\makeatletter
\def\fps@figure{htbp}
\makeatother

\PassOptionsToPackage{dvipsnames}{xcolor}

\usepackage[a4paper, left=1.5cm, right=6cm, top=2cm, bottom=2cm, marginpar=4.4cm]{geometry}
\usepackage{chemfig}            % Typeset molecules
\usepackage{fancyhdr}           % Header and footers
\usepackage{float}              % Control how images float
\usepackage{newunicodechar}     % Unicode where it's needed
\usepackage{tcolorbox}          % Boxes
\usepackage{marvosym}
\usepackage{qrcode}
\usepackage{enumitem}
\usepackage{xcolor} % Colors!
\usepackage{libertine, libertinust1math}
\usepackage{titlesec}
\usepackage{nameref}
\usepackage{bookmark}
\usepackage{todonotes}
\usepackage{ragged2e}
\renewcommand{\familydefault}{\sfdefault}

\PassOptionsToPackage{
    activate=true,
    protrusion=true,
    expansion=true,
    final,
    tracking=true,
    kerning=true,
    spacing=true,
%    factor=1100,
%    stretch=10,
%    shrink=10
}{microtype} % Typographic perfection, sadly only with pdftex

% Hyperref
\urlstyle{mono}

% Titlesec {{{
% \titleformat{command}[shape]{format}{label}{sep}{before-code}[after-code]
\titleformat{\part}{\flushright\huge\normalfont\itshape}{\thepart}{1em}{\phantomsection}[\vspace{8em}]  % \part
\titleformat{\chapter}{\huge\normalfont\scshape\lowercase}{\thechapter}{1em}{\phantomsection}           % Book class
\titleformat{\section}{\huge\normalfont\scshape\lowercase}{\thesection}{1em}{\phantomsection}           % # Header #
\titleformat{\subsection}{\LARGE\bfseries}{}{0em}{\phantomsection}                                      % ## Header ##
\titleformat{\subsubsection}{\Large\bfseries}{}{0em}{\phantomsection}                                   % ### Header ###
\titleformat{\paragraph}{\large\bfseries}{}{0em}{\phantomsection}                                       % #### Header ####
\titleformat{\subparagraph}{\normalfont\bfseries}{}{0em}{\phantomsection}                               % ##### Header #####
\titleformat{\subsubparagraph}{\itshape}{}{0em}{\phantomsection}
% }}}

\renewcommand*\ttdefault{txtt}

\setcounter{secnumdepth}{2}  % Number only up to \section
\setcounter{tocdepth}{6}     % Put everything up to \subsubparagraph in toc

\renewcommand\footnoterule{}
\setlength{\skip\footins}{3em}

\newcommand{\marginfig}[1]{\marginpar{{\textsf{Vedi \textbf{Figura \ref{#1}}}}}}
\newcommand{\marginnote}[1]{\marginpar{\footnotesize← \emph{#1}}}
\newcommand{\asidefigure}[2]{\marginpar{\includegraphics{#1}\\\footnotesize\emph{#2}}}

\newunicodechar{½}{$\frac{1}{2}$}
\newunicodechar{¼}{$\frac{1}{4}$}
\newunicodechar{¾}{$\frac{3}{4}$}
\newunicodechar{⅔}{$\frac{2}{3}$}
\newunicodechar{⅓}{$\frac{1}{3}$}
\newunicodechar{¬}{$\neg$}
\newunicodechar{±}{$\pm$}
\newunicodechar{×}{$\times$}
\newunicodechar{÷}{$\div$}
\newunicodechar{…}{$\dots$}
\newunicodechar{ℕ}{$\mathbb{N}$}
\newunicodechar{ℚ}{$\mathbb{Q}$}
\newunicodechar{ℝ}{$\mathbb{R}$}
\newunicodechar{ℤ}{$\mathbb{Z}$}
\newunicodechar{←}{$\leftarrow$}
\newunicodechar{↑}{$\uparrow$}
\newunicodechar{→}{$\rightarrow$}
\newunicodechar{🡪}{$\rightarrow$}
\newunicodechar{↓}{$\downarrow$}
\newunicodechar{↔}{$\leftrightarrow$}
\newunicodechar{⇒}{$\Rightarrow$}
\newunicodechar{⇐}{$\Leftarrow$}
\newunicodechar{⇔}{$\Leftrightarrow$}
\newunicodechar{∀}{$\forall$}
\newunicodechar{∃}{$\exists$}
\newunicodechar{∅}{$\emptyset$}
\newunicodechar{∈}{$\in$}
\newunicodechar{∉}{$\notin$}
\newunicodechar{∋}{$\ni$}
\newunicodechar{∎}{$\blacksquare$}
\newunicodechar{∑}{$\sum$}
\newunicodechar{∓}{$\mp$}
\newunicodechar{∗}{$\ast$}
\newunicodechar{∘}{$\circ$}
\newunicodechar{∙}{$\bullet$}
\newunicodechar{∝}{$\propto$}
\newunicodechar{∞}{$\infty$}
\newunicodechar{∥}{$\parallel$}
\newunicodechar{∧}{$\land$}
\newunicodechar{∨}{$\lor$}
\newunicodechar{∩}{$\cap$}
\newunicodechar{∪}{$\cup$}
\newunicodechar{∴}{$\therefore$}
\newunicodechar{∵}{$\because$}
\newunicodechar{≈}{$\approx$}
\newunicodechar{≠}{$\neq$}
\newunicodechar{≡}{$\equiv$}
\newunicodechar{≤}{$\leq$}
\newunicodechar{≥}{$\geq$}
\newunicodechar{⊂}{$\subset$}
\newunicodechar{⊃}{$\supset$}
\newunicodechar{⊆}{$\subseteq$}
\newunicodechar{⊇}{$\supseteq$}
\newunicodechar{⊢}{$\vdash$}
\newunicodechar{⊤}{$\top$}
\newunicodechar{⊥}{$\bot$}
\newunicodechar{⊨}{$\vDash$}
\newunicodechar{⋅}{$\cdot$}
\newunicodechar{⋮}{$\vdots$}
\newunicodechar{⋯}{$\cdots$}
\newunicodechar{α}{$\alpha$}
\newunicodechar{Α}{$\Alpha$}
\newunicodechar{β}{$\beta$}
\newunicodechar{Β}{$\Beta$}
\newunicodechar{γ}{$\gamma$}
\newunicodechar{Γ}{$\Gamma$}
\newunicodechar{δ}{$\delta$}
\newunicodechar{Δ}{$\Delta$}
\newunicodechar{ε}{$\varepsilon$}
\newunicodechar{Ε}{$\Epsilon$}
\newunicodechar{ζ}{$\zeta$}
\newunicodechar{Ζ}{$\Zeta$}
\newunicodechar{η}{$\eta$}
\newunicodechar{Η}{$\Eta$}
\newunicodechar{θ}{$\theta$}
\newunicodechar{Θ}{$\Theta$}
\newunicodechar{ι}{$\iota$}
\newunicodechar{Ι}{$\Iota$}
\newunicodechar{κ}{$\kappa$}
\newunicodechar{Κ}{$\Kappa$}
\newunicodechar{λ}{$\lambda$}
\newunicodechar{Λ}{$\Lambda$}
\newunicodechar{μ}{$\mu$}
\newunicodechar{Μ}{$\Mu$}
\newunicodechar{∇}{$\nabla$}
\newunicodechar{ν}{$\nu$}
\newunicodechar{Ν}{$N$}
\newunicodechar{ξ}{$\xi$}
\newunicodechar{Ξ}{$\Xi$}
\newunicodechar{ο}{$\omicron$}
\newunicodechar{Ο}{$\Omicron$}
\newunicodechar{π}{$\pi$}
\newunicodechar{Π}{$\Pi$}
\newunicodechar{ρ}{$\rho$}
\newunicodechar{ϱ}{$\varrho$}
\newunicodechar{Ρ}{$\Rho$}
\newunicodechar{σ}{$\sigma$}
\newunicodechar{ς}{$\varsigma$}
\newunicodechar{Σ}{$\Sigma$}
\newunicodechar{τ}{$\tau$}
\newunicodechar{Τ}{$\Tau$}
\newunicodechar{υ}{$\upsilon$}
\newunicodechar{Υ}{$\Upsilon$}
\newunicodechar{φ}{$\varphi$}
\newunicodechar{ϕ}{$\phi$}
\newunicodechar{Φ}{$\Phi$}
\newunicodechar{χ}{$\chi$}
\newunicodechar{Χ}{$\Chi$}
\newunicodechar{ψ}{$\psi$}
\newunicodechar{Ψ}{$\Psi$}
\newunicodechar{ω}{$\omega$}
\newunicodechar{Ω}{$\Omega$}
\newunicodechar{°}{$^{\circ}$}
\newunicodechar{💙}{$\heartsuit$}
\newunicodechar{₀}{~0~}
\newunicodechar{₁}{~1~}
\newunicodechar{₂}{~2~}
\newunicodechar{₃}{~3~}
\newunicodechar{₄}{~4~}
\newunicodechar{₅}{~5~}
\newunicodechar{₆}{~6~}
\newunicodechar{₇}{~7~}
\newunicodechar{₈}{~8~}
\newunicodechar{₉}{~9~}

% Anatomia
\definecolor{ossa}{HTML}{BEAE84}
\newcommand{\mus}[1]{\colorbox{Salmon}{\textcolor{white}{\textsc{#1}}}}
\newcommand{\oss}[1]{\colorbox{ossa}{\textcolor{white}{\textsc{#1}}}}
\newcommand{\ven}[1]{\colorbox{RoyalBlue}{\textcolor{white}{\textsc{#1}}}}
\newcommand{\art}[1]{\colorbox{RedOrange}{\textcolor{white}{\textsc{#1}}}}
\newcommand{\tol}[1]{\colorbox{Aquamarine}{\textcolor{white}{\textsc{#1}}}}
\newcommand{\ner}[1]{\colorbox{Dandelion}{\textcolor{white}{\textsc{#1}}}}
\newcommand{\lin}[1]{\colorbox{PineGreen}{\textcolor{white}{\textsc{#1}}}}
\newcommand{\far}[1]{ \fbox{\textsc{#1}} } % Farmaco (principio attivo)
\newcommand{\farf}[1]{\fbox{\fbox{\textsc{#1}}} } % Famiglia di farmaci
\newcommand{\pat}[1]{\colorbox{black}{\textcolor{white}{\textsc{#1}}}}
\renewcommand{\a}[1]{\underline{\textsc{#1}}}

% Sistema nervoso
\newcommand{\nere}[1]{\colorbox{Dandelion}{\textcolor{Maroon}{\textsc{#1}}}} % fibre nervose efferenti
\newcommand{\nera}[1]{\colorbox{Dandelion}{\textcolor{NavyBlue}{\textsc{#1}}}} % fibre nervose afferenti
\newcommand{\nerm}[1]{\colorbox{Dandelion}{\textcolor{Purple}{\textsc{#1}}}} % fibre nervose miste
\newcommand{\nerdisc}[1]{\colorbox{Dandelion}{\textcolor{Maroon}{\textsc{#1}}}} % fibre nervose efferenti
\newcommand{\nerasc}[1]{\colorbox{Dandelion}{\textcolor{NavyBlue}{\textsc{#1}}}} % fibre nervose afferenti
\newcommand{\nermist}[1]{\colorbox{Dandelion}{\textcolor{Purple}{\textsc{#1}}}} % fibre nervose miste
\newcommand{\nerorto}[1]{\colorbox{Dandelion}{\textcolor{Red}{\textsc{#1}}}} % SN ortosimpatico o simpatico
\newcommand{\nerpara}[1]{\colorbox{Dandelion}{\textcolor{ForestGreen}{\textsc{#1}}}} % SN parasimpatico
\newcommand{\nerent}[1]{\colorbox{Dandelion}{\textcolor{Blue}{\textsc{#1}}}} % SN neurotenterico

% Riferimenti a libri
\newcommand{\gray}[1]{\textsf{ADG, pag. #1}}
\newcommand{\adg}[1]{\textsf{ADG, pag. #1}}
\newcommand{\prom}[1]{\textcolor{NavyBlue}{\textsf{Prometheus, pag. #1}}}
\newcommand{\netter}[1]{ \fbox{\textsf{Netter (2014), tav. #1}} }
\newcommand{\nnetter}[1]{ \fbox{\textsf{Netter (2018), plate #1}} }

% Tcolorbox
\tcbuselibrary{breakable}
\newcommand{\normalbox}[2]{\begin{tcolorbox}[title=#1]#2\end{tcolorbox}} % Box normale
\newcommand{\simplebox}[2]{\begin{tcolorbox}[title=#1]#2\end{tcolorbox}} % Box normale
\newcommand{\greenbox}[2]{\begin{tcolorbox}[title=#1,colback=green!5,colframe=green!35!black]#2\end{tcolorbox}} % Risvolti utili nella pratica clinica
\newcommand{\warningbox}[2]{\begin{tcolorbox}[title=#1,colback=yellow!5,colframe=yellow!75!red, coltitle=black]#2\end{tcolorbox}} % Risvolti utili nella pratica clinica: importante
\newcommand{\yellowbox}[2]{\begin{tcolorbox}[title=#1,colback=yellow!5,colframe=yellow!75!red, coltitle=black]#2\end{tcolorbox}} % Risvolti utili nella pratica clinica importante
\newcommand{\redbox}[2]{\begin{tcolorbox}[title=#1,colback=red!5,colframe=red!75!black]#2\end{tcolorbox}} % Risvolti utili nella pratica clinica: molta attenzione!
\newcommand{\casoclinico}[3]{\begin{tcolorbox}[title=Caso clinico: #1,colback=cyan!5,colframe=cyan!75!cyan, coltitle=black, list inside=clinic]#2 \tcblower #3 \end{tcolorbox}} % Caso clinico: presentazione e management
\newcommand{\cyanbox}[2]{\begin{tcolorbox}[title=#1,colback=cyan!5,colframe=cyan!75!cyan, coltitle=black]#2\end{tcolorbox}} % Da usare quando si espone un caso clinico non preciso

% Altra roba
\newcommand{\att}[0]{ $\oplus$ }                                        % Attivazione o regolazione positiva
\newcommand{\down}{$\downarrow$}
\newcommand{\fig}[1]{\textsf{\textbf{Figura \ref{#1}}}}
\newcommand{\goldstandard}{\textcircled{$\star$} }                      % Gold standard (*)
\newcommand{\ini}[0]{ $\otimes$ }                                       % Inibizione o regolazione negativa
\newcommand{\q}[1]{\textcolor{blue}{\textbf{\textsf{Q: #1?}}}}          % Domandona
\newcommand{\sint}[1]{\textsf{#1}}                                      % Segno o sintomo
\newcommand{\sos}[1]{\textsf{#1}}                                       % segno o sintomo
\newcommand{\up}{$\uparrow$}
\newcommand{\TODO}[1]{\textcolor{red}{\textsf{\footnotesize{TODO #1}}}} % TODO

% Itemize and enumerate
\setlistdepth{9}
\renewlist{itemize}{itemize}{9}
\setlist[itemize,1]{leftmargin=1.2cm, label=\textbullet}
\setlist[itemize,2]{leftmargin=1.2cm, label=\textendash}
\setlist[itemize,3]{leftmargin=1.2cm, label=$\circ$}
\setlist[itemize,4]{leftmargin=1.2cm, label=\textasteriskcentered}
\setlist[itemize,5]{leftmargin=1.2cm, label=$\diamond$}
\setlist[itemize,6]{leftmargin=1.2cm, label=\textperiodcentered}
\setlist[itemize,7]{leftmargin=1.2cm, label=\textperiodcentered}
\setlist[itemize,8]{leftmargin=1.2cm, label=\textperiodcentered}
\setlist[itemize,9]{leftmargin=1.2cm, label=\textperiodcentered}
\renewlist{enumerate}{enumerate}{9}
\setlist[enumerate,1]{leftmargin=1.2cm, label=$\arabic*.$}
\setlist[enumerate,2]{leftmargin=1.2cm, label=$\alph*.$}
\setlist[enumerate,3]{leftmargin=1.2cm, label=$\roman*.$}
\setlist[enumerate,4]{leftmargin=1.2cm, label=$\arabic*.$}
\setlist[enumerate,5]{leftmargin=1.2cm, label=$\alpha*$}
\setlist[enumerate,6]{leftmargin=1.2cm, label=$\roman*.$}
\setlist[enumerate,7]{leftmargin=1.2cm, label=$\arabic*.$}
\setlist[enumerate,8]{leftmargin=1.2cm, label=$\alph*.$}
\setlist[enumerate,9]{leftmargin=1.2cm, label=$\roman*.$}
\ifnum 0\ifxetex 1\fi\ifluatex 1\fi=0 % if pdftex
  \usepackage[shorthands=off,main=italian]{babel}
\else
  % load polyglossia as late as possible as it *could* call bidi if RTL lang (e.g. Hebrew or Arabic)
  \usepackage{polyglossia}
  \setmainlanguage[]{italian}
\fi

\date{}

\begin{document}

\newgeometry{top=4cm, bottom=4cm, left=4cm, right=4cm}

\title{Splancnoanatomia}
\author{Emanuele Vicinelli}
\date{a.a. 2019/2020}

\maketitle

\begin{center}\rule{0.5\linewidth}{0.5pt}\end{center}

\begin{figure}[H]
\vspace{2cm}
\centering
\includegraphics[width=8cm]{../head.pdf}
\end{figure}

\thispagestyle{fancy}
\fancyhead{}
\fancyfoot{}
\renewcommand{\headrulewidth}{0pt}
\rfoot{\today}

\newpage

\hypertarget{legenda}{%
\subsection*{Legenda}\label{legenda}}
\addcontentsline{toc}{subsection}{Legenda}

\begin{itemize}
\item
  \texttt{FGF}: molecola, ormone secreto o sostanza chimica
\item
  \a{parte anatomica generica}
\item
  \ner{strutture nervose}
\item
  \lin{strutture sistema linfatico}
\item
  \oss{ossa}
\item
  \mus{muscoli}
\item
  \art{vasi arteriosi}
\item
  \ven{vasi venosi}
\item
  \tol{strutture tendinee, legamentose o connettivali}
\item
  \far{farmaci}
\item
  \pat{patologie}
\end{itemize}

\hypertarget{testi-utilizzati}{%
\subsection*{Testi utilizzati}\label{testi-utilizzati}}
\addcontentsline{toc}{subsection}{Testi utilizzati}

\begin{itemize}
\tightlist
\item
  \netter{21}: tavola 21 del \emph{Netter, Atlante di Anatomia
  Umana}\textsuperscript{{[}\protect\hyperlink{ref-netter2014atlas}{1}{]}}
\item
  \nnetter{21}: tavola 21 del \emph{Netter, Atlante di Anatomia
  Umana}\textsuperscript{{[}\protect\hyperlink{ref-netter2018atlas}{2}{]}}
\item
  \gray{134}: pagina 134 dell'\emph{Anatomia del
  Gray}\textsuperscript{{[}\protect\hyperlink{ref-gray2017anatomia}{3}{]}}
\item
  \prom{212}: pagina 212 dell'atlante di anatomia
  Prometheus\textsuperscript{{[}\protect\hyperlink{ref-schunke2010thieme}{4}{]}}.
  Da qui vengono anche quasi tutte le immagini (se ci sono)
\end{itemize}

\restoregeometry

\newpage

\tableofcontents

\newpage

\listoffigures

\newpage

\part{Regione della testa e del collo}

\hypertarget{cavita-orale}{%
\section{Cavita' orale}\label{cavita-orale}}

\begin{itemize}
\tightlist
\item
  Localizzazione: porzione inferiore della faccia, sotto alla cavità
  nasale. Di forma ovoidale, ha 6 pareti

  \begin{enumerate}
  \def\labelenumi{\arabic{enumi}.}
  \tightlist
  \item
    Parete anteriore (labbra)
  \item
    Parete posteriore (palato molle + istmo delle fauci)
  \item
    Pareti laterale dx (guancia)
  \item
    Parete laterale sx (guancia)
  \item
    Parete superiore (volta palatina
  \item
    Parete inferiore (muscolo \mus{milojoideo})
  \end{enumerate}
\item
  La collaborazione delle due \a{arcate gengivo-dentali} suddividono la
  cavità in due regioni:

  \begin{enumerate}
  \def\labelenumi{\arabic{enumi}.}
  \tightlist
  \item
    Una regione anteriore alle arcate: \a{vestibolo della bocca}
  \item
    Una regione posteriore alle arcate: \a{cavità orale propria}
  \end{enumerate}

  \begin{itemize}
  \tightlist
  \item
    Vestibolo e cavità comunicano con \a{spazio retromolare} e
    \a{spazi interdentali}
  \end{itemize}
\item
  Dimensioni principali

  \begin{itemize}
  \tightlist
  \item
    Diametro AP (\textasciitilde{} 7 cm) \textgreater{}\textgreater{}
    diametro LM (\textasciitilde{} 5.5 cm)
  \item
    Diametro vert a bocca chiusa: \textasciitilde{} 2.5 cm
  \end{itemize}
\item
  Caratteristica generale di tutte le regioni della faccia: cute e
  sottocute sono molto uniti

  \begin{itemize}
  \tightlist
  \item
    Cute
  \item
    Sottocute
  \item
    SMAS (Sistema Muscolo-Aponeurotico Superficiale) -- sistema di
    aponeurosi che serve da fare ancoraggio per il sottostante strato
    dei muscoli mimici con lo strato del sottocute. È lo strato che
    permette che l'azione dei muscoli mimici abbia una conseguenza
    superficiale. Presenta fibre connettivali dense molto robuste, a
    tratti con caratteristiche simil-legamentose
  \item
    Fascia
  \item
    Muscoli
  \item
    Periostio
  \end{itemize}
\end{itemize}

\greenbox{Innervazione e dermatomeri del viso}{

\begin{figure}[H]
\centering
\includegraphics[width=\textwidth]{img/innervazione-viso.png}
\end{figure}

Tutta l'innervazione sensitiva della faccia è di competenza del \ner{trigemino}.
Individuiamo 3 regioni

\begin{itemize}
\tightlist
\item
  \ner{nervo oftalmico} (\textcolor{JungleGreen}{verde}) -- apice del
  naso, si porta posteriormente escludendo le pinne nasali verso
  palpebra inferiore, passa poco posteriormente allo \a{pterion} e si
  conclude nella linea immaginaria verticale che passa dall'orecchio
\item
  \ner{nervo mascellare} (\textcolor{Salmon}{rosa}) -- rima buccale,
  verso angolo mandibolare, per risalire sempre verso lo pterion
\item
  \ner{nervo mandibolare} (\textcolor{Dandelion}{giallo}) -- mandibola,
  angolo, parte anteriore del lobo, vertice della linea orecchio-mento
\end{itemize}

Le regioni non sono delimitate in maniera netta: i confini sono sfumati
e di competenza di 2 nervi

}

\hypertarget{limiti-e-pareti}{%
\subsection{Limiti e pareti}\label{limiti-e-pareti}}

\hypertarget{parete-anteriore-labbra}{%
\subsubsection{Parete anteriore: labbra}\label{parete-anteriore-labbra}}

\begin{itemize}
\tightlist
\item
  Labbra: 2 pliche \textbf{muscolo-membranose} che contornano la cavità
  orale

  \begin{enumerate}
  \def\labelenumi{\arabic{enumi}.}
  \tightlist
  \item
    Superficie esterna cutanea
  \item
    SMAS, muscolatura mimica e fascia
  \item
    Superficie interna mucosale
  \end{enumerate}
\item
  \textbf{Superficie esterna cutanea}

  \begin{itemize}
  \tightlist
  \item
    Parte rossa: \a{vermiglio} (superiore o inferiore)
  \item
    Medialmente al \a{vermiglio} troviamo un solco che scende
    verticalmente dal \a{setto nasale} al \a{labbro superiore} -- il
    \a{filtro} (o \a{prolabio})
  \item
    \a{solco naso-labiale} che corre diagonalmente dalle pareti laterali
    delle narici alla \a{commissura labiale}\footnote{Punto di
      congiunzione delle due labbra}
  \item
    Inferiormente al labbro inferiore, in posizione mediale corre
    orizzontalmente la \a{fossetta mentale}
  \item
    \a{solco mento-labiale} che parte dalle commissure labiali e va
    diagonalmente al mento
  \end{itemize}
\item
  \textbf{Muscolatura mimica labiale} è garantita dai muscoli
  \mus{orbicolari}, che hanno la funzione fisiologica di
  \textbf{costringere} labbra\footnote{ma anche occhi e la superficie
    circostante le orecchie, in quanto sono presenti anche lì muscoli
    orbicolari}, collaborando all'espressione facciale. L'aderenza della
  parte sovrastante al muscolo è garantita dallo SMAS

  \begin{itemize}
  \tightlist
  \item
    Divisione in 4 quadranti (2 superiori, 2 inferiori) convergenti
    nella commissura labiale\footnote{dove arrivano anche \textbf{altri
      muscoli mimici}

      \begin{itemize} \tightlist \item \mus{risorio} \item \mus{zigomatico} \item \mus{elevatore} e \mus{depressore} del labbro superiore \end{itemize}}
  \end{itemize}
\item
  \textbf{Superficie interna mucosa} (\netter{56})

  \begin{itemize}
  \tightlist
  \item
    Il labbro superiore è separato dall'arcata gengivo-dentale tramite
    il \a{solco labiale}. Il solco prosegue posteriormente, e la sua
    prosecuzione, che conclude in questo modo l'arco a ferro di cavallo,
    è detto \a{fornice}
  \item
    La mucosa dell'arcata gengivale, in posizione mediana, si continua
    nella superficie interna formando due pieghe, dette \a{frenuli}
  \item
    La mucosa dell'arcata gengivale è maggiormente cheratinizzata
    rispetto ad una mucosa classica (epitelio
    \emph{ortocheratinizzato}). Avviandosi (senza raggiungerla) verso
    una cheratinizzazione completa, risulta meno rosea della mucosa
    buccale
  \end{itemize}
\end{itemize}

\hypertarget{vascolarizzazione-e-innervazione}{%
\paragraph{Vascolarizzazione e
innervazione}\label{vascolarizzazione-e-innervazione}}

\begin{itemize}
\tightlist
\item
  \textbf{Vascolarizzazione}: da \art{arteria faciale})\footnote{Ramo
    anteriore della \art{carotide esterna}, che si stacca dopo che
    questa ha ceduto prima la \art{tiroidea superiore} e la
    \art{linguale}}. Si forma un circolo anastomotico arterioso: la
  \art{faciale} scavalca da dentro a fuori la mandibola, risale verso le
  labbra e cede \marginnote{\netter{72--73}\\ \nnetter{83}}

  \begin{itemize}
  \item
    \art{labiale inferiore}
  \item
    \art{labiale superiore}
  \end{itemize}

  Che fanno anastomosi con le controlaterali e generano il circolo che
  provvede alla vascolarizzazione dell'area labiale
\item
  \textbf{Innervazione}: \ner{mascellare} per labbro superiore (tramite
  il ramo \ner{infraorbitario}), \ner{mandibolare} per labbro inferiore
  (tramite il ramo \ner{mentale})
\end{itemize}

\hypertarget{pareti-laterali-guance}{%
\subsubsection{Pareti laterali: guance}\label{pareti-laterali-guance}}

\begin{itemize}
\tightlist
\item
  \textbf{Lato cutaneo}

  \begin{itemize}
  \tightlist
  \item
    Cute e sottocute molto adesi, come in tutta la faccia
  \item
    Ritroviamo (nei lattanti) un corpo adiposo sviluppato
    (\a{bolla di bichat}), che si pensa abbia il ruolo di favorire la
    suzione. Quando il lattante viene svezzato, la bolla si sposta verso
    l'alto verso l'arcata zigomatica
  \end{itemize}
\item
  \textbf{Piano muscolare}

  \begin{itemize}
  \tightlist
  \item
    Il corpo carnoso della guancia è dato dai muscoli: il principale è
    il \mus{buccinatore}\footnote{lamina sottile, con inserzione a forma
      di U. Raggiunge posteriormente (attaccandosi) il
      \a{rafe pterigomandibolare} per tutta la lunghezza. Nel margine
      inferiore, raggiunge l'\a{arcata alveolare inferiore} per poi,
      anteriormente, dividere le sue fibre: le superiori vanno al labbro
      inferiore e le inferiori vanno al labbro superiore, convergendo
      sul \mus{orbicolare}.\\
      \textbf{Azione}: irrigidimento della guancia (è quel muscolo che,
      se non funziona, ha la colpa del morso alla guancia quando si
      mastica)}
  \end{itemize}
\item
  \textbf{Lato mucoso}

  \begin{itemize}
  \tightlist
  \item
    A livello della corona del II molare superiore troviamo l'orifizio
    del \a{dotto di stenone} (della \a{ghiandola parotide}) -- che è
    anche l'unica struttura degna di nota del vestibolo
  \item
    Vediamo anche il \a{rafe pterigomandibolare}, che costituisce il
    limite posteriore delle guance
  \end{itemize}
\end{itemize}

\hypertarget{vascolarizzazione-arteriosa}{%
\paragraph{Vascolarizzazione
arteriosa}\label{vascolarizzazione-arteriosa}}

\begin{itemize}
\tightlist
\item
  Superficiale: \art{trasversa della faccia}\footnote{L'arteria faciale
    trasversa è un ramo che si stacca trasversalmente dall'arteria
    \art{temporale superficiale}, terminazione della
    \art{carotide esterna}. Questa sale, entra nel parenchima della
    \a{parotide} e, a livello di \a{condilo mandibolare} (punto di
    repere per il polso temporale) diventa \emph{temporale superficiale}
    -- la quale va verso l'alto e stacca subito la
    \art{trasversale della faccia}. Successivamente la temporale
    superficiale stacca rami trasversali che provvedono alla
    vascolarizzazione della zona temporale (\art{temporale media} e
    \art{zigomatico-orbitaria}) per concludersi con i rami terminali
    dell'arteria \art{frontale} e \art{parietale}}, con il contributo di
  rami dall'arteria \art{faciale} e \art{infraorbitale} (\netter{72})
\item
  Profondo: arteria \art{buccale} (o buccinatoria), ceduta dall'arteria
  mascellare (\netter{72})
\end{itemize}

\hypertarget{section}{%
\subparagraph{\texorpdfstring{\art{mascellare interna}}{}}\label{section}}

\begin{itemize}
\tightlist
\item
  \netter{51A, 51B} (per visione focalizzata), \netter{72} (per visione
  d'insieme della vascolarizzazione della regione testa/collo)
\item
  Si stacca dalla \art{carotide esterna}, costruendo un bivio dopo il
  quale la carotide prosegue come \emph{temporale superficiale}

  \begin{itemize}
  \tightlist
  \item
    Corre nella fossa infratemporale\footnote{Quindi coperta dall'arcata
      della mandibola} \textbf{verso l'avanti e medialmente},
    approfondandosi \emph{sotto} lo strato muscolare della mandibola con
    andamento \emph{sinuoso}
  \item
    Ha rapporto molto stretto con la \oss{manidibola}, tenuta aderente a
    questa dal \tol{legamento sfenomandibolare}
  \item
    Corre prima sopra e poi nel pertugio tra i due capi del
    \mus{pteriogideo esterno}, addentrandosi nella
    \a{fossa pterigopalatina} (\netter{54})
  \item
    Entra nel \a{foro sfenopalatino}, per raggiungere le cavità nasali
    posteriori e le vascolarizza con il suo ramo terminale
    (\emph{arteria sfeno-palatina})
  \end{itemize}
\item
  Durante il suo decorso stacca \textbf{14 rami} (direzione dall'origine
  verso il termine):

  \begin{figure}[H]\centering \includegraphics[width=15cm]{img/vascolarizzazione-cavità-nasali-labbro.png}\end{figure}

  ~

  \begin{itemize}
  \tightlist
  \item
    Ascendenti (5)

    \begin{enumerate}
    \def\labelenumi{\arabic{enumi}.}
    \tightlist
    \item
      \art{auricolare profonda}, che si distribuisce al canale acustico~
    \item
      \art{timpanica}, che sale verso l'alto e passa nella
      \a{scissura di glasser}, per dirigersi verso l'orecchio medio~
    \item
      \art{meningea media}, che sale ed entra nella cavità cranica
      tramite il \a{foro spinoso} dello \oss{sfenoide}. Si distribuisce
      alla \tol{dura madre} e al periostio~
    \item
      \art{temporale profonda anteriore}, che, insieme alla
      \art{temporale media}, vascolarizza il \mus{temporale}~
    \item
      \art{temporale profonda posteriore}~
    \end{enumerate}
  \item
    Discendenti (5)

    \begin{enumerate}
    \def\labelenumi{\arabic{enumi}.}
    \setcounter{enumi}{5}
    \tightlist
    \item
      \art{alveolare inferiore}, che penetra nel \a{canale alveolare}
      per vascolarizzare l'arcata dentale inferiore
    \item
      \art{masseterina}, che si porta al \mus{massetere}, passando tra
      il collo e il processo coronoideo
    \item
      Rami per i \mus{pterigoidei} (2 o 3)
    \item
      \art{buccale} (o buccinatoria), che si adagia sul buccinatore,
      rilasciando rami profondi per la guancia
    \item
      \art{palatina discendente}, che si impegna nel canale
      pterigopalatino e, all'interno di esso, si divide nella
      \art{palatina maggiore} (che attraversa il canale incisivo e si
      anastomizza con la sfenopalatina) e in un altro ramo, che
      vascolarizza il velo palatino
    \end{enumerate}
  \item
    Anteriori (2)

    \begin{enumerate}
    \def\labelenumi{\arabic{enumi}.}
    \setcounter{enumi}{10}
    \tightlist
    \item
      \art{alveolare posterosuperiore}: si adagia sulla tuberosità del
      mascellare, penetrando nei forellini della tuberosità,
      partecipando in questo modo a vascolarizzare i denti
      posterosuperiori
    \item
      \art{arteria infraorbitale}, che prosegue verso l'avanti e
      intramuralizza, uscendo livello di \a{foro intraorbitarior}
    \end{enumerate}
  \item
    Posteriori (2)

    \begin{enumerate}
    \def\labelenumi{\arabic{enumi}.}
    \setcounter{enumi}{11}
    \item
      \art{arteria del canale pterigoideo}
    \item
      \art{arteria faringea}, diretta alla rinofaringe
    \end{enumerate}
  \item
    Terminale

    \begin{itemize}
    \tightlist
    \item
      \art{sfenopalatina}, che attraversa il \a{foro sfenopalatino} per
      andare a vascolarizzare le cavità nasali (cfr)
    \end{itemize}
  \end{itemize}
\end{itemize}

\hypertarget{ritorno-venoso}{%
\paragraph{Ritorno venoso}\label{ritorno-venoso}}

\begin{itemize}
\item
  \netter{73}
\item
  Superficiale: a carico prevalentemente di

  \begin{itemize}
  \tightlist
  \item
    Vena \ven{faciale}, che raccoglie \ven{labiale} sup e inf
  \item
    \ven{trasversa della faccia}: decorre parallela
    all'\a{arcata zigomatica}. Incontra massetere, superficie laterale
    della guancia, sottocutaneo. Termina nella
    \ven{temporale superficiale} (diventerà tronco retromandibolare),
    che discende e afferisce alla \ven{carotide}
  \end{itemize}
\item
  Profondo: di competenza prevalentemente del \ven{plesso pterigoideo}
  \textgreater{}\ven{tronco retromandibolare} \textgreater{}
  \ven{giugulare interna} (o \a{angolo giugulo-succlavio} per una parte)

  \begin{itemize}
  \tightlist
  \item
    Rete venosa ampia, che si adagia sul \mus{pterigoideo esterno}
  \item
    Raccoglie il refluo profondo della faccia, ma anche il ritorno dalle
    cavità nasali
  \item
    Confluisce in due \ven{vene mascellari}, che si uniscono tra loro e
    terminano anche loro nella \ven{temporale superficiale}, costituendo
    in questa maniera il \ven{tronco retromandibolare}
    \marginnote{Che si trova proprio dietro l'angolo della mandibola}
  \item
    Il tronco retromandibolare si divide in due

    \begin{itemize}
    \tightlist
    \item
      Un ramo prosegue verso il basso, unendosi con la \ven{faciale} e
      con la \ven{linguale}\footnote{Talvolta anche con la
        \ven{laringea superiore}}, andando a costituire un
      \emph{grosso} tronco comune; questo tronco si apre nella
      \ven{giugulare interna}
    \item
      Un ramo, più superficiale, si unisce con la
      \ven{vena auricolare posteriore}, andando in questo modo a formare
      la \ven{giugulare esterna}, che termina
      nell'\a{angolo giugulo-succlavio}
    \end{itemize}
  \end{itemize}
\end{itemize}

\hypertarget{innervazione}{%
\paragraph{Innervazione}\label{innervazione}}

\begin{itemize}
\tightlist
\item
  \textbf{Innervazione} -- a cavallo di due regioni

  \begin{itemize}
  \tightlist
  \item
    Sensitiva: di competenza di rami del \ner{trigemino}

    \begin{itemize}
    \tightlist
    \item
      La parte più \textbf{anteriore cutanea}, dal lato buccale, sarà di
      competenza del \ner{mascellare} (ramo \emph{infraorbitario} e
      \emph{zigomatico-facciale})
    \item
      La parte più \textbf{posteriore cutanea} sarà di competenza del
      \ner{mandibolare} (ramo \emph{auricolo-temporale} e
      \emph{buccale})
    \item
      La parte mucosa è sempre di competenza del \ner{buccale}, ramo del
      \ner{mandibolare}\footnote{a sua volta staccato dal
        \ner{trigemino}}
    \end{itemize}
  \item
    Motoria: tutto di competenza del nervo \ner{faciale}
  \end{itemize}
\end{itemize}

\hypertarget{parete-superiore-volta-palatina-palato-duro}{%
\subsubsection{Parete superiore: volta palatina (palato
duro)}\label{parete-superiore-volta-palatina-palato-duro}}

\begin{itemize}
\tightlist
\item
  Rivestita dalla stessa \textbf{mucosa masticatoria} con epitelio
  ortocheratinizzato \textbackslash{}marginnote\{Durante la masticazione
  l'insulto meccanico che la mucosa deve sopportare è tale e quale a
  quello delle gengive
\item
  In posizione mediana troviamo una cresta, che termina con la
  \a{papilla incisiva}: mucosa che ricopre il \a{canale incisivo}

  \begin{itemize}
  \tightlist
  \item
    Dalla cresta mediana partono delle creste trasversali
  \item
    Nella sottomucosa troviamo, sparse ovunque e piccole,
    \a{ghiandole salivari accessorie} (⇒ mucosa non liscia)
  \end{itemize}
\item
  \textbf{Vascolarizzazione} (\netter{57})

  \begin{itemize}
  \tightlist
  \item
    Di competenza della \art{palatina maggiore}\footnote{\art{mascellare interna}
      \textgreater{} \art{palatina maggiore} \textgreater{} foro
      palatino maggiore}, la quale si dirige verso il canale incisivo ed
    effettua anastomosi con \art{sfenopalatina}
  \end{itemize}
\item
  \textbf{Innervazione}

  \begin{itemize}
  \tightlist
  \item
    Da due rami del \ner{mascellare}, che scendono nel
    \a{foro palatino maggiore} ed emettono rami fino a livello dei
    canini

    \begin{itemize}
    \item
      \ner{palatino maggiore}
    \item
      \ner{sfenopalatino}
    \end{itemize}
  \item
    La parte anteriormente ai canini viene innervata sempre dal ramo
    dello sfenopalatino, ma con un ramo che passa dal
    \a{canale incisivo}
  \end{itemize}
\end{itemize}

\hypertarget{parete-posteriore-velo-palatino-palato-molle-e-istmo-delle-fauci}{%
\subsubsection{Parete posteriore: velo palatino (palato molle) e istmo
delle
fauci}\label{parete-posteriore-velo-palatino-palato-molle-e-istmo-delle-fauci}}

\begin{itemize}
\tightlist
\item
  In continuità con palato duro, ma incompleto posteriormente (si apre
  sulla orofaringe)
\item
  Di carattere muscolomembranoso (palato \emph{molle}), quindi si muove

  \begin{itemize}
  \tightlist
  \item
    Il movimento di verticalizzazione è fondamentale per chiudere
    posteriormente la cavità orale, permettendo la continuità della
    nasofaringe con l'orofaringe e la laringe
    \marginnote{Questo si evidenzia bene nella suzione, quando questa chiusura è permessa dall'abbassamento della radice della lingua}
  \item
    Il movimento orizzontale è fondamentale per chiudere il passaggio
    tra rinofaringe e orofaringe, permettendo continuità dell'orofaringe
    con la faringe, e in conseguenza anche con l'esofago
  \end{itemize}
\item
  Ha un tratto limitato orizzontale, poi piega verso il basso. Ha un
  lato concavo (faccia linguale) e un lato convesso (faccia
  rinofaringea) \marginnote{\netter{64--65}}
\item
  In visione anteriore, a bocca aperta, è ben visibile la parte
  verticale

  \begin{itemize}
  \tightlist
  \item
    Vediamo medialmente l'\a{ugola}
  \item
    Lateralmente all'ugola partono due ordini di archi, gli
    \a{archi palatoglossi} (delimitazione dell'\a{istmo delle fauci} e
    termine della cavità orale)
  \item
    Subito posteriormente vediamo altri due ordini di archi, detti
    \a{archi palatofaringei}
  \item
    Tra le due coppie di archi troviamo la \a{fossa tonsillare}
  \end{itemize}
\item
  \textbf{Struttura}

  \begin{itemize}
  \tightlist
  \item
    Mucosa (in continuità con quella del palato molle)
  \item
    Sottomucosa
  \item
    Aponeurosi palatina -- si fissa medialmente sulla spina nasale
    posteriore e sulla lamina orizzontale dell'osso palatino, fino
    all'uncino pterigoideo. Medialmente presenta una fessura nella quale
    passerà l'\mus{ugola}
  \item
    Muscoli del velo palatino (\netter{57B})
  \end{itemize}
\end{itemize}

\hypertarget{muscoli-del-velo-palatino}{%
\paragraph{Muscoli del velo palatino}\label{muscoli-del-velo-palatino}}

\begin{itemize}
\tightlist
\item
  \mus{tensore del velo palatino}~

  \begin{itemize}
  \tightlist
  \item
    \textbf{Origine}: lamina laterale della \a{tuba uditiva} e
    \a{fossetta scafoidea}\footnote{Margine della lamina mediale del
      processo pterigoideo}
  \item
    \textbf{Inserzione}: \tol{aponeurosi palatina}
  \item
    \textbf{Azione}: Decorso a L ruotata. Scende verticalmente, circonda
    ad angolo l'\a{uncino pterigoideo} e da lì prosegue orizzontalmente,
    attaccandosi all'aponeurosi ⇒ contraendosi \emph{tende} il velo
    palatino, siccome l'angolo fatto attorno all'uncino non gli permette
    di alzarsi. Facendo punto fisso sull'angolo dell'uncino, tira anche
    la \a{tuba uditiva}, aprendola e permettendo l'ingresso dell'aria
  \end{itemize}
\item
  \mus{elevatore del velo palatino}~

  \begin{itemize}
  \tightlist
  \item
    \textbf{Origine}: base cranica, nella parte anteriore al
    \a{foro carotico esterno} della \a{piramide} del \oss{temporale}. Si
    dirige obliquamente verso il basso
  \item
    \textbf{Inserzione}: \tol{aponeurosi palatina}, con la quale si
    intreccia strettamente
  \item
    \textbf{Azione}: garantisce l'azione di sollevamento del velo
    palatino, siccome è legato superiormente all'aponeurosi, e questa è
    rigida
  \end{itemize}
\item
  \mus{palatofaringeo}~

  \begin{itemize}
  \tightlist
  \item
    \textbf{Origine}: \tol{aponeurosi palatina}. Di fatto costituisce
    l'\a{arco palatofaringeo} del lato in cui si trova
  \item
    \textbf{Inserzione}: si congiunge con le fibre del muscolo
    \mus{salpingofaringeo}
  \item
    \textbf{Azione}: abbassamento o innalzamento dell'aponeurosi
    palatina
  \end{itemize}
\item
  \mus{palatoglosso} (\netter{65})~

  \begin{itemize}
  \tightlist
  \item
    \textbf{Origine}: \tol{aponeurosi palatina}, subito anteriormente a
    quella del palatofaringeo (vedi \netter{54A} e confronto con la
    54B). Di fatto costituisce l'\a{arco palatoglosso} del lato in cui
    si trova
  \item
    \textbf{Inserzione}: margine laterale del dorso della lingua
  \item
    \textbf{Azione}: abbassamento del palato molle e contemporaneo
    innalzamento della radice della lingua
  \end{itemize}
\item
  \mus{ugola}~

  \begin{itemize}
  \tightlist
  \item
    Sono 2 muscoli che si accostano medialmente
  \item
    \textbf{Azione}: contrazione ⇒ ingrossamento ⇒ interruzione della
    continuità tra nasofaringe e orofaringe
  \end{itemize}
\end{itemize}

\hypertarget{vascolarizzazione-e-innervazione-1}{%
\paragraph{Vascolarizzazione e
innervazione}\label{vascolarizzazione-e-innervazione-1}}

\begin{itemize}
\tightlist
\item
  \textbf{Vascolarizzazione}

  \begin{enumerate}
  \def\labelenumi{\arabic{enumi}.}
  \tightlist
  \item
    \art{palatina discendente} (\textless{} mascellare) --- fuoriesce
    dal canale pterigopalatino
  \item
    \art{arteria faringea ascendente} (\textless{} carotide esterna)
  \item
    Rami minori (principalmente \art{palatina ascendente}\footnote{Ramo
      dell'arteria faciale che risale aderente ai muscoli della laringe,
      vedi \netter{68B}})
  \end{enumerate}
\item
  \textbf{Innervazione}

  \begin{itemize}
  \tightlist
  \item
    L'innervazione sensitiva è di competenza del \ner{mascellare}

    \begin{itemize}
    \tightlist
    \item
      Ramo \emph{palatino minore} per la mucosa
    \end{itemize}
  \item
    L'innervazione motoria è di competenza del \ner{plesso faringeo},
    composto da \ner{vago} e \ner{accessorio del vago}\footnote{Che ci
      sta, perché di fatto è un movimento primariamente
      \textbf{riflesso}: inizialmente si decide volontariamente di
      deglutire, successivamente si ha il riflesso protettivo della
      pervietà delle vie composto dall'abbassamento della glottide e
      dall'orizzontalizzazione del palato molle}
  \end{itemize}
\end{itemize}

\hypertarget{parete-inferiore-pavimento}{%
\subsubsection{\texorpdfstring{Parete inferiore (pavimento)
\label{lblmuscolisovraioidei}}{Parete inferiore (pavimento) }}\label{parete-inferiore-pavimento}}

\begin{itemize}
\tightlist
\item
  Costituita principalmente da muscolatura tesa tra due archi ossei:
  \marginnote{\netter{58}}

  \begin{itemize}
  \tightlist
  \item
    Anteriore: corpo della \oss{mandibola}
  \item
    Posteriore: \oss{ioide}. Ora, lo ioide è ancorato da legamenti e
    muscoli, che sono quelli che lo stabilizzano
  \end{itemize}
\item
  I muscoli del pavimento sono 4 (collettivamente definiti come
  \emph{muscoli sovraioidei}) \marginnote{Direizone: ext → int}

  \begin{enumerate}
  \def\labelenumi{\arabic{enumi}.}
  \tightlist
  \item
    \mus{digastrico} --- muscolo esterno costituito da 2 ventri carnosi
    in serie separati da un tendine intermedio

    \begin{itemize}
    \tightlist
    \item
      Ventre anteriore

      \begin{itemize}
      \tightlist
      \item
        \textbf{Origine}: \a{solco digastrico}\footnote{Solco compreso
          tra il processo mastoideo del \oss{temporale} e tra il solco
          lasciato dall'arteria \art{occipitale}}
      \item
        \textbf{Inserzione}: si dirige avanti e verso il basso per
        concludersi nel tendine intermedio
      \item
        \textbf{Azione}: a seconda del punto fisso: abbassa la mandibola
        o alza lo ioide
        \marginnote{Ricordiamoci che il corpo della mandibola è più in alto dell'osso ioide}
      \end{itemize}
    \item
      Ventre posteriore

      \begin{itemize}
      \tightlist
      \item
        \textbf{Origine}: tendine intermedio
      \item
        \textbf{Inserzione}: \a{fossetta digastrica} nel margine
        inferiore del corpo della \oss{mandibola}
      \item
        \textbf{Azione}: a seconda del punto fisso: alza lo ioide o
        promuove l'iperestensione del capo
      \end{itemize}
    \item
      Il tendine comune non è fissato direttamente dallo ioide, ma è
      solidale all'osso mediante un'ansa fibrosa che parte dallo ioide,
      avvolge il segmento tendineo e ritorna all'osso
    \end{itemize}
  \item
    \mus{miloioideo} --- coppia di muscoli piatti e triangolari, in
    posizione controlaterale, uniti da un \tol{rafe mediano}.
    Costituiscono il vero pavimento della cavità orale

    \begin{itemize}
    \tightlist
    \item
      \textbf{Origine}: tutta la \a{linea miloioidea} nella faccia
      interna del corpo della \oss{mandibola}
    \item
      \textbf{Inserzione}: corpo dello \oss{ioide}, verso il quale
      convergono a triangolo tutte le fibre
    \item
      \textbf{Azione}: solleva il pavimento della cavità, spingendo la
      lingua sul palato\footnote{Movimento funzionale durante la prima
        fase della deglutizione: la seconda è involontaria e mediata da
        riflessi vagali}
    \end{itemize}
  \item
    \mus{genioioideo} --- muscolo che rinforza la parete inferiore
    posizionandosi di fatto sul punto debole (il rafe) del miloioideo

    \begin{itemize}
    \tightlist
    \item
      \textbf{Origine}: \a{apofisi geni}\footnote{Piccola apofisi
        situata medialmente nel lato interno del corpo della mandibola
        (\netter{17})}
    \item
      \textbf{Inserzione}: faccia anteriore del corpo dello \oss{ioide}
    \item
      \textbf{Azione}: a seconda del punto fisso: innalza e avanza lo
      ioide oppure abbassa e arretra la mandibola
      \marginnote{In questo si oppone ai muscoli masticatori, che invece tendono a serrarla}
    \end{itemize}
  \item
    \mus{stiloioideo} --- non parte del pavimento, ma collabora con il
    \emph{digastrico}

    \begin{itemize}
    \tightlist
    \item
      \textbf{Origine}: \a{processo stiloideo} del \oss{temporale}
    \item
      \textbf{Inserzione}: porzione laterale del corpo e
      \a{piccolo corno} dello \oss{ioide}
    \item
      \textbf{Azione}: collabora con il digastrico (\emph{ventre
      posteriore}) + se fa punto fisso sul processo stiloideo alza lo
      ioide (⇒ accorciando la faringe)
    \end{itemize}
  \end{enumerate}
\end{itemize}

\hypertarget{vascolarizzazione-e-innervazione-2}{%
\paragraph{Vascolarizzazione e
innervazione}\label{vascolarizzazione-e-innervazione-2}}

\footnotesize

La vascolarizzazione completa è nel paragrafo sulla
\protect\hyperlink{lingua}{lingua} (pagina \pageref{lbllingua})
\normalsize

\begin{itemize}
\tightlist
\item
  \textbf{Vascolarizzazione}: 3 sorgenti arteriose
\end{itemize}

\begin{enumerate}
\def\labelenumi{\arabic{enumi}.}
\tightlist
\item
  Arteria \art{sottomentale}~

  \begin{itemize}
  \tightlist
  \item
    Vascolarizza il ventre anteriore del digastrico
  \end{itemize}
\item
  Arteria \art{miloioidea} (\textless{} \art{alveolare inferiore}
  \textless{} \art{mascellare interna})

  \begin{itemize}
  \item
    \mus{miloioideo}
  \end{itemize}
\item
  Arteria \art{sottolinguale} (\textless{} \art{linguale} \textless{}
  \art{carotide esterna})
  \marginnote{La carotide esterna cede la \emph{linguale} tra la \emph{faciale} e la \emph{tiroidea superiore}}

  \begin{itemize}
  \item
    \a{ghiandola sottolinguale}
  \item
    Tutto il pavimento della cavità orale che si trova sopra al
    \mus{miloioideo}
  \end{itemize}
\end{enumerate}

\begin{itemize}
\tightlist
\item
  \textbf{Innervazione}
\item
  Sensitiva (per la mucosa): \ner{linguale} (\textless{}
  \ner{mandibolare} \textless{} \ner{trigemino})
\item
  Muscolare: complessa, perché dipende dal muscolo
\end{itemize}

\begin{longtable}[]{@{}ll@{}}
\toprule
Muscolo & Innervazione\tabularnewline
\midrule
\endhead
\mus{miloioideo} & \ner{miloioideo} \textless{}
\ner{mandibolare}\tabularnewline
\mus{genoioideo} & \textless{} \ner{C1}\tabularnewline
\mus{digastrico} (ventre anteriore) & \ner{miloioideo}\tabularnewline
\mus{digastrico} (ventre posteriore) & \ner{faciale}\tabularnewline
\bottomrule
\end{longtable}

\hypertarget{lingua}{%
\subsection{\texorpdfstring{Lingua
\label{lbllingua}}{Lingua }}\label{lingua}}

\begin{itemize}
\tightlist
\item
  Organo muscolare e fibroso mobile adagiato sul pavimento della cavità
  orale

  \begin{itemize}
  \tightlist
  \item
    Articolazione del linguaggio
  \item
    Masticazione e deglutizione
  \item
    Gusto
  \end{itemize}
\item
  Si individuano 3 parti

  \begin{itemize}
  \tightlist
  \item
    Radice (termina alla \a{v linguale})
  \item
    Corpo
  \item
    Apice
  \end{itemize}
\end{itemize}

\hypertarget{landmark-anatomici}{%
\subsubsection{Landmark anatomici}\label{landmark-anatomici}}

\begin{itemize}
\tightlist
\item
  Radice

  \begin{itemize}
  \tightlist
  \item
    Tre \a{pieghe glosso-epiglottiche} (2 laterali e 1 mediana), che
    derivano dalla continuità di mucosa che ricopre la regione glottica
    e si estende fino alla radice della lingua

    \begin{itemize}
    \tightlist
    \item
      Le 3 pieghe individuano 2 cavità (\a{vallecole})
    \end{itemize}
  \item
    \a{v linguale} (o \a{solco terminale}): si trova ad 1/4 della
    lunghezza dalla radice, e alloggia il \a{foro cieco}. Davanti si
    trovano le \a{papille vallate}
  \end{itemize}
\item
  Corpo

  \begin{itemize}
  \tightlist
  \item
    \a{papille filiformi}~
  \item
    \a{solco mediano}~
  \item
    \a{frenulo} (nella faccia rivolta verso il pavimento), che
    testimonia la continuità di mucosa in tutte le zone della cavità
    orale
  \end{itemize}
\item
  Apice
\item
  Regione sublinguale (vedi)

  \begin{itemize}
  \tightlist
  \item
    Frenulo della lingua (piega mucosa)
  \item
    2 papille laterali al frenulo (\a{caruncole sottolinguali}): sono
    gli sbocchi delle \a{ghiandole sottomandibolari (di Wharton)}
  \item
    Visibili due rilievi laterali (che nascondono sotto la
    \a{ghiandola sottolinguale})
  \end{itemize}
\end{itemize}

\hypertarget{muscolatura}{%
\subsubsection{Muscolatura}\label{muscolatura}}

\begin{itemize}
\item
  3 componenti

  \begin{itemize}
  \tightlist
  \item
    Strutture fibrose di supporto
  \item
    Muscolatura estrinseca (muove la lingua)
  \item
    Muscolatura intrinseca (provoca cambiamenti nella \emph{forma} della
    lingua)
  \end{itemize}
\item
  \netter{59}
\end{itemize}

\hypertarget{strutture-fibrose-di-supporto}{%
\paragraph{Strutture fibrose di
supporto}\label{strutture-fibrose-di-supporto}}

\begin{itemize}
\tightlist
\item
  \tol{membrana ioglossa} --- si origina dal margine superiore
  dell'intero osso \oss{ioide} (corpo e piccolo corno) e corre
  verticalmente verso l'alto, per disperdersi nella radice della lingua
\item
  \tol{setto mediano} --- corre perpendicolarmente rispetto alla
  \emph{membrana ioglossa} per fornire un supporto orizzontale ai
  muscoli che compongono il corpo
\end{itemize}

\hypertarget{muscolatura-estrinseca}{%
\paragraph{Muscolatura estrinseca}\label{muscolatura-estrinseca}}

\begin{itemize}
\tightlist
\item
  \mus{genioglosso} --- il muscolo più grande e più esteso che partecipa
  alla costruzione della lingua

  \begin{itemize}
  \tightlist
  \item
    \textbf{Origine}: \a{apofisi geni}, per poi aprirsi verticalmente a
    ventaglio in 3 gruppi di fibre (superiori, intermedie e inferiori)
  \item
    \textbf{Inserzione}:

    \begin{itemize}
    \tightlist
    \item
      Fibre inferiori: osso \oss{ioide}
    \item
      Intermedie: radice e dorso della lingua
    \item
      Superiori: apice della lingua
    \end{itemize}
  \item
    \textbf{Azione}
  \end{itemize}
\item
  \mus{stiloglosso}~

  \begin{itemize}
  \tightlist
  \item
    \textbf{Origine}: dal \a{processo stiloideo}. Decorre in basso,
    medialmente e verso l'avanti
  \item
    \textbf{Inserzione}: le fibre inferiori convergono e si intrecciano
    con il muscolo \mus{joglosso}, nella parte inferoposteriore del
    corpo della lungua. Le fibre superiori si portano avanti, senza
    intrecciarsi, per formare il \emph{margine laterale} della lingua
  \item
    \textbf{Azione}: retrazione della lingua
  \item
    \textbf{Innervazione}
  \end{itemize}
\item
  \mus{jo-glosso} --- muscolo sottile, piatto e quadrangolare che
  insieme al \mus{genioglosso} costituisce la \a{radice} della lingua

  \begin{itemize}
  \tightlist
  \item
    \textbf{Origine}: margine e parte laterale dello \oss{ioide}. Si
    porta verso l'alto e medialmente, intrecciandosi con le fibre dello
    \mus{stiloglosso}. Attraversa lo spessore della lingua e esce
    superiormente per inserirsi
  \item
    \textbf{Inserzione}: \a{setto trasverso mediano}
  \item
    \textbf{Azione}: retrazione e abbassamento della lingua
  \end{itemize}
\item
  \mus{palatoglosso}~

  \begin{itemize}
  \tightlist
  \item
    \textbf{Origine}: \tol{aponeurosi palatina}. Si porta verso il basso
    e verso l'avanti
  \item
    \textbf{Inserzione}: margine laterale della lingua
  \item
    \textbf{Azione}: tira verso l'alto la lingua, per chiudere
    posteriormente la comunicazione tra cavo orale e orofaringe
  \item
    \textbf{Innervazione}: \ner{plesso faringeo}
  \end{itemize}
\item
  \mus{faringoglosso} --- parte inferiore e terminale del
  \mus{costrittore superiore della faringe}\footnote{Che corre
    verticalmente per la lunghezza della faringe}

  \begin{itemize}
  \tightlist
  \item
    \textbf{Origine}: \mus{costrittore superiore della faringe}
  \item
    \textbf{Inserzione}: parte posteriore dei lati della lingua
  \item
    \textbf{Azione}: leggera ritrazione ed elevazione della lingua
  \end{itemize}
\item
  \mus{amigdaloglosso}~

  \begin{itemize}
  \tightlist
  \item
    \textbf{Origine}: sempre \mus{costrittore superiore della faringe}.
    Costituisce la parete muscolare laterale della
    \a{loggia tonsillare}\footnote{Parete molto sottile, nel momento in
      cui si rimuovono le tonsille bisogna stare attenti a non lacerarla}
  \item
    \textbf{Inserzione}: parte posteriore dei lati della lingua
  \item
    \textbf{Azione}: leggera ritrazione ed elevazione della lingua
  \end{itemize}
\end{itemize}

\hypertarget{muscolatura-intrinseca}{%
\paragraph{Muscolatura intrinseca}\label{muscolatura-intrinseca}}

\begin{itemize}
\item
  \mus{longitudinale superiore della lingua} e
  \mus{longitudinale inferiore della lingua}~

  \begin{itemize}
  \tightlist
  \item
    \textbf{Origine}: \a{piega glossoepiglottica mediana} della mucosa,
    tesa tra lingua ed epiglottide e corre longitudinalmente alla lingua
  \item
    \textbf{Inserzione}: non c'è, è libera, si porta fino all'apice
    della lingua
  \item
    \textbf{Azione}: innalzamento della punta della lingua
  \end{itemize}
\item
  \mus{trasverso della lingua}~

  \begin{itemize}
  \tightlist
  \item
    Corre trasversalmente, rendendo concava la lingua con la contrazione
  \end{itemize}
\item
  \mus{verticale}

  \begin{itemize}
  \tightlist
  \item
    Va dalla faccia dorsale a quella inferiore, in senso verticale (non
    ha origine e/o inserzione

    \begin{itemize}
    \tightlist
    \item
      \textbf{Azione}: riduce lo spessore verticale della lingua
    \end{itemize}
  \end{itemize}
\end{itemize}

\hypertarget{vascolarizzazione}{%
\subsubsection{Vascolarizzazione}\label{vascolarizzazione}}

\begin{itemize}
\tightlist
\item
  \textbf{Vascolarizzazione} molto abbondante\footnote{E anche di
    difficile gestione in caso di emorragia. Solitamente si va a legare
    l'arteria \art{linguale}. Il punto più a monte (prima che si stacchi
    qualsiasi ramo) è il passaggio nel \a{triangolo di beclard}. In caso
    di emorragia al solo corpo, si lega l'arteria nel
    \a{triangolo di Pigroff}, situato dopo il distacco della
    \art{dorsale}}

  \begin{itemize}
  \tightlist
  \item
    Afflusso: 1 sistema

    \begin{itemize}
    \tightlist
    \item
      Arteria \art{linguale}\footnote{\nnetter{70 B}} (ceduta dalla
      \art{carotide} in corrispondenza del \a{grande corno} dello
      \oss{ioide})
      \marginnote{Per accedere chirurgicamente all'arteria linguale si sfruttano i triangoli di \a{beclard} e di \a{pigroff} (vedi box)}

      \begin{itemize}
      \tightlist
      \item
        Attraversa dalla radice verso la punta
      \item
        Subito all'inizio, dalla radice, cede la/le arterie
        \art{dorsali della lingua}, che vascolarizzano la parte dorsale
      \item
        Decorre profondamente tra lo \mus{joglosso} (l) e il
        \mus{genioglosso} (m) (\nnetter{68 B})
      \item
        Cede l'arteria \art{sottolinguale}, che continua adagiata sul
        \mus{miloioideo}. Vascolarizza il pavimento e la
        \a{ghiandola sottolinguale}
      \item
        Prosegue come arteria \art{linguale profonda}, che i esaurisce
        nella punta
      \end{itemize}
    \item
      Come visto, alla (o alle) arteria \art{dorsale} compete la radice
    \end{itemize}
  \item
    Ritorno: 2 sistemi che convergono a formare la vena \ven{linguale},
    parallela all'arteria

    \begin{itemize}
    \tightlist
    \item
      Si apre nella \ven{giugulare interna}
    \item
      Vena \ven{linguale}: da convergenza

      \begin{itemize}
      \tightlist
      \item
        Del circolo laterale della regione anteriore (dalla
        \ven{linguale profonda})
      \item
        Il secondo ramo è dato dalla \ven{sottolinguale}, che confluisce
        nella linguale
      \end{itemize}

      Decorre poi come satellite del \ner{ipoglosso}, adagiata sul
      muscolo \mus{joglosso} (sempre \nnetter{68 B})
    \end{itemize}
  \end{itemize}
\end{itemize}

\hypertarget{innervazione-1}{%
\subsubsection{Innervazione}\label{innervazione-1}}

\begin{itemize}
\tightlist
\item
  \textbf{Innervazione}: nella zona di competenza del mandibolare

  \begin{itemize}
  \tightlist
  \item
    Sensitivo

    \begin{itemize}
    \tightlist
    \item
      \textbf{Corpo}:

      \begin{itemize}
      \tightlist
      \item
        Sensitiva (non gustativa): \ner{linguale}, che afferirà al
        \ner{mandibolare}
      \item
        Senso gustativo: \ner{faciale}
      \end{itemize}
    \item
      \textbf{Radice}: tattile e gustativa \ner{glossofaringeo} (che
      \textbf{non} afferisce al mandibolare, ma è nervo cranico a sé
      stante), che si fa carico sia della innervazione motoria
    \end{itemize}
  \item
    Innervazione \textbf{muscolare} mediata dall'\ner{ipoglosso}, che
    innerva \emph{tutti} i muscoli tranne il \mus{palatoglosso}
  \end{itemize}
\end{itemize}

\normalbox{I triangoli della zona sopraioidea, label=lbltriangolilinguali}{

\includegraphics[width=\textwidth]{img/beclard-pigroff.png}

Ci sono 2 triangoli principali che garantiscono l'accesso all'arteria
linguale

\begin{itemize}
\tightlist
\item \a{triangolo di beclard}~
  \begin{itemize}
  \tightlist
  \item P: ventre posteriore del \mus{digastrico}
  \item A: margine anteriore dello \mus{sternocleidomastoideo}
  \item I: \a{grande corno} dell'osso \oss{ioide}
  \end{itemize}
\item \a{triangolo di pigroff}~
  \begin{itemize}
  \tightlist
  \item S: \ner{ipoglosso}
  \item A: margine posteriore del \mus{miloioideo}
  \item I: ventre anteirore del \mus{digastrico}
  \end{itemize}
\end{itemize}
}

\hypertarget{dentatura}{%
\subsection{Dentatura}\label{dentatura}}

\begin{itemize}
\tightlist
\item
  32 denti permanenti (16 per arcata), ogni arcata divisa in 2 metà ⇒ 4
  quadranti

  \begin{itemize}
  \tightlist
  \item
    2 incisivi (1 cuspide)
  \item
    1 canino (2 cuspidi)
  \item
    2 premolari (2 cuspidi)
  \item
    3 molari (4 cuspidi) -- l'8/o fa il cazzo che gli pare
  \end{itemize}
\item
  Superfici

  \begin{itemize}
  \tightlist
  \item
    Esterna: labiale vs buccale
  \item
    Interna: palatale vs linguale
  \end{itemize}
\item
  Il dente è costituito a strati

  \begin{itemize}
  \tightlist
  \item
    Strato esterno di \textbf{idrossiapatite}
  \item
    Uno strato connettivo \textbf{mineralizzato}, detto
    \textbf{dentina}. Al suo interno presenta i \textbf{tubuli
    dentinati}, tubuli con andamento radiale
  \item
    Lo strato profondo della dentina è caratterizzato da
    \textbf{odontoblasti}, cellule che emettono prolungamenti nei tubuli
    dentinati
  \item
    Intimamente troviamo la \textbf{cavità pulpare}, grande nella corona
    e piccola nelle radici, che presenta vasi e nervi
  \item
    All'apice della radice troviamo i \a{fori apicali}, in maniera da
    permettere a vasi e nervi di raggiungere la cavità pulpare dalle
    \a{cavità alveolari}
  \end{itemize}
\item
  L'articolazione tra denti e cavità alveolari è una \textbf{gonfosi}

  \begin{itemize}
  \tightlist
  \item
    Ogni radice è inserita nella cavità
  \item
    Ogni radice è circondata e \textbf{ancorata} da numerosi
    \tol{legamenti interdentali} e \tol{legamenti alveolodentali}
  \end{itemize}
\item
  La parte della mucosa gengivale è innervata

  \begin{itemize}
  \tightlist
  \item
    Sensitivamente, per avere informazioni tattili
  \item
    Propriocettivamente, per avere informazioni su quanto e come
    masticare
  \end{itemize}
\item
  \textbf{Vascolarizzazione}

  \begin{itemize}
  \tightlist
  \item
    Per l'arcata inferiore: da arteria \art{mascellare interna}, che
    emette

    \begin{itemize}
    \tightlist
    \item
      \art{arteria alveolare inferiore}: passa nel ramo e nel corpo
      della mandibola, emettendo rami per gli alveoli e un
      \art{arteria mentale}, che si porta esternamente
    \end{itemize}
  \item
    Per l'arcata superiore:

    \begin{itemize}
    \tightlist
    \item
      \art{arteria alveolare media}~
    \item
      \art{arteria alveolare superoposteriore}~
    \item
      \art{arteria infraorbitaria}~
    \end{itemize}
  \end{itemize}
\item
  \textbf{Innervazione} -- a carico dei \ner{nervi alveolari}

  \begin{itemize}
  \tightlist
  \item
    Arcata inferiore: \ner{trigemino} \textgreater{} \ner{mandibolare}
    \textgreater{} \ner{nervi alveolari}
  \item
    Arcata superiore: \ner{trigemino} \textgreater{} \ner{mascellare} e
    qui si divide, ricalcando perfettamente il deocrso arterioso

    \begin{itemize}
    \item
      \ner{nervi alveolari superiori anteriori}
    \item
      \ner{nervi alveolari superiori medi}
    \item
      \ner{nervo infraorbitario}
    \end{itemize}
  \end{itemize}
\end{itemize}

\hypertarget{ghiandole-salivari-maggiori}{%
\subsection{Ghiandole salivari
maggiori}\label{ghiandole-salivari-maggiori}}

\begin{itemize}
\tightlist
\item
  Sono le 3 ghiandole salivari \emph{più grandi}

  \begin{enumerate}
  \def\labelenumi{\arabic{enumi}.}
  \tightlist
  \item
    \a{parotide}¬
  \item
    \a{ghiandola sottomandibolare}~
  \item
    \a{ghiandola sottolinguale}
    \asidefigure{img/ghiandole-salivari-maggiori.jpg}{}
  \end{enumerate}

  Oltre a queste esistono centinaia di altre piccole ghiandole
  (collettivamente note come \a{ghiandole salivari minori}) localizzate
  in svariati punti della sottomucosa orale (dalle labbra alla faringe)
\end{itemize}

\hypertarget{loggia-sottolinguale}{%
\subsubsection{Loggia sottolinguale}\label{loggia-sottolinguale}}

\begin{itemize}
\item
  \netter{46--47}
\item
  Posizionata inferolateralmente rispetto alla lingua
\item
  \textbf{Limiti} (\netter{51})

  \begin{itemize}
  \tightlist
  \item
    AL: \a{fossa sottolinguale} del corpo della \oss{mandibola}
  \item
    PM: muscoli della lingua (\mus{jo-glosso} e \mus{genioglosso})
  \item
    I: Muscolo \mus{miloioideo}
  \item
    S: Mucosa del pavimento della cavità orale
  \end{itemize}
\item
  Subito sotto al sottilissimo strato di mucosa si trova un esteso
  \textbf{plesso venoso} (\a{plesso venoso sublinguale}). Questo rende
  la via sublinguale una via di somministrazione \emph{estremamente}
  rapida
\end{itemize}

\hypertarget{ghiandola-sottolinguale}{%
\paragraph{Ghiandola sottolinguale}\label{ghiandola-sottolinguale}}

\begin{itemize}
\tightlist
\item
  Forma e dimensione di una mandorla (è la più piccola tra le ghiandole
  salivari maggiori)
\item
  I dotti escretori sboccano nella mucosa buccale, in maniera separata
\item
  Servita da arteria \art{sottolinguale} e vena omonima (che,
  ricordiamoci, serve anche il pavimento)
\item
  Innervata dal parasimpatico dalle fibre postgangliari provenienti dal
  \ner{ganglio sottomandibolare}
\end{itemize}

\hypertarget{loggia-sottomandibolare}{%
\subsubsection{Loggia sottomandibolare}\label{loggia-sottomandibolare}}

\begin{itemize}
\tightlist
\item
  Situata nella \textbf{regione sovraioidea}\footnote{\includegraphics[width=\textwidth,height=5cm]{img/loggia-sottomandibolare.png}},
  la quale ha per limiti

  \begin{itemize}
  \tightlist
  \item
    L: Margine anteriore dello \mus{sternocleidomastoideo}
  \item
    I: piano orizzontale dello \oss{ioide}
  \item
    S: margine inferiore della \oss{mandibola}
  \end{itemize}
\item
  In termini di epitelio che riveste:

  \begin{enumerate}
  \def\labelenumi{\arabic{enumi}.}
  \tightlist
  \item
    Cute
  \item
    Sottocute + SMAS + \mus{platisma}
  \item
    \tol{fascia cervicale superficiale}~
  \end{enumerate}
\item
  La \textbf{loggia sottomandibolare} in sé per sé ha limiti più
  ristretti, trovandosi \emph{entro} la regione sovraioidea

  \begin{itemize}
  \tightlist
  \item
    SM: \mus{miloioideo}
  \item
    SL: corpo della \textbackslash{}oss\{mandibola
  \item
    I: \tol{fascia cervicale superficiale}, che segue la cute nel punto
    in cui questa piega per sovrastare la regione sovraioidea
  \item
    P: \mus{ioglosso}~
  \end{itemize}
\end{itemize}

\hypertarget{ghiandola-sottomandibolare}{%
\paragraph{Ghiandola
sottomandibolare}\label{ghiandola-sottomandibolare}}

\begin{itemize}
\tightlist
\item
  Forma a noce
\item
  Individuiamo 2 porzioni\footnote{\includegraphics[width=\textwidth,height=5cm]{img/rapporto-sottomandibolare-sottolinguale.png}}

  \begin{enumerate}
  \def\labelenumi{\arabic{enumi}.}
  \tightlist
  \item
    Porzione superficiale --- parte più estesa della ghiandola, si
    adatta al corpo della mandibola e alla fascia cervicale
  \item
    Porzione profonda --- prolungamento della porzione superficiale che
    si dirige medialmente, accolto \emph{tra} i due ventri del
    \mus{digastrico}, fino ad avvolgere il margine libero del muscolo
    \mus{miloioideo} per occupare la parte posteriore della loggia
    sottolinguale\footnote{Una conseguenza di questo è che \textbf{le
      logge sono in comunicazione} l'una con l'altra, e quindi un
      fenomeno infettivo che interessa una delle due logge non ha
      ostacoli a propagarsi anche nell'altra} (\nnetter{69B})
  \end{enumerate}
\item
  La ghiandola ha \textbf{un solo dotto escretore}
  (\a{dotto di warton}), che origina dalla \textbf{porzione
  superficiale}. Si apre nella \a{caruncola sottilinguale} (e quindi
  nella \emph{loggia sottolinguale})

  \begin{itemize}
  \tightlist
  \item
    Il dotto origina dalla porzione superficiale, ma per portarsi sopra
    il \emph{miloioideo} deve curvare verso l'alto in maniera molto
    ripida, subito sotto la caruncola. Questo è il punto principale in
    cui eventuali calcoli rimangono bloccati, provocando un
    ingrossamento della ghiandola
  \end{itemize}
\item
  Rapporti

  \begin{itemize}
  \tightlist
  \item
    Attraverso la ghiandola passa il nervo \ner{linguale} (\textless{}
    \ner{mandibolare}), il quale segue il dotto e innerva in questo modo
    il pavimento della lingua
  \item
    Arteria \art{faciale}; che provvede anche alla vascolarizzazione
    della ghiandola stessa tramite alcuni rami che cede
  \end{itemize}
\end{itemize}

\hypertarget{loggia-parotidea}{%
\subsubsection{Loggia parotidea}\label{loggia-parotidea}}

\begin{itemize}
\tightlist
\item
  Loggia che accoglie la \a{parotide}, e si sviluppa in senso
  lateromediale, arrivando fino a raggiungere i muscoli che
  costituiscono la parete della \a{faringe}

  \begin{itemize}
  \tightlist
  \item
    S: arcata zigomatiaca
  \item
    I: angolo della \oss{mandibola}
  \item
    A: ramo della \oss{mandibola} e \mus{massetere}
  \item
    P: margine anteriore dello \mus{sternocleidomastoideo}
  \item
    L: SMAS
  \item
    M: ventre posteriore del \mus{digastrico}, \mus{stiloideo} e muscoli
    della parete esterna della faringe
  \end{itemize}
\item
  Delimitata da uno sdoppiamento della fascia superficiale, che si
  scolla in un \emph{foglietto esterno}, che rimane fedelmente attaccato
  allo SMAS e un \emph{foglietto interno}, che delimita internamente lo
  scavo profondo

  \begin{itemize}
  \tightlist
  \item
    La loggia è \textbf{chiusa} ⇒ non c'è connessione tra loggia
    parotidea e sottomandibolare
  \item
    Processi infettivi rimangono confinati\footnote{Vedi \pat{parotite}:
      abbiamo ingrossamento \textbf{localizzato che non diffonde} ⇒
      dolore per compressione nervosa dei rapporti intriseci con i
      nervi, visibili in \nnetter{54}}
  \end{itemize}
\end{itemize}

\hypertarget{ghiandola-parotidea}{%
\paragraph{Ghiandola parotidea}\label{ghiandola-parotidea}}

\begin{itemize}
\tightlist
\item
  Ghiandola a secreto sieroso che partecipa alla produzione del secreto
  salivare
\item
  Contrae numerosi rapporti

  \begin{itemize}
  \tightlist
  \item
    Estrinseci: vedi limiti della loggia
  \item
    Intrinseci

    \begin{itemize}
    \tightlist
    \item
      \art{carotide esterna} --- attraversa il parenchima della
      ghiandola con il suo ultimo tratto, e si divide nei rami terminali
      al suo interno
    \item
      \ven{temporale superficiale} --- attraversa il parenchima della
      ghiandola dopo aver ricevuto la \ven{trasversa della faccia}.
      All'interno della parotide si forma un grande tronco venoso che
      decorre nel parenchima ghiandolare (\a{tronco retromandibolare})

      \begin{itemize}
      \tightlist
      \item
        Una parte di questo (\a{plesso pterigoideo}, che riceve le vene
        mascellari) afferirà alla \ven{giugulare esterna} una volta
        uscito dal parenchima ghiandolare\footnote{Assieme alla
          \art{vena auricolare posteriore}}
      \item
        Una parte del \a{tronco retromandibolare}, una volta ricevuto la
        \ven{faciale} e \ven{linguale}, si getterà nella
        \ven{giugulare interna}
      \end{itemize}
    \item
      Nervo \ner{faciale} --- decorre nello spessore della parotide e al
      suo interno si divide nei 5 rami terminali
      \marginnote{Provvede all'innervazione motoria di tutti i muscoli mimici, infatti una paralisi a questi può essere anche secondaria ad un danno alla parotide}
    \item
      Nervo \ner{auricolotemporale} (\textless{} \ner{mandibolare})
    \end{itemize}
  \end{itemize}
\item
  Vascolarizzazione: \art{trasversa della faccia} e
  \ven{tronco retromandibolare}

  \begin{itemize}
  \tightlist
  \item
    \textless{} \art{temporale superficiale}. Talvolta si stacca proprio
    all'interno del parenchima ghiandolare
  \item
    A volte: rami da altre arterie circostanti

    \begin{itemize}
    \tightlist
    \item
      \art{auricolare posteriore}~
    \item
      \art{carotide esterna}~
    \end{itemize}
  \end{itemize}
\end{itemize}

\hypertarget{drenaggio-linfatico}{%
\subsection{Drenaggio linfatico}\label{drenaggio-linfatico}}

\begin{itemize}
\tightlist
\item
  La maggior densità di linfonodi si ritrova nella regione di
  testa/collo
\item
  Il drenaggio linfatico della regione è ad opera di linfonodi
  superficiali e profondi. Questi ultimi drenano anche i linfonodi
  superficiali

  \begin{itemize}
  \tightlist
  \item
    Linfonodi superficiali

    \begin{itemize}
    \tightlist
    \item
      Catena orizzontale (mento → occipite)
    \item
      Catena verticale (segue la giugulare esterna)
    \end{itemize}
  \item
    Linfonodi profondi

    \begin{itemize}
    \tightlist
    \item
      Catena verticale (segue la giugulare interna)
    \end{itemize}
  \end{itemize}
\item
  Drenaggio: tessuto \textgreater{} \lin{linfonodi superficiali}
  \textgreater{} \lin{linfonodi profondi} \textgreater{}
  \lin{dotto giugulare} (dx e sx) \textgreater{} \lin{dotto succlavio}
  \textgreater{} convergenza nel \a{triangolo giugulo-succlavio}
\end{itemize}

\begin{figure}
\centering
\includegraphics{img/drenaggio-linfatico-cavita-orale.png}
\caption{Drenaggio linfatico della regione testa/collo}
\end{figure}

\hypertarget{linfonodi-superficiali}{%
\subsubsection{Linfonodi superficiali}\label{linfonodi-superficiali}}

\begin{itemize}
\tightlist
\item
  Catena orizzontale

  \begin{itemize}
  \tightlist
  \item
    \lin{linfonodi mentali} --- sono 2, si trovano nel triangolo mentale
    (tra i ventri anteriori dei due muscoli digastrici)
  \item
    \lin{linfonodi sottomandibolari} --- sono 4/5, si dispongono sulla
    superficie della \a{ghiandola sottomandibolare}
  \item
    \lin{linfonodi buccali} --- seguono la v. \ven{faciale}
  \item
    \lin{linfonodi parotidei} --- disposti sulla superficie della
    \a{parotide} \lin{linfonodi preauricolari} --- disposti
    anteriormente al padiglione auricolare, molto vicini al gruppo dei
    lfnd parotidei
  \item
    \lin{linfonodi mastoidei} --- situati sul processo mastoideo, dietro
    al padiglione auricolare
  \item
    \lin{linfonodi occipitali}~
  \end{itemize}
\item
  Catena verticale

  \begin{itemize}
  \tightlist
  \item
    \lin{linfonodi cervicali superficiali} --- seguono la giugulare
    esterna, e sono quelli del collo che si sentono spesso durante
    l'esame obiettivo
  \end{itemize}
\end{itemize}

\hypertarget{territori-di-drenaggio}{%
\paragraph{Territori di drenaggio}\label{territori-di-drenaggio}}

\begin{longtable}[]{@{}ll@{}}
\toprule
\begin{minipage}[b]{0.47\columnwidth}\raggedright
Distretto\strut
\end{minipage} & \begin{minipage}[b]{0.47\columnwidth}\raggedright
Linfonodi di competenza\strut
\end{minipage}\tabularnewline
\midrule
\endhead
\begin{minipage}[t]{0.47\columnwidth}\raggedright
Labbro inferiore (parte mediana)\strut
\end{minipage} & \begin{minipage}[t]{0.47\columnwidth}\raggedright
Linfonodi sottomentali\strut
\end{minipage}\tabularnewline
\begin{minipage}[t]{0.47\columnwidth}\raggedright
Labbro inferiore (parte laterale), labbro superiore, porzione laterale
della priamide nasale\strut
\end{minipage} & \begin{minipage}[t]{0.47\columnwidth}\raggedright
Linfonodi sottomandibolari\strut
\end{minipage}\tabularnewline
\begin{minipage}[t]{0.47\columnwidth}\raggedright
Guancia\strut
\end{minipage} & \begin{minipage}[t]{0.47\columnwidth}\raggedright
Linfonodi buccali \textgreater{} sottomandibolari\strut
\end{minipage}\tabularnewline
\begin{minipage}[t]{0.47\columnwidth}\raggedright
Palpebra superiore e inferiore, cute frontale e parietale
(parzialmente)\strut
\end{minipage} & \begin{minipage}[t]{0.47\columnwidth}\raggedright
Linfonodi parotidei\strut
\end{minipage}\tabularnewline
\begin{minipage}[t]{0.47\columnwidth}\raggedright
Regione parietale e temporale\strut
\end{minipage} & \begin{minipage}[t]{0.47\columnwidth}\raggedright
Linfonodi mastoidei\strut
\end{minipage}\tabularnewline
\begin{minipage}[t]{0.47\columnwidth}\raggedright
Occipite\strut
\end{minipage} & \begin{minipage}[t]{0.47\columnwidth}\raggedright
Linfonodi occipitali\strut
\end{minipage}\tabularnewline
\bottomrule
\end{longtable}

\hypertarget{linfonodi-profondi}{%
\subsubsection{Linfonodi profondi}\label{linfonodi-profondi}}

\begin{itemize}
\tightlist
\item
  Sono disposti a catena, e seguono la \ven{giugulare interna}
\item
  Drenano gli organi contenuti nella cavità orale

  \begin{itemize}
  \tightlist
  \item
    \a{lingua}: drenaggio complicato, dipende dalla porzione della
    lingua che si sta considerando
    \asidefigure{img/drenaggio-lingua.png}{}
  \item
    \a{tonsilla palatina}: \lin{linfonodo digastrico} (angolo del
    \mus{digastrico})
  \item
    \a{faringe}: viene drenata prima da linfonodi locali (anche se
    alcuni linfonodi faringei degenerano con l'età), che confluiscono
    nella catena cervicale profonda
  \end{itemize}
\end{itemize}

\hypertarget{collo}{%
\section{Collo}\label{collo}}

\begin{itemize}
\tightlist
\item
  Regione piramidale compresa tra

  \begin{itemize}
  \tightlist
  \item
    Piano individuato da base cranica e base della mandibola
  \item
    Apertura del torace (sterno e clavicola)
  \end{itemize}
\item
  Il collo viene diviso in 2 triangoli (\textbf{triangolo anteriore} e
  \textbf{triangolo posteriore}) dalla presenza dello
  \mus{sternocleidomastoideo} \asidefigure{img/triangoli-collo.png}{}

  \begin{itemize}
  \tightlist
  \item
    Triangolo anteriore
  \item
    Triangolo posteriore
  \end{itemize}
\end{itemize}

\hypertarget{fasce-del-collo}{%
\subsection{Fasce del collo}\label{fasce-del-collo}}

\begin{itemize}
\tightlist
\item
  Le fasce del collo sono \textbf{piani di scorrimento distinti}:
  servono a svincolare strutture adiacenti, in modo che l'intero
  contenuto di una fascia possa scorrere indipendentemente dal contenuto
  di un'altra fascia
\end{itemize}

\begin{itemize}
\tightlist
\item
  3 (+1) fasce \marginnote{\netter{26}}

  \begin{enumerate}
  \def\labelenumi{\arabic{enumi}.}
  \tightlist
  \item
    \tol{fascia cervicale superficiale}
    (\textcolor{OrangeRed}{in rosso nella tavola})~
  \item
    \tol{fascia cervicale media}
    (\textcolor{Plum}{in viola nella tavola})~
  \item
    \tol{fascia cervicale profonda}, o \emph{prevertebrale}
    (\textcolor{Dandelion}{in giallo nella tavola})
  \item
    \tol{guaina viscerale} (\textcolor{LimeGreen}{in verde e in}
    \textcolor{SkyBlue}{azzurro nella tavola})
  \end{enumerate}
\end{itemize}

\greenbox{Fascio vascolonervoso del collo}{
Guaina che racchiude la \ven{giugulare interna}, la \art{carotide comune} e il nervo \ner{vago}
}

\hypertarget{fascia-cervicale-superficiale}{%
\subsubsection{Fascia cervicale
superficiale}\label{fascia-cervicale-superficiale}}

\begin{itemize}
\tightlist
\item
  Sistema fasciale a cilindro cavo che ricopre tutto il collo, sia
  anteriormente che posteriormente

  \begin{itemize}
  \tightlist
  \item
    Inserzione superiore: dalla protuberanza occipitale al processo
    mastoideo, per proseguire fino al margine inferiore della mandibola
  \item
    Inserzione inferiore: \a{incisura giugulare}, \oss{clavicola}, spina
    della \oss{scapola}, per terminare sul \emph{processo spinoso} di
    \oss{c7}
  \end{itemize}
\item
  La fascia cervicale superficiale \textbf{si sdoppia in 2 foglietti}
  per circondare vari muscoli superficiali (naturalmente in maniera
  simmetrica in entrambi i lati) e varie strutture:

  \begin{itemize}
  \tightlist
  \item
    Lo \mus{sternocleidomastoideo} nel triangolo anteriore
  \item
    Il \mus{trapezio} nella porzione posteriore\}
  \item
    La \a{parotide}
    \marginnote{Ricordiamo che la loggia parotidea è prodotta proprio dallo sdoppiamento dei foglietti di questa fascia}
  \end{itemize}
\end{itemize}

\hypertarget{fascia-cervicale-media}{%
\subsubsection{Fascia cervicale media}\label{fascia-cervicale-media}}

\begin{itemize}
\tightlist
\item
  Esclusivamente anteriore, a forma di trapezio

  \begin{itemize}
  \tightlist
  \item
    Limite superiore: \oss{ioide}
  \item
    Limite inferiore: giugulo e clavicola, per terminare sul margine
    posteriore della scapola
  \item
    Limite laterale: muscolo \mus{omoioideo}
  \end{itemize}
\item
  Si sdoppia a rivestire i soli \textbf{muscoli sottoioidei}
  (\mus{sternoioideo}, \mus{omoioideo} e \mus{sternotiroideo})
\end{itemize}

\hypertarget{fascia-cervicale-profonda-o-prevertebrale}{%
\subsubsection{Fascia cervicale profonda (o
prevertebrale)}\label{fascia-cervicale-profonda-o-prevertebrale}}

\begin{itemize}
\tightlist
\item
  Circolare, a forma di manicotto
\item
  Riveste tutti i \textbf{muscoli profondi} della regione del
  collo\footnote{\mus{prevertebrali}, \mus{scaleni},
    \mus{suboccipitali}\ldots{}}

  \begin{itemize}
  \tightlist
  \item
    Inserzione alta: base del cranio
  \item
    Inserzione bassa: inesistente, si disperde nel connettivo e nel
    mediastino
  \end{itemize}
\end{itemize}

\hypertarget{guaina-viscerale}{%
\subsubsection{Guaina viscerale}\label{guaina-viscerale}}

\begin{itemize}
\tightlist
\item
  Fascia che riveste i visceri del collo (\a{faringe}, \a{esofago} e
  \a{tiroide} in primis)
\item
  A strettissimo contatto con la \textbackslash{}tol\{fascia cervicale
  media, subito antistante
\item
  Tra la \tol{guaina viscerale} e la \tol{fascia cervicale profonda} si
  trova uno \a{spazio retroviscerale} (anche noto come
  \a{spazio retroesofageo} o \a{spazio retrofaringeo} a seconda di quale
  struttura si considera)

  \begin{itemize}
  \tightlist
  \item
    Connessione diretta con il mediastino posteriore
  \item
    Spazio ripieno di tessuto adiposo lasso per svincolare bene i
    visceri durante i movimenti del collo (molto più sensibili dei
    muscoli allo stiramento)
  \end{itemize}
\end{itemize}

\normalbox{Differenze di nomenclatura delle fasce tra denominazione europea e denominazione anglosassone}{
\begin{longtable}[]{@{}ll@{}} \toprule \begin{minipage}[b]{0.47\columnwidth}\raggedright Denominazione italiana\strut \end{minipage} & \begin{minipage}[b]{0.47\columnwidth}\raggedright Denominazione anglosassone\strut \end{minipage}\tabularnewline \midrule \endhead \begin{minipage}[t]{0.47\columnwidth}\raggedright Sottocute\strut \end{minipage} & \begin{minipage}[t]{0.47\columnwidth}\raggedright Fascia cervicale superficiale\strut \end{minipage}\tabularnewline \begin{minipage}[t]{0.47\columnwidth}\raggedright Fascia cervicale superficiale\strut \end{minipage} & \begin{minipage}[t]{0.47\columnwidth}\raggedright Foglietto superficiale della fascia cervicale profonda\strut \end{minipage}\tabularnewline \begin{minipage}[t]{0.47\columnwidth}\raggedright Guaina viscerale\strut \end{minipage} & \begin{minipage}[t]{0.47\columnwidth}\raggedright Strato medio della fascia cervicale profonda\strut \end{minipage}\tabularnewline \begin{minipage}[t]{0.47\columnwidth}\raggedright Fascia cervicale profonda\strut \end{minipage} & \begin{minipage}[t]{0.47\columnwidth}\raggedright Strato viscerale della fascia cervicale profonda\strut \end{minipage}\tabularnewline \bottomrule \end{longtable} 
}

\hypertarget{triangolo-anteriore}{%
\subsection{Triangolo anteriore}\label{triangolo-anteriore}}

\begin{itemize}
\tightlist
\item
  La presenza dello \oss{ioide} divide il triangolo anteriore in due
  regioni

  \begin{enumerate}
  \def\labelenumi{\arabic{enumi}.}
  \tightlist
  \item
    \textbf{Regione sovraioidea}

    \begin{itemize}
    \tightlist
    \item
      \a{ghiandola sottomandibolare}~
    \item
      Arteria \art{faciale} e vena \ven{faciale}~
    \item
      Nervo \ner{ipoglosso}~
    \item
      \lin{linfonodi sottomentali}~
    \item
      Muscolatura sovraioidea
    \end{itemize}
  \item
    \textbf{Regione sottoioidea}

    \begin{itemize}
    \tightlist
    \item
      Muscolatura sottoioidea
    \end{itemize}
  \end{enumerate}
\end{itemize}

\hypertarget{muscolatura-1}{%
\subsubsection{Muscolatura}\label{muscolatura-1}}

\footnotesize

Questi fanno parte della \textbf{regione sottoioidea}, perché i muscoli
della \emph{regione sovraioidea} sono già stati descritti parlando della
muscolatura del pavimento della cavità orale e quella subito sottostante
(da pagina \pageref{lblmuscolisovraioidei}) \normalsize

\marginnote{\netter{27, 28}}

\begin{itemize}
\tightlist
\item
  Muscolatura superficiale

  \begin{itemize}
  \tightlist
  \item
    M. \mus{sternoioideo}~

    \begin{itemize}
    \tightlist
    \item
      \textbf{Origine}: corpo dello \oss{ioide} (margine inferiore)
    \item
      \textbf{Inserzione}: manubrio dello \oss{sterno}
    \item
      \textbf{Azione}: abbassa lo ioide
    \end{itemize}
  \item
    M. \mus{omoioideo} --- muscolo nastriforme subito laterale allo
    sternoioideo, sviluppato in un \emph{ventre superiore} ed un
    \emph{ventre inferiore}

    \begin{itemize}
    \tightlist
    \item
      \textbf{Origine}: margine superiore della \oss{scapola}
    \item
      \textbf{Inserzione}: subito lateralmente allo \mus{sternoioideo}
    \item
      \textbf{Azione}: tende la \tol{fascia cervicale media}
    \end{itemize}
  \end{itemize}
\item
  Muscolatura profonda

  \begin{itemize}
  \tightlist
  \item
    M. \mus{sternotiroideo}~

    \begin{itemize}
    \tightlist
    \item
      \textbf{Origine}: faccia posteriore della prima
      \tol{cartilagine costale}
    \item
      \textbf{Inserzione}: \tol{cartilagine tiroidea}
    \item
      \textbf{Azione}: abbassa la laringe
    \end{itemize}
  \item
    M. \mus{tiroioideo}~

    \begin{itemize}
    \tightlist
    \item
      \textbf{Origine}: corpo e corno grande dello \oss{ioide}
    \item
      \textbf{Inserzione}: \a{linea obliqua} della
      \tol{cartilagine tiroidea}
    \item
      \textbf{Azione}: abbassa lo ioide
    \end{itemize}
  \end{itemize}
\end{itemize}

\hypertarget{triangolo-posteriore}{%
\subsection{Triangolo posteriore}\label{triangolo-posteriore}}

\footnotesize

\emph{Di fatto è la regione \emph{laterale} del collo}, ed è
principalmente muscolare \normalsize

\begin{itemize}
\tightlist
\item
  Muscolatura superficiale: \marginnote{\netter{25}}

  \begin{enumerate}
  \def\labelenumi{\arabic{enumi}.}
  \tightlist
  \item
    M. \mus{platisma} --- muscolo mimico superficiale situato nello
    spessore dello SMAS

    \begin{itemize}
    \tightlist
    \item
      \textbf{Origine}: margine inferiore della \oss{mandibola}. Decorre
      poi come lamina quadrangolare
    \item
      \textbf{Inserzione}: non ha una vera e propria inserzione: si
      disperde nella regione \emph{sottoclaveare} e in quella
      \emph{sottoacromiale}
    \item
      \textbf{Azione}: tensione dello strato cutaneo che lo accoglie
    \item
      \textbf{Innervazione} \ner{faciale}~
    \end{itemize}
  \item
    M. \mus{sternocleidomastoideo} --- muscolo che divide il collo nel
    triangolo anteriore e nel triangolo posteriore

    \begin{itemize}
    \tightlist
    \item
      \textbf{Origine}: \a{processo mastoideo} e
      \a{linea nucale superiore}. Decorre avanti e inferiormente,
      dividendosi in 2 capi
    \item
      \textbf{Inserzione}:

      \begin{itemize}
      \tightlist
      \item
        Capo sternale: tendine di forma conica che si innesta nel
        manubrio dello \oss{sterno}~
      \item
        Capo clavicolare: lamina che si innesta nel margine superiore
        della \a{clavicola}
      \end{itemize}
    \item
      \textbf{Azione}: flessione o rotazione del capo (se si contraggono
      i muscoli di entrambi i lati o solo di un lato); espansione del
      torace (se punto fisso è a livello della nuca)
    \end{itemize}
  \end{enumerate}
\item
  Muscolatura profonda \marginnote{\netter{29}}

  \begin{enumerate}
  \def\labelenumi{\arabic{enumi}.}
  \tightlist
  \item
    Mm. \mus{scaleni}: tris di muscoli disposti a ``scala dei pompieri''
    \marginnote{\netter{30}}

    \begin{itemize}
    \tightlist
    \item
      \textbf{Origine}: \a{processi trasversi} delle vertebre, a partire
      da \oss{c2}
    \item
      \textbf{Inserzione}: prima costa (scaleno anteriore e medio) o
      seconda costa (scaleno posteriore)
    \item
      \textbf{Disposizione}: il posteriore copre il medio, che a sua
      volta copre l'anteriore
    \item
      \textbf{Rapporti}

      \begin{itemize}
      \tightlist
      \item
        \{plesso brachiale\} (accolto nello spazio interscalenico)
      \item
        Vena \ven{succlavia} (decorre davanti allo scaleno anteriore
      \item
        Arteria \art{succlavia} (decorre posteriormente allo scaleno
        anteriore)
      \end{itemize}
    \end{itemize}
  \end{enumerate}
\end{itemize}

\hypertarget{tiroide}{%
\subsection{Tiroide}\label{tiroide}}

\begin{itemize}
\tightlist
\item
  \textbf{Ghiandola endocrina} situata a contatto con i primi anelli
  tracheali (tra \oss{c5} e \oss{t1})
\item
  È composta da 3 parti

  \begin{itemize}
  \tightlist
  \item
    Una porzione centrale (\a{istmo}) sottile
  \item
    Due \a{lobi} laterali, di forma conica (apice rivolto verso l'alto)
  \item
    Un eventuale \a{processo piramidale} che si prolunga verso l'alto
  \end{itemize}
\item
  Rapporti

  \begin{itemize}
  \tightlist
  \item
    Anteriormente: \mus{sternotiroideo}
  \item
    Posteriormente

    \begin{itemize}
    \tightlist
    \item
      Anelli tracheali
    \item
      \a{paratiroide} (4 ghiandole totali, 2 per lato)
    \item
      Nervo \ner{laringeo ricorrente}
    \end{itemize}
  \end{itemize}
\item
  Vascolarizzazione arteriosa

  \begin{enumerate}
  \def\labelenumi{\arabic{enumi}.}
  \tightlist
  \item
    Arteria \art{tiroidea superiore}

    \begin{itemize}
    \tightlist
    \item
      \textless{} \art{carotide esterna}
    \item
      Anastomizza con controlaterale
    \end{itemize}
  \item
    Arteria \art{tiroidea inferiore}

    \begin{itemize}
    \tightlist
    \item
      \textless{} \art{tronco tireocervicale}
    \item
      Anastomizza con \emph{tiroidea superiore}
    \end{itemize}
  \end{enumerate}

  \begin{itemize}
  \tightlist
  \item
    Talvolta riceve anche l'arteria \art{tiroidea ima} (originata in
    maniera variabile da arco aortico o dal tronco brachiocefalico)
  \end{itemize}
\item
  Ritorno venoso

  \begin{enumerate}
  \def\labelenumi{\arabic{enumi}.}
  \tightlist
  \item
    \ven{tiroidea superiore} \textgreater{} \ven{giugulare interna}
  \item
    \ven{tiroidea media} \textgreater{} \ven{giugulare interna}
  \item
    \ven{tiroidea inferiore} \textgreater{} \ven{brachiocefalica}
  \end{enumerate}
\end{itemize}

\footnotesize

Sia la vascolarizzazione che il ritorno vanno a creare due reti
indipendenti a livello della capsula. La regione che determinano è detta
``\emph{spazio pericoloso}'', perché è una regione che, durante gli
interventi alla tiroide, è ad alto rischio emorragico \normalsize

\clearpage
\part{Apparato digerente}

\hypertarget{tratto-prossimale-testacollo}{%
\section{Tratto prossimale
(testa/collo)}\label{tratto-prossimale-testacollo}}

\hypertarget{faringe}{%
\subsection{Faringe}\label{faringe}}

\begin{itemize}
\tightlist
\item
  La faringe è una struttura muscolare semicilindrica, aperta nella sua
  parte anteriore

  \begin{itemize}
  \tightlist
  \item
    2 pareti laterali muscolari
  \item
    Una parete posteriore fortemente discontinua
  \end{itemize}

  Le pareti si inseriscono nelle strutture anteriori alla faringe:
  cavità nasali, cavità orale e laringe
\end{itemize}

\centering

\includegraphics[width=\textwidth,height=8cm]{img/faringe-schema.png}~
\justify

\begin{itemize}
\tightlist
\item
  Faringe: Base cranica → \oss{C6} (\textasciitilde{} 15 cm), dopo si
  parla di \a{esofago}

  \begin{enumerate}
  \def\labelenumi{\arabic{enumi}.}
  \tightlist
  \item
    \a{rinofaringe} (\textasciitilde{} 4.5 cm)
  \item
    \a{orofaringe} (\textasciitilde{} 4 cm)
  \item
    \a{laringofaringe} (\textasciitilde{} 5 cm)
  \end{enumerate}
\item
  La faringe è composta da una sovrapposizione di varie tonache (ext →
  int)

  \begin{itemize}
  \tightlist
  \item
    Rivestimento fasciale (che di fatto è la \tol{guaina viscerale} del
    collo)
  \item
    Piano muscolare
  \item
    Aponeurosi faringea (struttura connettivale particolarmente robusta,
    che funge da ancoraggio per tutta la faringe)
  \item
    Tonaca mucosa (in continuità con le mucose che ricoprono le cavità
    che si aprono nella faringe)
  \end{itemize}
\end{itemize}

\hypertarget{guaina-perifaringea}{%
\subsubsection{Guaina perifaringea}\label{guaina-perifaringea}}

\begin{itemize}
\tightlist
\item
  Guaina che racchiude la tonaca muscolare e sarà in continuità con la
  \tol{guaina viscerale}
\item
  Tra guaina perifaringea e \tol{fascia cervicale profonda} si apre uno
  spazio (\a{spazio retrofaringeo} prima e \a{spazio retroesofageo}
  dopo, a seconda dell'altezza alla quale si conduce la sezione)

  \begin{itemize}
  \tightlist
  \item
    Normalmente riempito da tessuto fibroadiposo
  \item
    È spazio che permette la diffusione
  \end{itemize}
\end{itemize}

\hypertarget{superficie-interna-orofaringea}{%
\subparagraph{Superficie interna
orofaringea}\label{superficie-interna-orofaringea}}

\begin{itemize}
\tightlist
\item
  \{tonsilla palatina\}~

  \begin{itemize}
  \tightlist
  \item
    \textbf{Vascolarizzazione}

    \begin{itemize}
    \tightlist
    \item
      \art{tonsillare} (dalla \art{palatina ascendente})
    \item
      A volte sono presenti rami ceduti dalla
      \art{palatina tonsillare superiore} (dalla
      \art{palatina discendente} per la parte superiore
    \item
      A volte sono presenti rami ceduti dalla \art{dorsale della lingua}
      per la parte inferiore
    \end{itemize}
  \end{itemize}
\item
  \a{laringofaringe}~

  \begin{itemize}
  \tightlist
  \item
    Notiamo le \a{fosse piriformi}, per convogliare il bolo verso
    l'\a{esofago}
  \end{itemize}
\end{itemize}

\hypertarget{rapporti}{%
\subparagraph{Rapporti}\label{rapporti}}

\begin{itemize}
\tightlist
\item
  \textbf{Laterali}: con \a{regione carotidea}\footnote{Tra
    \mus{sternocleidomastoideo}, \oss{colonna vertebrale} e \a{faringe}}
  Sotto lo sternocleido vediamo il \textbf{fascio vascolonervoso del
  collo} (\ven{giugulare}, \art{carotide} e \ner{vago}, \nnetter{76})
\end{itemize}

\hypertarget{tonaca-muscolare}{%
\subsubsection{\texorpdfstring{Tonaca muscolare
(\nnetter{75})}{Tonaca muscolare ()}}\label{tonaca-muscolare}}

\begin{itemize}
\tightlist
\item
  \textbf{Muscoli costrittori} -- sono 3 muscoli che riducono il lume
  della faringe. La disposizione delle fibre assicura che il lume sia
  sempre pervio - Sono disposti come \textbf{tegole rovesciate}, con una
  parte dell'inferiore che copre parzialmente il medio, il quale a sua
  volta copre parzialmente il superiore 1. \mus{costrittore inferiore}~
  - Costituisce, con i due fasci, un anello attorno alla porzione
  distale della faringe - \textbf{Fascio tiroideo} (si inserisce sulla
  linea obliqua della \textbackslash{}tol\{cartilagine tiroidea) -
  \textbf{Fascio cricoideo} (si inserisce sulla
  \tol{cartilagine tirooidea} - I due fasci controlaterali convergono
  sul \tol{rafe faringeo} mediano - Le fibre sono disposte a ventaglio
  2. \mus{costrittore medio}~ - Le fibre sono disposte sempre a
  ventaglio, con le fibre inferiori che si portano decisamente verso il
  basso (e infatti sono nascoste dal costrittore inferiore). Quelle
  superiori si portano verso l'alto - Origina con due fasci, uno dal
  \a{piccolo corno} e l'altro dal \tol{legamento stiloioideo} 3.
  \mus{costrittore superiore} - \textbf{Origine} - Una porzione parte
  dalla \a{lingua} - Una porzione parte dalla \a{lina milo-ioidea} -
  \textbf{Inserzione}: \tol{rafe pterigomandibolare}, nella parte
  posteriore (va a sfiorare la \a{lamina pterigoidea mediale}) - Anche
  qui le fibre sono a ventaglio: le fibre inferiori si portano in basso,
  coperte dal costrittore medio - Le superiori vanno a fissarsi sul
  \a{tubercolo faringeo} - Siccome le inserzioni sono laterali, la
  contrazione dei costrittori \textbf{restringe ma non chiude} il lume
  della faringe - \textbf{Piega di \TODO{Passavat}} -- si ritrova nella
  parte interna alta della faringe, e funge da punto di arresto (e
  anche, in conseguenza, da sigillo) per la chiusura del palato molle
\item
  \textbf{Muscoli elevatori} -- 2 muscoli con andamento decisamente
  verticale, molto sottili

  \begin{itemize}
  \item
    \mus{palatofaringeo}

    \begin{itemize}
    \item
      \TODO{}

      \begin{itemize}
      \tightlist
      \item
        \textbf{Inserzione}

        \begin{itemize}
        \tightlist
        \item
          Un fascio sull margine dorsocraniale della
          \tol{cartilagine tiroidea}
        \end{itemize}
      \item
        \textbf{Azione}: eleva, accorciandola, la faringe
      \end{itemize}
    \end{itemize}
  \item
    \mus{stilofaringeo}

    \begin{itemize}
    \item
      \TODO{}
    \end{itemize}
  \end{itemize}
\item
  \textbf{Vascolarizzazione} - \TODO{}

  \begin{itemize}
  \item
    \a{faringea ascendente}
  \item
    \a{faringea}
  \item
    \a{palatina ascendente}
  \end{itemize}
\item
  \textbf{Innervazione}: a carico del \ner{plesso faringeo}, servito dal
  \ner{vago}
\end{itemize}

\hypertarget{aponeurosi-faringea}{%
\subsubsection{Aponeurosi faringea}\label{aponeurosi-faringea}}

\begin{itemize}
\tightlist
\item
  Struttura connettivale \emph{continua}\footnote{Si sviluppa senza
    interruzioni dalla base cranica fino al tratto digerente (cosa per
    esempio non vera per la tonaca muscolare)} che permette l'ancoraggio
  della sovrastante tonaca muscolare

  \begin{itemize}
  \tightlist
  \item
    Inserzione superiore (a forma di U): base cranica,
    \a{tubercolo faringeo}, membrana del \a{foro lacero}, scavalca la
    cartilagine della tuba, si aggancia alla \emph{lamina mediale} del
    \a{processo pterigoideo}
  \item
    Inserzione anteriore: \a{processo pterigoideo},
    \tol{rafe pterigomandibolare}, \a{linea miloioidea},
    \tol{legamento stiloioideo} \a{grande corno} dello \oss{ioide},
    \tol{legamento tiroideo laterale} e \a{linea obliqua} della
    \tol{cartilagine tiroidea}, superficie posterolaterale della
    \tol{cartilagine cricoidea}
  \item
    Inserzione inferiore: \emph{tonaca mucosa} dell'\a{esofago}
  \end{itemize}
\end{itemize}

\hypertarget{tonaca-mucosa}{%
\subsubsection{Tonaca mucosa}\label{tonaca-mucosa}}

\hypertarget{vascolarizzazione-1}{%
\subsubsection{Vascolarizzazione}\label{vascolarizzazione-1}}

\hypertarget{innervazione-2}{%
\subsubsection{Innervazione}\label{innervazione-2}}

\hypertarget{esofago}{%
\subsection{Esofago}\label{esofago}}

\begin{itemize}
\tightlist
\item
  Inizio del tubo digerente
\item
  Molto lungo, (24-35 cm). Transita su molte zone

  \begin{itemize}
  \tightlist
  \item
    Inizia attorno a \oss{c6}
  \item
    Termina con l'\a{orifizio cardiale}, attorno a \oss{t10}/\oss{t11}.
    La \a{giunzione cardiale} forma, con il fondo dello stomaco, un
    angolo, detto \a{angolo di his}
  \end{itemize}
\item
  Sagittalmente sono ben evidenti i rapporti con le strutture più
  dorsali all'esofago

  \begin{itemize}
  \tightlist
  \item
    Cranialmente ha un rapporto diretto con la colonna
  \item
    Nel mediastino, l'arco dell'aorta sposta l'esofago e ne prende il
    posto. A livello di \oss{t5} con la colonna è in rapporto
    l'\art{aorta toracica}
  \item
    Caudalmente, l'esofago si trova più lateralmente a destra separato
    dalla colonna tramite l'aorta
  \end{itemize}
\item
  Notiamo, da una visione frontale, 2 convessità

  \begin{itemize}
  \tightlist
  \item
    Una verso sinistra, in corrispondenza dell'arco aortico. Infatti
    notiamo l'impronta esofagea sul polmone destro
  \item
    Caudalmente, si porta decisamente verso sinistra (lo stomaco è a
    sinistra) e l'impronta la troviamo sul polmone sinistro
  \end{itemize}
\item
  L'esofago, pur essendo sempre pervio, è schiacciato, se vuoto. In
  stato di riempimento si notano \textbf{dei restringimenti}

  \begin{itemize}
  \tightlist
  \item
    \a{restrigimento superiore} dovuto al \a{fascio cricoideo} della
    laringe che spinge contro l'esofago
  \item
    \a{restringimento mediano} tra la biforcazione tracheale e l'arco
    aortico, perché queste strutture spingono sull'esofago
  \item
    \a{restringimento inferiore}\footnote{anche detto impropriamente
      \a{sfintere esofageo}. È uno sfintere solo funzionale, non
      anatomico (non troviamo un ispessimento di muscolatura che
      caratterizza anatomicamente uno sfintere)} nel momento in cui
    l'esofago passa per l'\a{orifizio esofageo} muscolare diaframmatico
  \end{itemize}
\end{itemize}

\hypertarget{rapporti-1}{%
\subsubsection{Rapporti}\label{rapporti-1}}

\begin{enumerate}
\def\labelenumi{\arabic{enumi}.}
\tightlist
\item
  Tratto cervicale

  \begin{itemize}
  \tightlist
  \item
    Anteriormente: \a{trachea} (nella sua porzione membranacea\footnote{La
      quale scambia fibre con lo strato longitudinale esterno della
      muscolatura dell'esofago (\netter{231})})
  \item
    Lateralmente: nervo \ner{laringeo ricorrente} (tratto ascendente e
    discendente)
  \item
    Posteriormente: la \tol{fascia cervicale profonda} delimita lo
    \a{spazio retroesofageo}, dorsalmente al quale si trova la
    \a{colonna vertebrale}
  \end{itemize}
\item
  Tratto toracico
  \marginnote{Occorre distinguere una sezione che precede l'arco aortico e una che lo segue, perché l'aorta si dispone in modo da \"prendere il posto\" dell'esofago, correndo all'indietro e leggermente verso sx}

  \begin{itemize}
  \tightlist
  \item
    Sezione sovra-aortica (\textasciitilde{} T3)

    \begin{itemize}
    \tightlist
    \item
      Anteriormente: trachea (che si biforca)
    \item
      Lateralmente: \ner{vago} di dx e di sx, \a{pleura} dx e sx
    \item
      Posteriormente: sempre colonna vertebrale, \lin{dotto toracico}
    \end{itemize}
  \item
    Sezione sotto-aortica (\textasciitilde{} T7)

    \begin{itemize}
    \tightlist
    \item
      Anteriormente: lato sx del \a{cuore} (mediato da \emph{pericardio}
      e dal \emph{seno obliquo})
    \item
      Lateralmente: \a{pleura} mediastinica e il \ner{vago} (di dx o sx)
    \item
      Posteriormente: a sx con \art{aorta} discendente, mentre a dx con
      la vena \ven{azygos}. Tra le due contrae rapporto con il
      \lin{dotto toracico}\footnote{L'andamento del \lin{dotto toracico}
        è un landmark importante nel riconoscimento delle sezioni delle
        lastreIl dotto risale, portandosi a sx, per ricevere il
        \lin{tronco broncomediastinico sinistro}, il
        \lin{tronco giugulare} e il \lin{tronco succlavio}. Termina
        nell'\a{angolo giugulosucclavio} (incrocio tra \ven{succlavia} e
        \ven{giugulare interna}. L'andamento complessivo lo porta a
        \textbf{incrociare la parete dell'esofago da sinistra verso
        destra}}
    \end{itemize}

    \begin{enumerate}
    \def\labelenumii{\arabic{enumii}.}
    \setcounter{enumii}{2}
    \tightlist
    \item
      Tratto diaframmatico\footnote{Porzione molto breve
        (\textasciitilde{} 1cm), perché il tratto diaframmatico
        individua solo il segmento che attraversa il diaframma. Qui si
        scambiano fibrocellule tra la porzione esofagea e la porzione
        del diaframma (\emph{legamenti frenoesofagei})}
      \marginnote{\netter{232}}
    \end{enumerate}
  \item
    \tol{legamento frenoesofageo superiore} --- legamento risultante
    dallo scambio di fibrocellule tra diaframma e muscolatura esterna
    dell'esofago, ha la funzione di ancorare l'esofago al diaframma,
    mantenendolo in situ
  \item
    \tol{legamento frenoesofageo inferiore} --- interessa la porzione
    addominale, è più lasso e non è a contatto diretto con l'esofago, ma
    con un anello di tessuto adiposo che circonda l'esofago stesso
  \end{itemize}
\item
  Tratto addominale

  \begin{itemize}
  \item
    Anteriormente: \a{fegato} (lobo sx o legamento triangolare, a
    seconda di quanto il lobo sia esteso)
  \item
    Lateralmente: su entrambi i lati il \ner{vago}
  \item
  \end{itemize}
\end{enumerate}

\normalbox{Ernia iatale}{
Lo \a{iatus esofageo} del diaframma è uno iato \emph{esclusivamente muscolare}: è costituito dall'andata e dal ritorno delle fibre muscolari del \emph{pilastro destro} del diaframma che si organizzano in una struttura ad anello. In caso di \emph{lassità dei legamenti frenoesofagei} si possono determinare situazioni di \pat{ernia iatale} -- ovvero situazioni in cui l'esofago perde la sua posizione naturale nei confronti del diaframma: \begin{itemize} \tightlist \item \pat{ernia iatale da scivolamento} --- in caso di lassità del \tol{legamento frenoesofageo superiore} l'esofago addominale ``scivola'' nella cavità toracica \item \pat{ernia iatale paraesofagea} --- il \emph{fondo dello stomaco} è erniato a causa delle grandi dimensioni dello \a{iato esofageo} \end{itemize}
}

\hypertarget{innervazione-3}{%
\subsubsection{Innervazione}\label{innervazione-3}}

\netter{236}

\begin{itemize}
\tightlist
\item
  L'innervazione principale è di competenza del nervo \ner{vago} (dx
  \emph{e} sx)
\item
  \textbf{I due nervi vaghi, nel decorrere verso il basso, avvolgono
  l'esofago}: il vago di sinistra si porta sul lato ventrale
  dell'esofago, mentre il vago di destra si porta dietro
\item
  La collaborazione di entrambi i nervi vaghi porta alla costruzione del
  \ner{plesso esofageo}
\end{itemize}

\hypertarget{vascolarizzazione-2}{%
\subsubsection{Vascolarizzazione}\label{vascolarizzazione-2}}

\begin{itemize}
\tightlist
\item
  Siccome è un tratto molto lungo, i contributi (sia arteriosi che
  venosi) sono diversi a seconda della regione
\end{itemize}

\hypertarget{afferenze-arteriose}{%
\paragraph{Afferenze arteriose}\label{afferenze-arteriose}}

\marginnote{\netter{233}}

\begin{enumerate}
\def\labelenumi{\arabic{enumi}.}
\tightlist
\item
  Porzione cervicale: \emph{rami esofagei} dell'arteria
  \art{tiroidea inferiore}
\item
  Porzione mediastinica: rete anastomotica che si crea con contributi
  dei \emph{rami esofagei} delle aa \art{intercostali}, delle aa
  \art{bronchiali} e del tratto toracico dell'\art{aorta}
\item
  Porzione terminale: \emph{rami esofagei} dell'arteria
  \art{gastrica sinistra}
\end{enumerate}

\hypertarget{ritorno-venoso-1}{%
\paragraph{Ritorno venoso}\label{ritorno-venoso-1}}

\marginnote{\netter{234}}

\begin{enumerate}
\def\labelenumi{\arabic{enumi}.}
\tightlist
\item
  Porzione cervicale: vv \ven{tiroidee inferiori} (\textgreater{}
  \ven{brachiocefalica} omolaterale)
\item
  Porzione mediastinica: vena \ven{azygos} (a dx) o \ven{emiazygos} (a
  sx). Questa parte viene drenata da un grosso \textbf{plesso
  anastomotico periesofageo}
\item
  Porzione terminale: rami esofagei della vena
  \ven{gastrica sinistra}\footnote{Anche detta vena
    \ven{coronaria dello stomaco}} (\textgreater{} \ven{vena porta})
\end{enumerate}

\hypertarget{muscolatura-dellesofago}{%
\subsubsection{Muscolatura dell'esofago}\label{muscolatura-dellesofago}}

\netter{67, 70, 231}

\begin{itemize}
\tightlist
\item
  Muscolatura liscia
\item
  Non passaggio netto tra muscolatura striata della faringe e
  muscolatura liscia esofagea: c'è un cambiamento graduale (vedi
  microscopica)
\item
  In visione posteriore si notano

  \begin{itemize}
  \tightlist
  \item
    \mus{costrittore inferiore della faringe} --- visibili i suoi
    \textbf{due capi} (\emph{tiroideo} e \emph{cricoideo})

    \begin{itemize}
    \tightlist
    \item
      Capo tiroideo --- costruisce la parete prossimale posteriore
      dell'esofago
    \item
      Capo cricoideo --- è circolare, ed è origine della
      \textbf{muscolatura dello strato interno dell'esofago}, anch'essa
      circolare (vedi micro)
    \end{itemize}
  \item
    \textbf{Muscolatura longitudinale}

    \begin{itemize}
    \tightlist
    \item
      Origina nella faccia ventrale dell'esofago, dal lato dorsale della
      \tol{cartilagine cricoidea}
    \item
      Nel discendere \textbf{si apre a ventaglio}, a coprire tutta la
      superficie dell'esofago
    \item
      Nel punto in cui i fasci si riuniscono\footnote{Nella parte
        prossimale, subito sotto al capo cricoideo del
        \mus{costrittore inferiore della faringe}} si determina una zona
      a V \emph{libera} dalla muscolatura longitudinale (si vede solo
      quella circolare): è il \a{triangolo di laimer}
      \marginnote{Limiti: S. dal fascio cricoideo, inferiormente i due capi di muscolatura longitudinale che si separano a costruire la V}

      \begin{itemize}
      \tightlist
      \item
        Questa zona è a ↓ resistenza ⇒ se ↑ pressione intraesofagea
        \emph{è possibile l'erniazione della mucosa} (con formazione di
        un \pat{diverticolo di zenker} a fondo cieco)
        \asidefigure{img/zenker.png}{}
      \end{itemize}
    \end{itemize}
  \end{itemize}
\end{itemize}

\hypertarget{addome}{%
\section{Addome}\label{addome}}

\normalbox{Anatomia di superficie (landmark)}{
\begin{itemize}
\tightlist
\item \a{linea alba} --- linea mediana verticale che corre dal \a{processo xifoideo} alla \a{sinfisi pubica}. È un addensamento fibroso costruito dall'unione delle aponeurosi del \mus{obliquo destro} e \mus{obliquo sinistro}
\item \a{cicatrice ombelicale} --- proiettata tra L3 ed L4 in un individuo magro
\item \a{iscrizioni tendinee} --- laterali e perpendicolari alla linea alba, sono i tendini del \mus{retto dell'addome}
\item \a{linee semilunari} --- margini laterali del \mus{retto dell'addome}
\item \a{piega inguinale} --- limite inferiore della parete addominale; tesa tra \a{sinfisi pubica} e \a{cresta iliaca}
\end{itemize}

}

\begin{itemize}
\tightlist
\item
  Suddivisa in 9 sottoregioni da \marginnote{\netter{244}}

  \begin{itemize}
  \tightlist
  \item
    2 piani verticali: \textbf{piano emiclaveare} dx e sx
  \item
    2 piani trasversali: \textbf{piano transpilorica} (\oss{l1}) e
    \textbf{piano intertubercolare} (\oss{l5})
  \end{itemize}
\item
  Le 9 regioni sono

  \begin{itemize}
  \tightlist
  \item
    Sopra il piano transpilorico: \a{regione epigastrica}: zona
    dell'\a{ipocondrio destro}, dell'\a{epigastrio},
    dell'\a{ipocondrio sinistro}
  \item
    Intermedia: \a{regione mesogastrica}: regione \a{lombare destra},
    del \a{mesogastrio}, \a{lombare sinistra}
  \item
    Inferiore: \a{regione ipogastrica}: regione \a{iliaca destra},
    dell'\a{ipogastrio}, \a{iliaca sinistra}\footnote{Regioni
      \a{inguinali} se come line inferiore individuiamo quella tra le
      creste iliache}
  \end{itemize}
\end{itemize}

\hypertarget{parete-addominale}{%
\subsection{Parete addominale}\label{parete-addominale}}

\begin{enumerate}
\def\labelenumi{\arabic{enumi}.}
\tightlist
\item
  Cute e sottocute\footnote{Facilmente svincolabili tranne che nella
    regione periombelicale, dove invece sono strettamente aderenti}

  \begin{itemize}
  \tightlist
  \item
    Sottocute superficiale (\a{fascia di Camper}) --- strato adiposo
    sottocutaneo\footnote{Il pannicolo adiposo sottocutaneo}
  \item
    Sottocute profonda (\a{fascia fibrosa di Scarpa}) --- fascia fibrosa
    molto robusta
  \end{itemize}
\item
  \a{fascia trasversalis} --- piano fasciale che riveste i muscoli
  sottostanti
\item
  Piano muscolare
\item
  Tessuto adiposo profondo
\item
  \a{peritoneo} (\emph{foglietto parietale})
\end{enumerate}

\hypertarget{piano-muscolare}{%
\subsubsection{Piano muscolare}\label{piano-muscolare}}

\hypertarget{regione-anteriore}{%
\paragraph{Regione anteriore}\label{regione-anteriore}}

\begin{itemize}
\tightlist
\item
  Mm. \mus{retti dell'addome}~

  \begin{itemize}
  \tightlist
  \item
    Ricoperto dalla apposita \tol{guaina dei retti}
  \item
    Inframezzato da 3 \tol{iscrizioni tendinee} trasversali
  \item
    \textbf{Origine}: \a{processo xifoideo} + cartilagini costali della
    \oss{costa} 5, 6 e 7
  \item
    \textbf{Inserzione}: le fibre corrono verticali verso il basso e si
    inseriscono subito lateralmente alla \a{sinfisi pubica}
  \end{itemize}
\item
  M. \mus{piramidale}~

  \begin{itemize}
  \tightlist
  \item
    Posto ventralmente al \emph{retto dell'addome}, è molto sottile e
    corre parallelo alla \emph{linea alba}
  \item
    \textbf{Origine}: \a{processo xifoideo}
  \item
    \textbf{Inserzione}: lateralmente alla \a{sinfisi pubica} (la stessa
    del retto)
  \end{itemize}
\end{itemize}

\hypertarget{regione-laterale}{%
\paragraph{Regione laterale}\label{regione-laterale}}

\begin{itemize}
\tightlist
\item
  M. \mus{obliquo esterno}~

  \begin{itemize}
  \tightlist
  \item
    \textbf{Origine}

    \begin{itemize}
    \tightlist
    \item
      7/ma e 8/a \a{costa}
    \item
      Alcuni fasci partono dal \mus{dentato anteriore} (anteriormente)
    \item
      Alcuni fasci partono dal \mus{grande dorsale} (posteriormente)
    \end{itemize}
  \item
    \textbf{Inserzione}: le fibre si aprono a ventaglio e in avanti, a
    direzione \emph{mani in tasca}, per risolversi in una larga
    aponeurosi (\tol{aponeurosi dell'obliquo esterno}). Questa copre
    completamente il retto dell'addome e si inserisce in due punti

    \begin{enumerate}
    \def\labelenumi{\arabic{enumi}.}
    \tightlist
    \item
      labbro esterno della \a{cresta iliaca} (da \a{spina iliaca}
      anteriore) → \a{tubercolo pubico}. Il margine inferiore di questa
      sezione è libero, e si ispessisce a formare il
      \tol{legamento inguinale}
    \item
      \a{linea alba}~
    \end{enumerate}
  \end{itemize}
\item
  M. \mus{obliquo interno}~

  \begin{itemize}
  \tightlist
  \item
    \textbf{Origine}: 2/3 superiori del \tol{legamento inguinale}, a
    partire dalla \a{spina iliaca} anteriore
  \item
    \textbf{Inserzione}:

    \begin{itemize}
    \tightlist
    \item
      Fascia \tol{toracolombare} (dorso)
    \item
      \tol{aponeurosi dell'obliquo interno} (ventre), che va a formare
      la \tol{guaina dei retti}
    \item
      Cartilagini costali delle ultime 4 coste
    \end{itemize}
  \item
    Le fibre hanno direzione opposta rispetto a quelle dell'obliquo
    esterno
  \end{itemize}
\end{itemize}

\includegraphics{img/trasversi-addome.png}~

\begin{itemize}
\tightlist
\item
  M. \mus{trasverso dell'addome}~

  \begin{itemize}
  \tightlist
  \item
    \textbf{Origine}: come l'\emph{obliquo interno}, sul
    \tol{legamento inguinale}
  \item
    \textbf{Inserzione}

    \begin{itemize}
    \item
      Dalla 7/ma alla 12/ma costa
    \item
      \tol{fascia toracolombare}
    \item
      Partecipano alla \tol{guaina dei retti}
    \item
      Le fibre della porzioni inferiori si risolvono in un \emph{tendine
      congiunto} che si fissa sulla \a{cresta pettinea}
    \end{itemize}
  \end{itemize}
\end{itemize}

\normalbox{Sezione trasversale dell'addome}{

\netter{248}

Individuiamo 3 elementi principali

\begin{itemize}
\tightlist
\item
  I muscoli \mus{retti dell'addome}
\item
  La \tol{guaina dei retti}, che li avvolge. Questa è formata dalla
  collaborazione delle aponeurosi con cui terminano i 3 muscoli della
  parete addominale
\item
  I 3 muscoli della parete addominale (obliquo esterno, obliquo interno
  e trasverso)
\end{itemize}

La guaina dei retti nei suoi 3/4 superiori si comporta differente
rispetto all'ultimo quarto\footnote{Il passaggio tra una regione e
  l'altra è marcato dalla \a{linea arcuata}}:

\begin{itemize}
\tightlist
\item
  Nei 3/4 superiori --- abbraccia i \emph{retti} sia ventralmente che
  dorsalmente:

  \begin{itemize}
  \tightlist
  \item
    L'aponeurosi dell'obliquo esterno passa ventralmente ai retti
  \item
    L'aponeurosi dell'obliquo interno si sdoppia in due foglietti, uno
    che passa ventralmente ai retti e uno che passa dorsalmente
  \item
    L'aponeurosi del trasverso passa dorsalmente ai retti
  \end{itemize}
\item
  Nell'ultimo quarto --- tutte e tre le aponeurosi passano ventralmente
  ai muscoli retti dell'addome, per convergere sulla linea alba
\end{itemize}

Ventralmente alla fascia dei retti, sotto a tutti e tre gli ordini di
muscoli della parete, si ritrova la \tol{fascia trasversalis}
}

\hypertarget{vascolarizzazione-3}{%
\subsubsection{Vascolarizzazione}\label{vascolarizzazione-3}}

\begin{itemize}
\tightlist
\item
  Afferenza \asidenote{\netter{251}}

  \begin{enumerate}
  \def\labelenumi{\arabic{enumi}.}
  \tightlist
  \item
    Rete anastomotica superficiale --- composta primariamente
    dall'arteria \art{epigastrica superficiale} (\textless{}
    \art{femorale}) e dalla \art{circonflessa iliaca superficiale}. A
    questo si aggiungono i \emph{rami perforanti} dalla rete profonda
  \item
    \goldstandard rete profonda --- arteria \art{epigastrica inferiore}
    (\textless{} \art{iliaca esterna}) ed \art{epigastrica superiore}
    (\textless{} \art{mammaria interna}

    \begin{itemize}
    \tightlist
    \item
      L'anastomosi che si origina tra un ramo della \emph{mammaria
      interna} e dell'\emph{iliaca esterna} è detta
      \art{circonflessa iliaca profonda}
    \end{itemize}
  \end{enumerate}
\item
  Ritorno venoso \asidenote{\netter{252}}
\end{itemize}

\hypertarget{innervazione-4}{%
\subsubsection{Innervazione}\label{innervazione-4}}

\hypertarget{canale-inguinale}{%
\subsection{Canale inguinale}\label{canale-inguinale}}

\hypertarget{triangolo-di-hesselbach}{%
\subsubsection{Triangolo di Hesselbach}\label{triangolo-di-hesselbach}}

\begin{itemize}
\tightlist
\item
  Limiti

  \begin{itemize}
  \tightlist
  \item
    Caudo-laterlmente: \tol{legamento inguinale}
  \item
    Cranialmente: vasi \a{epigastrici inferiori}
  \item
    Medialmente: margine laterale del letto
  \end{itemize}
\item
  È coperto superficialmente dalla sola \tol{fascia trasversalis}: è la
  parte più debole. Se ↑ pressione addominale, l'ansa intestinale può
  erniare

  \begin{itemize}
  \tightlist
  \item
    \pat{ernia inguinale diretta}: l'ansa intestinale entra nel canale
    inguinale tramite l'\a{orifizio inguinale profondo} e fuoriesce
    \emph{direttamente} (senza passare per il canale) dal
    \a{orifizio inguinale superficiale}
  \item
    \pat{ernia inguinale indiretta}: l'ansa intestinale entra nel canale
    inguinale tramite l'\a{orifizio inguinale profondo}, segue il
    decorso del canale e fuoriesce nel funicolo spermatico
  \end{itemize}
\end{itemize}

\hypertarget{cavituxe0-addomino-pelvica}{%
\subsection{Cavità addomino-pelvica}\label{cavituxe0-addomino-pelvica}}

\hypertarget{peritoneo}{%
\subsubsection{Peritoneo}\label{peritoneo}}

\begin{itemize}
\tightlist
\item
  \textbf{Sierosa} più vasta e complessa di tutto il corpo

  \begin{itemize}
  \tightlist
  \item
    In rapporto con molti organi, alcuni mobili (e saranno completamente
    peritoneizzati\footnote{Ricoperti completamente dal peritoneo} per
    ancorarli) altri fissi
  \end{itemize}
\item
  Identifichiamo 2 fogli, che compongono una struttura unica (il
  foglietto esterno si riflette sulle pareti per formare il secondo
  foglietto) \a{peritoneo parietale}: riveste in maniera aderente la
  parete interna della parete addominale anteriore
  \a{peritoneo viscerale}: riveste in maniera aderente tutti i segmenti
  peritoneizzati

  \begin{itemize}
  \tightlist
  \item
    \a{cavità peritoneale}: cavità virtuale tra i due peritonei.
    Contiene un velo di liquido a base acquosa per scorrimenti, insieme
    a proteine e macrofagi
  \end{itemize}
\end{itemize}

\hypertarget{terminologia-del-peritoneo}{%
\paragraph{Terminologia del
peritoneo}\label{terminologia-del-peritoneo}}

\begin{itemize}
\tightlist
\item
  \a{lamina peritoneale}: una generica porzione di peritoneo piuttosto
  estesa. Può essere data indifferentemente di uno, due o quattro
  foglietti
\item
  \a{omento} (o \a{epiploon}): lamina peritoneale che, in particolare,
  si stacca dallo stomaco. Individuiamo un \a{piccolo omento} ed un
  \a{grande omento}
\item
  \a{meso}: tratto di peritoneo che si stacca dal
  \a{peritoneo parietale} per ancorarsi ad un segmento intestinale,
  diventando in questo momento \a{peritoneo viscerale}. Ha il ruolo di
  \emph{sospendere} l'ansa intestinale, permettendo passaggio di vasi e
  nervi che sovraintendono ad un tratto di intestino
\item
  \a{radice}: tratto compreso tra il foglietto di ``andata'' e di
  ``ritorno'' nel momento in cui i foglietti si staccano dalla parete
  per andare a costituire una riflessione del peritoneo
\item
  \a{legamenti peritoneali}: pieghe o lamine di peritoneo che uniscono
  organi tra loro o alla parete addominale. Spesso, ma non sempre, danno
  passaggio a vasi e/o nervi
\item
  \a{cavo} (o \a{sfondato}): una depressione ampia delimitata da organi
  rivestiti da peritoneo\footnote{Esempio: \a{cavo retto-vescicale} --
    la cavità che, nella pelvi maschile, è visualizzabile tra retto e
    vescica}
\item
  \a{recessi} e \a{fossette}: piccole depressioni che si realizzano tra
  un organo peritoneizzato e la parete, o tra una piega e una parete.
  Succede spesso da un incompleto accollamento del peritoneo alla parete
  (che in pratica lascia un buco)
\item
  Parliamo di un processo di \textbf{accollamento} (o
  \textbf{coalescenza}) nel momento in cui un meso, che normalmente è
  indipendente, aderisce (\_si accolla, appunto) ad un tratto di parete
  tramite strutture fibrose. L'accollamento verso la parete ha come
  risultato una progressiva riduzione della lunghezza del meso e un
  progressivo allontanamento del corpo del meso dalla radice: se prima
  il meso infatti si trovava in posizione coincidente con la radice, un
  accollamento da un lato sposta il corpo del meso sempre più
  lateralmente verso la parete addominale alla quale il meso si sta
  accollando. Quando l'accollamento è completato, il meso è praticamente
  inesistente: il peritoneo parietale è oramai indistinguibile
  dall'ammasso fibroso che ha provocato l'accollamento e il viscere
  stesso rimane coperto solo dal peritoneo viscerale, trovandosi di
  fatto ``al di fuori'' del peritoneo: diventa quindi un viscere
  \emph{non più} peritoneizzato, e pertanto \textbf{retroperitoneale}
\end{itemize}

\hypertarget{stomaco}{%
\paragraph{Stomaco}\label{stomaco}}

\begin{itemize}
\tightlist
\item
  Porzione dilatata del canale digerente
\item
  Cranialmente in continuità con \a{esofago}, mentre caudalmente in
  continuità con il \a{duodeno}
\item
  Occupa quasi tutto l'\textbf{ipocondrio sinistro}, anche se si estende
  in parte nell'\emph{epigastrio} e in parte nell'\TODO{}
\item
  Organizzato in senso cranio-caudale, ma anche un po' sdraiato in senso
  dorso-ventrale
\end{itemize}

\hypertarget{rapporti-2}{%
\subparagraph{Rapporti}\label{rapporti-2}}

\begin{itemize}
\item
  Una parte della superficie anteriore è a diretto contatto con la
  parete addominale ventrale
\item
  Una parte (fondo sopratutto) è nascosto dalla prete toracica: dalla V
  alla IX costa. È lo \a{spazio semilunare di traube}. È in rapporto con
  il \a{seno costodiaframmatico} e con la base polmonare
\item
  Posteriormente

  \begin{itemize}
  \tightlist
  \item
    Superficie ventrale della borsa omentale
  \end{itemize}
\item
  Speriormente con la riflessione del peritoneo con il diaframma (anche
  se una parte del fondo viene lasciata libera, siccome la riflesisone
  posteriore si trova ad un livello più caudale di quella anteriore)
\item
  Laterale sx: legamento lienale e milza
\item
  \TODO{mancano un buon numero di rapporti}
\end{itemize}

\hypertarget{conformazione-esterna}{%
\subparagraph{Conformazione esterna}\label{conformazione-esterna}}

\begin{itemize}
\tightlist
\item
  Organo normalmente mobile, fissato solo da

  \begin{itemize}
  \tightlist
  \item
    Continuità con esofago
  \item
    Una parte del fondo dello stomaco, che è accollata al
    \mus{diaframma}
  \end{itemize}
\item
  A forma di bisaccia, o \texttt{J}. Individuiamo (separate
  dall'\a{incisura angolare}) \TODO{img}

  \begin{itemize}
  \tightlist
  \item
    Una porzione che si sviluppa in verticale

    \begin{itemize}
    \tightlist
    \item
      \a{fondo}~
    \item
      \a{corpo}~
    \end{itemize}
  \item
    Una porzione caudale, che si sviluppa in orizzontale

    \begin{itemize}
    \tightlist
    \item
      \a{antro pilorico}~
    \item
      \a{canale pilorico}~
    \item
      \a{sfintere pilorico}~
    \end{itemize}
  \end{itemize}
\item
  Individuiamo 2 margini: una \a{piccola curvatura} ed una
  \a{grande curvatura}
\end{itemize}

\hypertarget{conformazione-interna}{%
\subparagraph{Conformazione interna}\label{conformazione-interna}}

\hypertarget{vascolarizzazione-4}{%
\subparagraph{Vascolarizzazione}\label{vascolarizzazione-4}}

Arteriosa

\begin{itemize}
\item
  Non appena l'\art{aorta addominale} passa sotto il diaframma si
  cominciano a staccano 3 rami, che si distribuiscono a 3 segmenti:

  \begin{itemize}
  \tightlist
  \item
    Un'\art{arteria celiaca}, che si distribuisce all'intestino
    superiore
  \item
    Una \art{arteria mesenterica superiore}, che si distribuisce
    all'intestino medio
  \item
    Un'altra \TODO{arteria mesenterica inferiore}, che si distribuisce
    all'intestino inferiore
  \end{itemize}
\item
  \art{tronco celiaco}~

  \begin{itemize}
  \tightlist
  \item
    Tronco molto breve (1.5 cm max), che cede 3 rami\footnote{\TODO{img variabilitàm anatomica prometheus}}

    \begin{itemize}
    \tightlist
    \item
      \art{arteria gastrica sinistra}: origina un circolo anastomotico
      della piccola curvatura, ed uno per la grande curvatura

      \begin{itemize}
      \tightlist
      \item
        \art{arterie gastriche} della piccola curvatura
      \item
        \art{arterie gastroepiploiche} della grande curvatura
      \end{itemize}

      le quali cedono numerosi rami diretti nella parte centrale dello
      stomaco, creando anche un circolo anastomotico con i rami staccati
      dall'arcata controlaterale
    \item
      \art{arteria lienale} verso sx
    \item
      \art{arteria epatica comune} verso dx
      \TODO{da qui in poi mi sono perso, non va bene quello che c'è scritto. Impostare decorso non in funzione delle arterie, ma in funzione del circolo anastomotico tra le due curvature}
    \end{itemize}
  \end{itemize}
\item
  \art{arterie gastriche}

  \begin{itemize}
  \tightlist
  \item
    Partono retroperitoneali, dall'aorta

    \begin{itemize}
    \tightlist
    \item
      Nel punto della riflessione dei foglietti peritoneali, in cui uno
      si riflette superiormente e l'altro inferiormente, l'arteria si
      tuffa tra i due, ovvero tra i due foglietti peritoneali. Qui si
      divide in gastrica destra e sinistra

      \begin{itemize}
      \item
        Gastrica sinistra

        \begin{itemize}
        \tightlist
        \item
          Rami esofagei, verso la porzione distale dell'esofago
        \item
          Rami \art{anteriori del fondo}, diretti alla superficie
          anteriore del fondo dello stomaco
        \end{itemize}
      \item
        Gastrica destra
      \item
        \TODO{NON è vero}
      \end{itemize}
    \end{itemize}
  \end{itemize}
\item
  \art{lienale}

  \begin{itemize}
  \tightlist
  \item
    Segue margine superiore del pancreas, fino ad arrivare alla punta
    della coda, correndo retroperitoneale

    \begin{itemize}
    \tightlist
    \item
      Qui cede l'\art{arteria gastroepiploica sinistra}
    \item
      L'arteria corre nel legamento lieno-pancraeatico. A livello della
      milza, cambia decorso per andare nel legamento gastrolienale
      (deve, dopotutto, arrivare da retroperitoneale a peritoneale),
      raggiungendo in questo modo la grande curvatura dello stomaco
    \end{itemize}
  \end{itemize}
\end{itemize}

Venosa \TODO{fare meglio}

\begin{itemize}
\tightlist
\item
  La maggior parte delle vene del ritorno appartiene al sistema della
  \ven{vena porta}, tranne

  \begin{itemize}
  \tightlist
  \item
    \ven{vene esofagee inferiori}, le quali si anastomizzano con le vene
    \ven{esofagee medie} in un plesso sottomucoso che è situato
    esattamente a cavallo del diaframma (tributarie alla \ven{azygos},
    siccome dopotutto siamo nella cavità toracica). Una parte del
    plesso, la parte sottodiaframmatica, confluisce comunque nella
    \ven{gastrica sinistra} e alla \ven{porta}
  \item
    Piccoli rami della faccia posteriore del fondo
  \end{itemize}
\item
  Sistema della \ven{vena porta}

  \begin{itemize}
  \tightlist
  \item
    Formata dalla confluenza di vena \ven{lienale} e vena
    \ven{mesenterica superiore}
  \item
    Raccoglie

    \begin{itemize}
    \tightlist
    \item
      Dalla piccola curvatura

      \begin{itemize}
      \item
        \ven{gastrica sinistra}
      \item
        \ven{gastrica destra}
      \end{itemize}
    \item
      Dalla grande curvatura

      \begin{itemize}
      \item
        \ven{}
      \end{itemize}
    \item
      Vena \ven{gastroepiploica destra}

      \begin{itemize}
      \tightlist
      \item
        Confluisce insieme alla
        \ven{pancreaticoduodenale superiore anteriore} in un tronco
        comune
      \item
        Il tronco si apre nella \ven{mesenterica superiore}
      \end{itemize}
    \end{itemize}
  \end{itemize}
\end{itemize}

\hypertarget{drenaggio-linfatico-1}{%
\subparagraph{Drenaggio linfatico}\label{drenaggio-linfatico-1}}

\begin{itemize}
\tightlist
\item
  4 stazioni

  \begin{itemize}
  \tightlist
  \item
    La parte superomedial confluisce in una catena linfonodale dei
    \lin{linfonodi gastrici superiori}
  \item
    La parte inferomediale confluisce nei \lin{lnn sovrapilorici}
  \item
    La parte superolaterale confluisce nei \lin{lnn lienali}
  \item
    La parte inferolaterale confluisce nei \lin{lnnsottopilorici}
  \end{itemize}
\item
  I collettori si gettano nella \lin{cisterna del chilo}, che raccoglie
  tutta la linfa proveniente dal digerente
\end{itemize}

\hypertarget{milza}{%
\subsection{Milza}\label{milza}}

\begin{itemize}
\tightlist
\item
  Situata nella loggia lenale (sx, dorsale c/o stomaco)
\item
  Rapporti della loggia lienale

  \begin{itemize}
  \tightlist
  \item
    Ventralmente: grande curvatura dello \a{stomaco}
  \end{itemize}
\item
  Completamente peritoneizzata

  \begin{itemize}
  \tightlist
  \item
    Compliante ai movimenti diaframmatici
  \end{itemize}
\item
  Morfologia esterna

  \begin{itemize}
  \tightlist
  \item
    \textbf{Superficie laterale} (diaframmatica): liscia, convessa
  \item
    \textbf{Superficie mediale}

    \begin{itemize}
    \tightlist
    \item
      Prominenza centrale che la suddivide in due parti. Detta
      \a{prominenza dell'ilo}, perché accoglie l'ilo della milza

      \begin{itemize}
      \tightlist
      \item
        Parte superiore: impronta gastrica
      \item
        Parte inferiore: impronta del rene sinistro
      \end{itemize}
    \end{itemize}
  \item
    \textbf{Impronta della fessura colica di sinistra}

    \begin{itemize}
    \tightlist
    \item
      Si appoggia sull'angolo sinistro del \a{colon discendente}
    \item
      Da qui parte il \a{legamento freno-colico} per il diaframma, per
      maggior fissità
    \end{itemize}
  \end{itemize}
\item
  Vascolarizzazione

  \begin{itemize}
  \tightlist
  \item
    Arteriosa: principalmente per \art{arteria lienale}

    \begin{itemize}
    \tightlist
    \item
      Prima di ingresso nell'ilo: suddivisione in numerosi
      \art{rami trabecolari}
    \end{itemize}
  \item
    Venosa: satellite con \ven{vena lienale}

    \begin{itemize}
    \tightlist
    \item
      Corre 1.5 cm più caudalmente, dietro al diaframma
    \item
      Afferisce alla \ven{vena porta}
    \end{itemize}
  \end{itemize}
\end{itemize}

\hypertarget{intestino}{%
\subsection{Intestino}\label{intestino}}

\hypertarget{duodeno}{%
\subsubsection{Duodeno}\label{duodeno}}

\begin{itemize}
\tightlist
\item
  Dal punto di vista dell'organogenesi risulta formato da 2 tratti

  \begin{itemize}
  \tightlist
  \item
    Porzione caudale dell'intestino anteriore
  \item
    Porzione craniale dell'intestino medio
  \end{itemize}
\item
  Molto fisso (quasi tutto retroperitoneale)
\item
  Si divide in

  \begin{enumerate}
  \def\labelenumi{\arabic{enumi}.}
  \tightlist
  \item
    Una porzione orizzontale (\a{bulbo duodenale}), che si porta
    leggermente a sinistra e un pelo verso l'alto, ma sopratutto in
    direzione ventrodorsale
  \item
    Un \a{tratto discendente}, che corre verticale tra circa L1 ed L3.
    Nel lume troviamo 2 strutture papillari (\emph{senso
    craniocaudale}):

    \begin{itemize}
    \tightlist
    \item
      \a{papilla duodenale minore}: sbocco del \a{dotto accessorio} (o
      \a{dotto di santorini}), sempre dal pancreas
    \item
      \a{papilla duodenale maggiore}: punto di apertura di 2 dotti

      \begin{itemize}
      \item
        \a{dotto pancreatico}
      \item
        \a{coledoco}, che è anche il principale responsabile
        dell'estrusione della papilla verso il lume, siccome nel tratto
        finale intramuralizza
      \end{itemize}
    \end{itemize}
  \item
    Un \a{tratto orizzontale}, che lo riporta a destra
  \item
    Un \a{tratto ascendnte}, che raggiunge circa L2
  \end{enumerate}
\item
  Prende rapporti molto stretti con la \a{testa de pancreas}
\item
  Legato dal \a{legamento epatoduodenale}, che dà passaggio alle
  strutture che afferiscono alla vena porta (sia vene, che vasi
  linfatici)
\item
  Durante l'embriogenesi il \a{mesocolon trasverso} va a dividere
  trasversalmente il duodeno in una porzione

  \begin{itemize}
  \tightlist
  \item
    Sottomesocolica: comprende la gran parte del duodeno
  \item
    Sovramesocolica: costituita dal bulbo e dalla parte prossimale del
    tratto discendente
  \end{itemize}

  E contemporaneamente avremo anche una divisione in senso longitudinale
  grazie a \TODO{chissà cosa}
\item
  Rapporti

  \begin{itemize}
  \tightlist
  \item
    \textbf{Bulbo duodenale}

    \begin{itemize}
    \tightlist
    \item
      Ventralmente: lobo sx fegato, che si appggia sopra
    \item
      Dorsalmente: \art{arteria gastroduodenale}, \a{coledoco},
      \ven{vena porta} (tutti traversanti il
      \a{legamento epatoduodenale}, che si innesta nel
      \a{piccolo omento} formando il \a{foro epiploico})
    \end{itemize}
  \item
    \textbf{Porzione discendente}

    \begin{itemize}
    \tightlist
    \item
      A metà: divisione nelle 2 regioni mesocoliche

      \begin{itemize}
      \tightlist
      \item
        Sovramesocolica

        \begin{itemize}
        \tightlist
        \item
          Ventralmente: lobo destro del fegato
        \item
          Posteriormente: di lato rene dx, testa pancreas
        \end{itemize}
      \item
        Sottomesocolica:

        \begin{itemize}
        \tightlist
        \item
          Ventralmente: colon trasverso, ma possibili anse intestinali
          adagiate
        \item
          Dorsalente: uretere, grande psoas
        \end{itemize}
      \end{itemize}
    \end{itemize}
  \item
    \textbf{Porzione orizzontale}

    \begin{itemize}
    \tightlist
    \item
      \textbf{compasso aorto-mesenterico}: il duodeno passa tra
      \art{mesenterica superiore} e \art{aorta addominale}
    \end{itemize}
  \item
    \textbf{Porzione ascendente}

    \begin{itemize}
    \tightlist
    \item
      Fessura duodenodigiunale mantenuta in situ da un apparato
      sospensore, costituito principalmente dal
      \mus{muscolo di treiz}\footnote{Fibrocellule di provenienza
        diaframmatica che si attccano cranialmente sul gomito
        duodeno-digiunale + componente fibrosa, che costituisce anche
        l'impalcatura del \a{tripode celiaco}}
    \end{itemize}
  \end{itemize}
\item
  Vascolarizzazione

  \begin{itemize}
  \tightlist
  \item
    Arteriorsa: arcata anteriore + arcata posteriore (rispetto alla
    testa del pancreas) in anastomosi pesa \TODO{fare meglio}

    \begin{itemize}
    \item
      \art{sovraduodenale}
    \item
      \art{gastroduodenale}

      \begin{itemize}
      \item
        \art{gastroepiploica dx}

        \begin{itemize}
        \item
          \art{pancreaticoduodenale superiore}

          \begin{itemize}
          \tightlist
          \item
            Ramo superiore anteriore

            \begin{itemize}
            \tightlist
            \item
              Ramo superiore posteriore
            \end{itemize}
          \end{itemize}
        \item
          \art{pancreaticoduodenale inferiore}\footnote{Dalla
            mesenterica superiore, è il I ramo dx della SMA}

          \begin{itemize}
          \tightlist
          \item
            Ramo inferiore anteriore
          \item
            Ramo inferiore posteriore
          \end{itemize}
        \end{itemize}
      \end{itemize}
    \item
      Da \art{arteria digiunale}\footnote{SMA \textgreater{} digiunale}
      si staccano rami anteriori rispetto al duodeno
    \end{itemize}
  \item
    Venosa

    \begin{itemize}
    \tightlist
    \item
      Le due arcate convergono

      \begin{itemize}
      \tightlist
      \item
        Inferiormente nella \ven{pancreaticoduodenale inferiore}
      \item
        Superiormente

        \begin{itemize}
        \tightlist
        \item
          Il circolo superiore anteriore va, insieme alla
          \ven{gastroepiploica}, in un tronco comune che va alla
          \ven{msv}
        \item
          Il circolo superiore posteriore si apre autonomamente nella
          \ven{porta}, perché tanto questa si apre praticamente
          dorsalmente
        \end{itemize}
      \end{itemize}
    \end{itemize}
  \end{itemize}
\item
  Drenaggio linfatico

  \begin{itemize}
  \tightlist
  \item
    Linfonodi che decorrono \TODO{}
  \end{itemize}
\end{itemize}

\hypertarget{intestinuo-tenue-mesenteriale}{%
\subsubsection{Intestinuo tenue
mesenteriale}\label{intestinuo-tenue-mesenteriale}}

\begin{itemize}
\tightlist
\item
  Tutte le anse intestinali si trovano nella regione sottomesocolica
\item
  Limiti:

  \begin{itemize}
  \tightlist
  \item
    Fessura duodeno-digiunale cranialmente
  \item
    Valvola ileo-ciecale caudalmente a dx
  \end{itemize}
\item
  6/8 m (sticazzi), calibro del lume gradualmente discendente (3 cm
  iniziali, 2 finali)
\end{itemize}

\hypertarget{radice-del-mesentere}{%
\paragraph{Radice del mesentere}\label{radice-del-mesentere}}

\TODO{}

\hypertarget{peritoneizzazione-ed-organogenesi}{%
\paragraph{Peritoneizzazione ed
organogenesi}\label{peritoneizzazione-ed-organogenesi}}

\begin{itemize}
\tightlist
\item
  L'intestino medio è sospeso alla parete addominale dal meso. È
  separato in 2 metà (cefalica e caudale) dal peduncolo vitellino che si
  trova a metà altezza del tubo intestinale
\end{itemize}

\footnotesize

Il limite superiore, a questo punto, è dato dal punto in cui il dotto si
apre nel duodeno, dove sta finendo ora il piccolo omento. A metà, in
senso ventrodorsale, troviamo l'MSA. Da qui si staccano rami
superiormente al segmento cefalico (moltissimi), e pochi in senso
caudale \normalsize

\begin{itemize}
\tightlist
\item
  Avvengono, contemoraneamente
  \TODO{non si capisce niente, guardare video dell'tizio dell'acland che è molto chiaro}

  \begin{enumerate}
  \def\labelenumi{\arabic{enumi}.}
  \tightlist
  \item
    Rotazione di 90° attorno all'asse della MSA

    \begin{itemize}
    \tightlist
    \item
      Il segmento caudale va a sx e il segmento cefalico a dx
    \end{itemize}
  \item
    Un allungamento impetuoso della parte craniale + un allungamento
    della parte caudale molto minore
  \item
    La porzione ciecale si porta nella fossa iliaca dx. Le anse, quindi,
    si portano a sx e riempono tutto il resto della cavità addominale
  \end{enumerate}
\end{itemize}

\hypertarget{organogenesi-del-contenuto-delladdome}{%
\subsection{Organogenesi del contenuto
dell'addome}\label{organogenesi-del-contenuto-delladdome}}

\begin{enumerate}
\def\labelenumi{\arabic{enumi}.}
\tightlist
\item
  Mesogastrio dorsale + meso ventrale (setto trasverso), entrambi
  costituiti da una \emph{coppia} di foglietti
\item
  Nel meso ventrale cresce abbozzo epatico
\item
  Il collegamento che rimane tra abbozzo epatico e intestino anteriore
  (margine inferiore libero del meso ventrale) verrà via biliare
\item
  Il meso ventrale viene divisa dal fegato in due parti

  \begin{itemize}
  \tightlist
  \item
    Piccolo omento
  \item
    Legamento falciforme
  \end{itemize}
\item
  Il piano del meso sepimenta la cavità celomatica in 2 cavità, dx e sx.
  In questo momento, avvengono 3 eventi contemporanei
  (\TODO{img langman pg 197})

  \begin{itemize}
  \tightlist
  \item
    Rotazione dello stomaco

    \begin{itemize}
    \tightlist
    \item
      Lo stomaco ruota per \textbf{1/4 di giro lungo l'asse
      longitudinale verso dx} (margine anteriore → destro, posteriore →
      sinistro)
    \item
      Contemporaneamente, rotazione \textbf{anteroposteriore}
    \item
      Tuttavia, il margine ora destro cresce molto più lentamente
      rispetto al sinistro: a sinistra si verrà a creare la grande
      curvatura, mentre a destra si marcherà ulteriormente la convessità
      della piccola curvatura
    \item
      Per via dell'accrescimento del lato sinistro, la porzione
      inferiore pilorica si sposterà verso l'alto, rendendosi
      orizzontale (sarà la porzione che si sviluppa in orizontale)
    \end{itemize}
  \item
    Formazione della borsa omentale

    \begin{itemize}
    \tightlist
    \item
      La rotazione dello stomaco tira il mesogastrio dorsale a sinistra
      (mesogastrio ventrale rimane vincolato alla parete ventrale)
    \item
      La rotazione porta la cavità destra a ridursi, e a portarsi
      leggermente posteriormente; la cavità celomatica sinistra, invece,
      si ingrandirà
    \item
      La cavità (ex dx) posteriore diventerà la
      \a{cavità retroperitoneale}
    \item
      Per far fronte alla rotazione del piano dei mesi verso dx, il meso
      dorsale (che si sta trovando sempre più a sx) si deve allungare
    \item
      Questo allungamento repentino porterà ad un eccesso di formazione,
      che scavalcherà il \a{colon trasverso}. L'eccesso diventa il
      \a{recesso omentale inferiore}. Siccome è un eccesso del foglietto
      esterno del meso, sarà connesso alla \a{borsa omentale} (ex cavità
      celomatica di destra). Il recesso si accolla al meso sul quale si
      era ripiegato, costituendo una formazione a 4 foglietti che
      diventa il \a{grande omento}
    \end{itemize}
  \item
    Fenomeni di accollamento

    \begin{itemize}
    \tightlist
    \item
      La parete dorsale della \a{borsa omentale} si accolla al lato
      ventrale della parete dorsale della cavità addominale
    \item
      L'accollamento si arresta prima di arrivare alla milza, che rimane
      quindi svincolata
    \item
      Il recesso inferiore della borsa omentale, con i suoi 4 foglietti
      che compongono il grande omento, si accollano a vicenda (come già
      detto). Dal basso verso l'alto, le 2 coppie diventano una
      struttura unica ⇒ la cavità del recesso inferiore diminuisce di
      estensione. L'accollamento termina a livello del
      \a{colon trasverso}

      \begin{itemize}
      \tightlist
      \item
        La parte superiore a quella accollata, pertanto a 2 foglietti,
        costituirà il \a{legamento gastro-colico}: teso, non a caso, tra
        colon trasverso e stomaco
      \item
        La parte tesa tra colon trasverso e parete dorsale, che va a
        costituire il foglietto che limita inferiormente la borsa
        omentale, si accolla con il foglietto che lega colon trasverso
        con la parete dorsale della cavità addominale, diventando di
        fatto un foglio unico
      \end{itemize}
    \item
      Si viene a creare un \a{foro epiploico} che mette in comunicazione
      la \a{grande cavità addominale} con la \a{borsa omentale}
    \end{itemize}
  \end{itemize}
\item
  In tutto sto casino, la milza viene tirata verso sinistra. I vasi che
  a lei afferiscono, pure loro si piegano verso sinistra

  \begin{itemize}
  \tightlist
  \item
    Al momento la milza continua ad essere avvolta dal peritoneo:
    l'accollamento tra la parete dorsale della borsa con il peritoneo
    parietale termina dopo il pancreas, ma prima della milza
  \end{itemize}
\item
  Dorsalmente allo stomaco, il peritoneo che lo riveste nel riflettersi
  contro il diaframma e contro la parete dorsale per diventare peritoneo
  dorsale, lascia scoperta una porzione di fondo dello stomaco, che è
  libera (in quanto non peritoneizzata), di aderire al diaframma

  \begin{itemize}
  \tightlist
  \item
    Ventralmente questa riflessione l'abbiamo lievemente superiore,
    tanto che pure una parte distalissima di esofago viene
    peritoneizzata
  \end{itemize}
\item
  Dal lato della grande curvatura dello stomaco, abbiamo ancora un
  tratto di peritoneo (il bordo laterale, che si sta facendo sempre più
  posteriore, del grande omento) che unisce la milza con stomaco.
  Diventa \a{legamento lienale}
\end{enumerate}

\hypertarget{intestino-anteriore}{%
\subsubsection{Intestino anteriore}\label{intestino-anteriore}}

\begin{itemize}
\tightlist
\item
  IV settimana

  \begin{itemize}
  \tightlist
  \item
    Addome comprende un abbozzo di fegato, di stomaco e di milza
  \item
    L'intestino anteriore è l'unico sospeso da 2 mesi: un meso dorsale
    ed un meso ventrale

    \begin{itemize}
    \tightlist
    \item
      Nel meso \textbf{ventrale} ritroviamo l'abbozzo epatico e la via
      biliare, che definisce il margine inferiore
    \item
      Nel meso \textbf{dorsale} ritroviamo l'abbozzo della milza
    \end{itemize}
  \end{itemize}
\item
  Tra i mesi individuiamo porzioni di peritoneo, che vanno a costituire
  svariati legamenti

  \begin{itemize}
  \item
    \TODO{}
  \end{itemize}
\end{itemize}

\hypertarget{duodeno-1}{%
\paragraph{Duodeno}\label{duodeno-1}}

\begin{itemize}
\tightlist
\item
  Duodeno, così come tutta ansa intestinale, è sospesa dal meso
\item
  Per via della rotazione dello stomaco, accompagnata dalla rotazione e
  torsione dell'intestino medio, portano il duodeno a dx con concavità
  rivolta verso sx e, sopratutto, dorsalmente, spingendolo insieme al
  pancreas a contatto con il peritoneo dorsale
\item
  Concavità accoglie la testa del pancreas; questa cresce e spinge il
  duodeno sempre più verso dx
\item
  Accollamento al retroperitoneo ⇒ retroperitoneizzazione di tutta la
  parte distale del duodeno e del pancreas con cui è a contatto (con
  contemporanea scomparsa del mesoduodeno che sospendeva il duodeno
  pre-accollamento)
\item
  La parte prossimale (prima parte orizzontale) viene rivestita
  dall'ultima parte del piccolo omento
\end{itemize}

\hypertarget{colon}{%
\paragraph{Colon}\label{colon}}

\begin{itemize}
\tightlist
\item
  L'accollamento del duodeno avviene nella parte centrale, e questo
  frammenta il meso completo in 2 pagine, unite nel punto di
  accollamento

  \begin{itemize}
  \tightlist
  \item
    Un mesentere
  \item
    Un mesocolon \TODO{WHAT}
  \end{itemize}
\item
  Nella cloaca si aprono, separati dal \a{setto uro-rettale}

  \begin{itemize}
  \tightlist
  \item
    Parte terminale dell'intestino posteriore
  \item
    Seno uro-genitale (\a{allantoide})
  \end{itemize}
\end{itemize}

\hypertarget{cieco}{%
\paragraph{Cieco}\label{cieco}}

\begin{itemize}
\tightlist
\item
  Intestino crasso situati adi sotto della linea della
  \a{valvola ileo-ciecale}
\item
  Rapporto

  \begin{itemize}
  \tightlist
  \item
    Ventralmente: parete addominale anteriore
  \item
    Dorsalmente: \mus{iliaco} (siamo nella fossa iliaca)
  \item
    Medialmente: intestino tenue mesenteriale
  \item
    Lateralmente: parete laterale della fossa iliaca
  \end{itemize}
\item
  Posizionato nella \a{fossa iliaca} per solo il 50\% dei casi:

  \begin{itemize}
  \tightlist
  \item
    1\% ipocondrio sx
  \item
    30\% ipocondrio dx
  \item
    50\% fossa iliaca dx
  \item
    15\% inguine dx
  \end{itemize}

  questo accade per via del fatto che, embriologicamente, il corretto
  posizionamento viene da una rotazione di \underline{circa} 270°, ma
  non è che l'embrione si mette lì con il goniometro: se ruota di più,
  troveremo il cieco più in basso; viceversa sarà più in alto
\item
  Alla base del cieco ritroviamo l'\a{appendice vermiforme}

  \begin{itemize}
  \tightlist
  \item
    Posizione variabile: spesso retrociecale, ma anche subciecale o
    pelvica con buona frequenza
  \item
    Origine: \textbf{punto di McBurney} (1/3 laterale della linea
    condotta tra ombelico e spina iliaca dx). Questo è valido nel 50\%
    degli individui, ovvero in quelli in cui la statistica ci dice che
    il fondo del cieco si trova nella fossa iliaca dx
  \item
    L'appendice è \emph{a volte} peritoneizzata

    \begin{itemize}
    \tightlist
    \item
      Se o è, è tramite un meso che diparte da quello intestinale
    \item
      Dà passaggio all'\art{arteria appendicolare}
    \end{itemize}
  \end{itemize}
\item
  \a{valvola ileo-ciecale}\footnote{È una valvola e non uno sfintere
    perché non abbiamo uno strato muscolare circolare \emph{proprio}
    della struttura, ma abbiamo la collaborazione tra la muscolautra
    iliaca e ciecale}

  \begin{itemize}
  \tightlist
  \item
    Mucosa
  \item
    Sottomucosa
  \item
    Strato muscolare circolare dell'ileo

    \begin{itemize}
    \tightlist
    \item
      Una parte piega verso l'alto o il basso, andando a irrobustire la
      tenia posteromediale che corre longitudinalmente
      (\a{tenia mesocolica})
    \item
      Una parte prosegue approfondandosi nella parete ciecale, andando a
      costituire la papilla muscolare che sporge nel lume del cieco
    \end{itemize}
  \item
    Strato muscolare circolare del cieco
  \item
    Strato muscolare longitudinale
  \item
    Peritoneo
  \end{itemize}
\item
  Il cieco è \textbf{completamente peritoneizzato}

  \begin{itemize}
  \tightlist
  \item
    Possiamo sollevare il cieco, esponendo la \a{fossa retrociecale}
  \item
    La fossa è chiusa dal punto di termine dell'accollamento del
    sovrastante \a{mesocolon ascendente}

    \begin{itemize}
    \tightlist
    \item
      Il punto di termine dell'accollamtno è variabile: normalmente
      corrisponde al margine superiore del cieco, ma alle volte si
      spinge verso il basso e alle volte verso l'alto, trasformando il
      recesso retrociecale in \a{recesso retrocolico ascendente}
    \end{itemize}
  \item
    Sulla linea di accollamento avremo la riflessione peritoneale
  \item
    La valvola ileociecale è peritoneizzata. Tuttavia spesso sono
    riscontrabili dei recessi date dalla sovrapposizione delle pieghe
    peritoneali (bloccati però dai fenomeni di accollamento del
    mesocolon, tali per cui non si formano delle tasche peritoneali, ma
    solo pieghe con piccoli recessi\footnote{Se non si fosse accollato
      il mesocolon, una piega con un recesso darebbe accesso ad una
      tasca fatta dalla parte non accollata del peritoneo})

    \begin{itemize}
    \tightlist
    \item
      Un recesso, la \a{piega ileociecale superiore}
    \item
      Un altro recesso, molto molto piccolo, dato da una piega del
      peritoneo (\a{piega ileociecale inferiore})
    \end{itemize}
  \end{itemize}
\end{itemize}

\hypertarget{colon-ascendente}{%
\paragraph{Colon ascendente}\label{colon-ascendente}}

\begin{itemize}
\tightlist
\item
  Dalla valvola ileociecale fino al gomito epatico
  (\a{fessura colica dx}). Circa 15 cm
\item
  Andamento ventrodorsale
\item
  Rapporti

  \begin{itemize}
  \tightlist
  \item
    Dorsali: \mus{quadrato dei lombi} e \TODO{}
  \item
    Ventrali: parete addominale
  \item
    Mediali: anse intestinali
  \item
    Laterali: \a{doccia paracolica dx}\footnote{Uno spazio della parete
      addominale che si estende dal colon ascendente dx fino alla parte
      sovraepatica e sottodiaframmatica}
  \end{itemize}
\end{itemize}

\hypertarget{colon-trasverso}{%
\paragraph{Colon trasverso}\label{colon-trasverso}}

\begin{itemize}
\tightlist
\item
  Parte dalla fessura colica dx, corre trasversale per tutta la
  larghezza dell'addome\footnote{Solitamente, perchè altre volte è molto
    più lungo e si estende verso il basso della parete addominale},
  risalendo leggermente nel lato sx fino a T11 e si arresta a gomito a
  sx
\item
  È completamente peritoneizzato

  \begin{itemize}
  \tightlist
  \item
    La radice \TODO{}
  \end{itemize}
\item
  I due ``gomiti'' compiuti dal colon sono più correttamente detti
  \a{fessure coliche}

  \begin{itemize}
  \tightlist
  \item
    Dx: 80°
  \item
    Sx: 50°, e tesa dal \a{legamento frenocolico} che la ancora a
    diaframma e parete addominale, andando a definire il limite
    superiore della \a{doccia paracolica sx}
  \end{itemize}
\end{itemize}

\hypertarget{colon-discendente}{%
\paragraph{Colon discendente}\label{colon-discendente}}

\begin{itemize}
\tightlist
\item
  Dalla fessura colica sx alla fossa iliaca sx (25 cm)
\item
  Andamento dorsoventrale
\item
  Duale al colon ascendente

  \begin{itemize}
  \tightlist
  \item
    Rapporti con anse intestinali mesenteriali e con la doccia
    paracolica
  \item
    Ugualmente coperti da peritoneo parietale
  \item
    Identica fascia di accollamento
  \end{itemize}
\end{itemize}

\hypertarget{colon-sigmoideo}{%
\paragraph{Colon sigmoideo}\label{colon-sigmoideo}}

\begin{itemize}
\tightlist
\item
  Dal gomito inferiore del colon discendente (sx) si porta nell'asse
  mediano. Occupa il \a{cavo retto-vesciale} (\a{cavo retto-uterino}
  nella donna)
\item
  Limiti

  \begin{itemize}
  \tightlist
  \item
    Inf: margine sup intestino retto (\oss{s3})
  \item
    Sup: gomito del colon disendente
  \end{itemize}
\item
  Rapporti
\item
  Morfologia esterna

  \begin{itemize}
  \tightlist
  \item
    Da 3 tenie passiamo a 2

    \begin{itemize}
    \tightlist
    \item
      Una tenia mesocolica
    \item
      Una tenia omentale (anteirore O posterolaterale)
    \end{itemize}
  \end{itemize}
\item
  Variazione in lunghezza (e quindi anche in disposizione della curva)
\item
  Porzione \textbf{peritoneizzata}. Persiste il meso (\a{mesosigma})

  \begin{itemize}
  \tightlist
  \item
    Individuabile una radice a forma di V aperta verso il basso e verso
    sx

    \begin{itemize}
    \tightlist
    \item
      Punta da L4/L5
    \item
      Una parte orizzontale (originato a valle all'accollamento del
      colon discendente) e una parte obliqua
    \end{itemize}
  \item
    Sollevando il mesosigma, all'apice della V, ritroviamo la
    \a{fosstta intersigmoidea}

    \begin{itemize}
    \tightlist
    \item
      Dà passaggio ai \a{vasi iliaci esterni} e all'\a{uretere}
    \end{itemize}
  \end{itemize}
\end{itemize}

\hypertarget{vascolarizzazione-del-crasso}{%
\paragraph{Vascolarizzazione del
crasso}\label{vascolarizzazione-del-crasso}}

\begin{itemize}
\tightlist
\item
  Arteriosa

  \begin{itemize}
  \tightlist
  \item
    Cieco, colon ascendente, 2/3 del trasverso: \art{MSA}, come tutto
    l'intestino anteriore???. Giunti al mesentere, proprio prima di
    incontrare il colon, cedono rami retti (che attraversano lo strato
    muscolare e la sottomucosa) e rami lunghi (che rimangono
    superficiali, circondano il mesentere e si distribuiscono alle
    appendici epiploiche)

    \begin{itemize}
    \tightlist
    \item
      \art{arteria colica media}: si porta nello spessore del
      \a{mesocolon trasverso}, diretta ai 2/3 prossimali del
      \a{colon trasverso}. Qui si divide

      \begin{itemize}
      \tightlist
      \item
        Ramo sx: diretto al 1/3 distale dei 2/3
      \item
        Ramo dx: diretto al 1/3 prossimale dei 2/3, verso la fessura
        epatica
      \end{itemize}
    \item
      \art{arteria colica destra}: diretta al \a{colon ascendente}. Si
      divide

      \begin{itemize}
      \tightlist
      \item
        Ramo ascendente: anastomosi a pieno canale con ramo dx della
        colica media
      \item
        Ramo discendente: anastomosi a pieno canale con ramo colico
      \end{itemize}
    \item
      \art{arteria ileocolica}: si divide

      \begin{itemize}
      \tightlist
      \item
        Ramo colico: diretto verso l'alto. anastomosi a pieno canale con
        ramo discendente della colica dx
      \item
        Ramo ileale: diretto verso il basso. Si anastomizza con
        \TODO{la porzione distale della MIA?}

        \begin{itemize}
        \item
          \art{ciecale anteriore}

          \begin{itemize}
          \tightlist
          \item
            Passa nello spessore della \a{piega ileociecale superiore}

            \begin{itemize}
            \tightlist
            \item
              Diretta alla superficie anteriore del cieco
            \end{itemize}
          \end{itemize}
        \item
          \art{ciecale posteriore}

          \begin{itemize}
          \tightlist
          \item
            Passa dietro la \a{giunzione ileociecale}, nascosta
            dall'ileo e dal cieco

            \begin{itemize}
            \tightlist
            \item
              Diretta alla superficie posteriore del cieco
            \end{itemize}
          \end{itemize}
        \item
          \art{ciecale appendicolare}

          \begin{itemize}
          \tightlist
          \item
            Decorre nel \a{meso appendicolare} che sostiene l'appendice

            \begin{itemize}
            \tightlist
            \item
              Vascolarizza l'\a{appendice vermiforme}
            \end{itemize}
          \end{itemize}
        \end{itemize}
      \end{itemize}
    \end{itemize}
  \item
    1/3 del trasverso, discendente, sigmoideo e retto: \art{MIA}. Questa
    origina dalla parte ant sx dell'\art{aorta}, scende obliquamente e
    si divide in 3

    \begin{itemize}
    \item
      \art{arteria colica sx}: retroperitoneale

      \begin{itemize}
      \tightlist
      \item
        Ramo ascendente: anastomosi con ramo sx colica media
      \item
        Ramo discendente: anastomosi con ramo superiore della prima
        sigmoidea
      \end{itemize}
    \item
      3 arterie sigmoidee, che si dividono in 2 rami (ramo superiore si
      anastomizza con arteria superiore, ramo inferiore con arteria
      inferiore). L'anastomosi è condotta per archi (e ci sono arcate
      successive di diverso ordine, con l'arcata più marginale {[}=
      esterna{]} che cede le arterie rette, essendo la più vicina al
      colon). Sono intraperitoneali (perché il colon sigmoideo èè
      peritoneizzato), e quindi passano nello spessore del \a{mesosigma}

      \begin{itemize}
      \item
        \art{sigmoidea superiore}
      \item
        \art{sigmoidea media}
      \item
        \art{sigmoidea inferiore}
      \end{itemize}
    \item
      \art{arteria rettale superiore}
    \end{itemize}
  \item
    Si sviluppano anastomosi tra le due reti
  \end{itemize}
\item
  Venosa

  \begin{itemize}
  \tightlist
  \item
    Satellite dell'arterioso
  \item
    Cieco, ascendente e 2/3 prossimali del trasverso drenano nella
    \ven{VMS}
  \item
    1/3 distale, discendente, sigmoideo drenano nella \ven{VMI}
  \item
    La \ven{vmi} ha un decorso diverso: confluisce decisamente più in
    alto

    \begin{itemize}
    \tightlist
    \item
      Talvolta nel circolo portale
    \item
      Talvolta nella \ven{vms}
    \item
      Talvolta nella \ven{lienale}
    \end{itemize}
  \end{itemize}
\item
  Drenaggio infatico

  \begin{enumerate}
  \def\labelenumi{\arabic{enumi}.}
  \item
    \lin{linfonodi paracolici}
  \item
    \lin{linfonodi intermedi}
  \item
    \lin{linfonodi superiori} a livello di MSA
  \item
    \lin{cisterna del chilo}
  \end{enumerate}
\end{itemize}

\hypertarget{intestino-retto}{%
\subsubsection{Intestino retto}\label{intestino-retto}}

\begin{itemize}
\tightlist
\item
  Inizia nel punto di scomparsa del meso che sospende il sigma
\item
  Lunghezza: 18cm cca
\item
  2 parti

  \begin{itemize}
  \tightlist
  \item
    \a{retto pelvico} (nella piccola pelvi)
  \item
    \a{canale anale} (che attraversa il pavimento pelvico, ultimi 3 cm)
  \end{itemize}
\item
  Morfologia esterna

  \begin{itemize}
  \tightlist
  \item
    Non più tenie (⇒ muscolatura longitudinale completa)
  \item
    Non più haustre nè appendici epiploiche
  \item
    Parete completamente liscia
  \end{itemize}
\end{itemize}

\hypertarget{retto-pelvico}{%
\paragraph{Retto pelvico}\label{retto-pelvico}}

\begin{itemize}
\tightlist
\item
  Situato nella \a{loggia rettale}

  \begin{itemize}
  \tightlist
  \item
    Post: sacroo e coccige, coerptura da \mus{m ichiococcigei} e
    \mus{piriforme}
  \item
    Lat: m \mus{elevatoore dell'ano}, con sua fascia
  \item
    Inf: aderenza delle fibre del m \{elevatore dell'ano\} con
    muscolatura longitudinale del retto (\mus{puborettale}, che circonda
    a fionda estremità intestino retto -- tirandola, quindi, in avanti
    verso la sinfisi)
  \end{itemize}
\item
  Decorso non proprio retto, ma con flessure

  \begin{itemize}
  \tightlist
  \item
    Ventrali

    \begin{itemize}
    \item
      \a{flessura sacrale}: causata dal puborettale che tira il retto
      ventralmente, verso la sinfisi
    \item
      \a{flessura perineale}
    \end{itemize}
  \item
    Laterali (visibili solo se retto è vacuo). Internamente, nel lume,
    costituiscono delle ``valvole''\footnote{Improprio perché non
      valvole dal punto di vista funzionale, ma solo delle pieghe
      rilevate nel lume}

    \begin{itemize}
    \tightlist
    \item
      1/a verso sx
    \item
      2/a verso dx
    \item
      3/a verso dx
    \end{itemize}
  \end{itemize}
\item
  Peritoneizzazione:

  \begin{itemize}
  \tightlist
  \item
    Solo parietale, nel primo segmento prossimale

    \begin{itemize}
    \tightlist
    \item
      Piccola parte superficie anteriore prossimale
    \item
      Porzione laterale
    \item
      Riflessione nella parete pelvica nella parte prossimale

      \begin{itemize}
      \tightlist
      \item
        Linea di riflessione va a delimitare 2 \a{fosse pararettali},
        laterali al retto
      \item
        Inferiormente alle fosse pararettali, delimitate sup dalla linea
        di riflessione del peritone o e inflat dal m
        \mus{elevatore dell'ano} individuiamo le
        \a{fosse ischiorettali}, riempite di tessuto fibroadiposo
        sottocutaneo
      \end{itemize}
    \item
      Nella donna

      \begin{itemize}
      \tightlist
      \item
        Riflette sulla parete posteriore della \a{vagina}
      \item
        Ricopre a lenzuolo l'\a{utero} (identifica una zona cava, detta
        \a{cavo retto-uterino})
      \end{itemize}
    \item
      Nell'uomo

      \begin{itemize}
      \tightlist
      \item
        Riflette sulle vesciche seminali e sulla parte dorsale della
        \a{vescica} (identifica una zona cava, detta
        \a{cavo retto-vescicale})
      \end{itemize}
    \end{itemize}
  \item
    Nella parte distale (inf zona di riflessione) non peritoneizzazione

    \begin{itemize}
    \tightlist
    \item
      Lo spazio è riempito da tessuto fibroadiposo di riempimento
      (\a{mesoretto}\footnote{Meso- solo per provenienza embriologica,
        non ha significato funzionale di sospensione come mesi
        intestinali superiori})
    \item
      Mesoretto delimitato da \tol{fascia mesorettale}
    \end{itemize}
  \end{itemize}
\item
  Rapporti

  \begin{itemize}
  \tightlist
  \item
    Post: sacro e coccige, mediante mesoretto.
    \art{arteria sacrale media} (prosecuzione terminale
    dell'\art{aorta}). Catena gangliare del simpatico
  \item
    Lat

    \begin{itemize}
    \tightlist
    \item
      Peritoneizzata: \a{fosse pararettali}
    \item
      Non peritoneizzata: \art{a rettale media}\footnote{da
        \art{epigastrica inf}}
    \end{itemize}
  \item
    Ant:

    \begin{itemize}
    \tightlist
    \item
      Maschio: \a{cavo rettovesciale}, prostata, vescichette seminali,
      vesciche (mediante \tol{fascia rettoprostatica} ben robusta -- o
      \tol{fascia di denonviliers})
    \item
      Femmina: \a{cavo rettouterino}, vagina e utero (mediante
      \tol{fascia rettovaginale} che è più sottile di quella maschile)
    \end{itemize}
  \end{itemize}
\end{itemize}

\hypertarget{canale-anale}{%
\paragraph{Canale anale}\label{canale-anale}}

\begin{itemize}
\tightlist
\item
  Circondato da m \mus{sfintere esterno dell'ano}
\item
  Tra \oss{ischio} e \a{ano} si individuano le \a{fosse ischioanali}
  riempite da tessuto fibroadiposo di riempimento. Il tessuto viene
  diviso trasversalmente

  \begin{itemize}
  \tightlist
  \item
    Porzione profonda: lobuli adiposi grandi, perché poco connettivo
    (ruolo ammortizzatore nei confronti del \oss{sacro})
  \item
    Porzione sottocutanea: lobuli adiposi piccoli, perché tanto
    connettivo
  \end{itemize}
\item
  Muscolatura organizzata in 2 fasci

  \begin{itemize}
  \tightlist
  \item
    Strato circolare interno

    \begin{itemize}
    \tightlist
    \item
      Ispessimento molto consistente della muscolatura inferiore alla
      sottomucosa
    \item
      Muscolatura liscia circolare
    \end{itemize}
  \item
    Strato longitudinale esterno

    \begin{itemize}
    \tightlist
    \item
      Più sottile di quello interno
    \item
      \mus{sfintere est dell'ano}: 3 parti

      \begin{itemize}
      \tightlist
      \item
        Sup:
      \item
        Med:
      \item
        Inf: insieme alle fibre verticali della muscolatura congiunta va
        a formare il corrguatore dell'ano
      \end{itemize}
    \item
      \textbf{Muscolatura congiunta}: prossimalmente la muscolatura
      dello strato longitudinale si fonde con le fibre più mediali
      dell'\mus{elevatore dell'ano}

      \begin{itemize}
      \tightlist
      \item
        Nel punto di fusione individuiamo il lim sup del
        \a{canale anale} (3 cm dalla fine)
      \item
        Le fibre della muscolatura congiunta distalmente si divide

        \begin{itemize}
        \tightlist
        \item
          Parte a circondare la muscolatura interna
        \item
          Parte (\emph{fibre verticali}) si sfrangiano e proseguono
          verticalmente a formare il \mus{corrugatore dell'ano}, che si
          fissa nel derma
        \end{itemize}
      \end{itemize}
    \end{itemize}
  \end{itemize}
\item
  Morfologia interna

  \begin{itemize}
  \tightlist
  \item
    \a{linea anorettale} è punto di fusione delle due muscolature, e
    marca il punto di passaggio tra la porzione anale e ileale del retto
  \item
    \a{colonne del morgagni}: terminano nella linea anorettale, e sono
    sporgenti perché, nella loro sottomucosa, ritroviamo vene che fanno
    \a{plesso emorroidale interno}
  \item
    \a{linea anocutanea}: punto di passaggio tra epitelio mucoso ed
    epitelio corneificato
  \end{itemize}
\end{itemize}

\hypertarget{vascolarizzazione-5}{%
\paragraph{Vascolarizzazione}\label{vascolarizzazione-5}}

\begin{itemize}
\tightlist
\item
  Arteriorsa

  \begin{itemize}
  \tightlist
  \item
    \art{arteria rettale superiore}\footnote{Ramo terminale della
      \art{AMI}}

    \begin{itemize}
    \tightlist
    \item
      Vascolarizza quasi tutto retto pelvico
    \item
      Scende nella pelvi tramite mesosigma da \art{AMI} fino a quando
      termina (livello di \oss{s3})
    \item
      Prosegue inferiormente nella superficie dorsale del retto e si
      divide

      \begin{itemize}
      \tightlist
      \item
        Ramo sx: diretto a superficie sx e vent del retto
      \item
        Ramo dx: diretto a superficie dx e dors del retto
      \end{itemize}
    \end{itemize}
  \item
    \art{arteria rettale media}\footnote{Ramo dell'\art{iliaca interna}}

    \begin{itemize}
    \tightlist
    \item
      Cede rami per prostata, vagina
    \item
      Vascolarizza la muscolatura esterna del retto
    \end{itemize}
  \item
    \art{arteria rettale inferiore}\footnote{Ramo della \art{pudenda}}

    \begin{itemize}
    \tightlist
    \item
      Vascolarizza il canale anale
    \item
      Parziale anastomosi con rettale superiore
    \end{itemize}
  \item
    Contributo incostante e ridotto della \art{sacrale media}
  \end{itemize}
\item
  Venosa: \TODO{tutto}

  \begin{itemize}
  \tightlist
  \item
    Al di sopra della linea pettinata
  \item
    Al di sotto della linea pettinata
  \item
    Partono da \ven{plessi emorroidali}, anastomizzato da vene che
    normalmente non sono attive

    \begin{itemize}
    \item
      \ven{plesso emorroidale sottocutaneo}
    \item
      \ven{plesso emorroidale profondo}
    \end{itemize}
  \item
    Risalgono e convergono nella \ven{vena rettale inferiore}
  \item
    La vri si apre nella \ven{pudenda} - \ven{ipogastrica} - \ven{\\}
  \end{itemize}
\end{itemize}

\hypertarget{ghiandole-addominali}{%
\subsubsection{Ghiandole addominali}\label{ghiandole-addominali}}

\hypertarget{pancreas}{%
\paragraph{Pancreas}\label{pancreas}}

\begin{itemize}
\tightlist
\item
  Parte dall'ipocondrio dx (testa) e termina nell'ipocondrio sx (coda)

  \begin{itemize}
  \tightlist
  \item
    Si adatta alle sporgenze a lui ventrali: il \emph{Testut} lo
    definisce come ``il sacco del mugnaio gettato sul dorso del
    somaro''. Oppure a martello

    \begin{itemize}
    \tightlist
    \item
      Concavità ventrale per scavalcare le vertebre
    \end{itemize}
  \end{itemize}
\item
  Configurazione esterna

  \begin{itemize}
  \tightlist
  \item
    \a{testa} che si rapporta mooolto strettamente con il duodeno. Si
    conclude, inferiormente, con il \a{processo uncinato}
  \item
    \a{collo} leggermente ristretto, definito superiormente dalla MSA e
    inferiormente dal tripode celicao
  \item
    \a{corpo} esteso e trapezoidale
  \item
    \a{coda} terminale. Da qui parte il
    \tol{legamento pancreaticolienale}
  \end{itemize}
\item
  Il dotto escretore è \textbf{unico} ma si sdoppia:

  \begin{itemize}
  \tightlist
  \item
    Il corpo è drenato dal \a{dotto di wirsung}
  \item
    La testa è drenata dal \a{dotto di santorini}
  \end{itemize}

  Si uniscono e prosegunono come \a{dotto principale} unico, che piega
  verso il basso e si unisce al \a{coledoco} biliare, e si aprono
  insieme nel duodeno a livello delle \a{papilla duodenale infeirore}
\item
  Peritoneizzazione

  \begin{itemize}
  \item
    \TODO{}
  \item
    Linea di riflessione trasversa che separa la parte retrioperitoneale
    da quella peritoneale
  \end{itemize}
\end{itemize}

\hypertarget{organogenesi}{%
\subparagraph{Organogenesi}\label{organogenesi}}

\begin{itemize}
\tightlist
\item
  Da due abbozzi, uno che originerà la testa e uno la coda. Ogni abbozzo
  ha il suo dotto (ecco perché ce ne sono 2 che si fondono)

  \begin{itemize}
  \tightlist
  \item
    \textbf{Abbozo dorsale}: contenuto nel mesoduodeno. Risale nel
    mesogastrio dorsale
  \item
    \textbf{Abbozzo ventrale}: origina a stretto contatto con via
    biliare. Ecco perché il dotto principale si unisce con il coledoco
  \end{itemize}
\item
  La \textbf{rotazione} porta l'abbozzo ventrale \emph{sotto} l'abbozzo
  dorsale, trascinando con sé la via biliare\footnote{A volte l'abbozzo
    ventrale non migra posteriormente, ma anche anteriormente. Questo
    porta ad avvolgere ad anello il duodeno (\pat{pancreas anulare}),
    che viene quindi strozzato}

  \begin{itemize}
  \tightlist
  \item
    I dotti si uniscono e si forma il dotto unico
  \item
    Il dotto unico si unisce nella via biliare
  \end{itemize}
\item
  La rotazione porta il pancreas ad andare verso sx, insieme all'arteria
\item
  In senso mediolaterale comincia l'accollamento

  \begin{itemize}
  \tightlist
  \item
    La testa viene accollata
  \item
    La coda no, e rimane sporgente nella \a{borsa omentale} (a lei
    ventrale)
  \end{itemize}
\end{itemize}

\hypertarget{fegato}{%
\paragraph{Fegato}\label{fegato}}

\TODO{lezione del 6 nov mattina}

\hypertarget{descrizione-anatomica}{%
\subparagraph{Descrizione anatomica}\label{descrizione-anatomica}}

\hypertarget{descrizione-funzionale}{%
\subparagraph{Descrizione funzionale}\label{descrizione-funzionale}}

\hypertarget{vascolarizzazione-6}{%
\subparagraph{Vascolarizzazione}\label{vascolarizzazione-6}}

Ipertensione portale

\begin{itemize}
\item
  \pat{ipertensione portale}: ↑ pressione all'intero della vena porta
\item
  Eziologia

  \begin{itemize}
  \tightlist
  \item
    Ostruzione del sistema portale

    \begin{itemize}
    \tightlist
    \item
      Dilatazione varicosa e \textbf{rottura} delle
      \a{anastomosi portocavali} perché il sangue cerca via alternativa
      a \textless{} resistenza
    \item
      Sovraccarico dell'ananstomosi portocavale
    \end{itemize}
  \end{itemize}
\item
  Presentazione clinica: sovraccarico a monte delle anastomosi
  portocavali

  \begin{itemize}
  \item
    Varici delle vene periesofagee dell'esofago sottodiaframmatico
    (anastomosi portocavale)
  \item
    Addome a \emph{caput medusae} (dilatazione varicosa estrema del
    circolo superficiale della parete anteriore, tributaria della
    \ven{paraomblicle} -- ostruita dall'ipt portale)
  \item
    Emorroidi (sovraccarico plesso rettale superficiale?)
  \item
  \end{itemize}
\end{itemize}

\hypertarget{drenaggio-linfatico-2}{%
\paragraph{Drenaggio linfatico}\label{drenaggio-linfatico-2}}

\begin{itemize}
\tightlist
\item
  \a{spazio di disse} è connesso a \a{spazio di mall}
\item
  Linfa drena \a{spazio di mall}
\item
  Linfa estreamente ricca di proteine
\item
  Circolo

  \begin{itemize}
  \tightlist
  \item
    Profondo

    \begin{itemize}
    \tightlist
    \item
      Decorso delle vene sovraepatiche nella parte superiore
    \item
      Decorso della vena porta nella parte inferiore
    \end{itemize}
  \item
    Superficiale
  \end{itemize}
\end{itemize}

\newpage

\hypertarget{bibliografia}{%
\section*{Bibliografia}\label{bibliografia}}
\addcontentsline{toc}{section}{Bibliografia}

\hypertarget{refs}{}
\leavevmode\hypertarget{ref-netter2014atlas}{}%
1. Netter F. Atlante di anatomia umana. Philadelphia, PA:
Saunders/Elsevier; 2014.

\leavevmode\hypertarget{ref-netter2018atlas}{}%
2. Netter F. Atlas of human anatomy. Philadelphia, PA: Elsevier; 2018.

\leavevmode\hypertarget{ref-gray2017anatomia}{}%
3. Gray H. Anatomia del Gray: le basi anatomiche per la pratica clinica.
Milano: Edra; 2017.

\leavevmode\hypertarget{ref-schunke2010thieme}{}%
4. Schunke M. Thieme atlas of anatomy. Stuttgart New York: Thieme; 2010.

\end{document}
