\PassOptionsToPackage{dvipsnames}{xcolor}

\usepackage[heightrounded, margin=4cm, marginpar=3cm]{geometry}
\usepackage{chemfig}            % Typeset molecules
\usepackage{fancyhdr}           % Header and footers
\usepackage{float}              % Control how images float
\usepackage{newunicodechar}     % Unicode where it's needed
\usepackage{tcolorbox}          % Boxes
\usepackage{marvosym}
\usepackage{qrcode}
\usepackage{enumitem}
\usepackage{xcolor} % Colors!
\usepackage{libertine, libertinust1math}
\usepackage{titlesec}
\usepackage{nameref}
\usepackage{bookmark}
\usepackage{todonotes}
\renewcommand{\familydefault}{\sfdefault}

\PassOptionsToPackage{
    activate=true,
    protrusion=true,
    expansion=true,
    final,
    tracking=true,
    kerning=true,
    spacing=true,
%    factor=1100,
%    stretch=10,
%    shrink=10
}{microtype} % Typographic perfection, sadly only with pdftex

% Titlesec {{{
% \titleformat{command}[shape]{format}{label}{sep}{before-code}[after-code]
\titleformat{\part}{\flushright\huge\normalfont\itshape}{\thepart}{1em}{\phantomsection}[\vspace{8em}]  % \part
\titleformat{\chapter}{\huge\normalfont\scshape\lowercase}\thechapter{}{0em}{\phantomsection}           % Book class
\titleformat{\section}{\huge\normalfont\scshape\lowercase}{\thesection}{1em}\phantomsection{}           % # Header #
\titleformat{\subsection}{\LARGE\bfseries}{}{0em}{\phantomsection}                                      % ## Header ##
\titleformat{\subsubsection}{\Large\bfseries}{}{0em}{\phantomsection}                                   % ### Header ###
\titleformat{\paragraph}{\large\bfseries}{}{0em}{\phantomsection}                                       % #### Header ####
\titleformat{\subparagraph}{\normalfont\bfseries}{}{0em}{\phantomsection}                               % ##### Header #####
\titleformat{\subsubparagraph}{\itshape}{}{0em}{\phantomsection}
% }}}

\renewcommand*\ttdefault{txtt}

\setcounter{secnumdepth}{2}  % Number only up to \section
\setcounter{tocdepth}{6}     % Put everything up to \subsubparagraph in toc

\renewcommand\footnoterule{}
\setlength{\skip\footins}{3em}

\newcommand{\marginfig}[1]{\marginpar{{\textsf{Vedi \textbf{Figura \ref{#1}}}}}}
\newcommand{\marginnote}[1]{\marginpar{\footnotesize← \emph{#1}}}
\newcommand{\asidefigure}[2]{\marginpar{\includegraphics{#1}\\ \footnotesize \emph{ #2}}}

\newunicodechar{½}{$\frac{1}{2}$}
\newunicodechar{¼}{$\frac{1}{4}$}
\newunicodechar{¾}{$\frac{3}{4}$}
\newunicodechar{⅔}{$\frac{2}{3}$}
\newunicodechar{⅓}{$\frac{1}{3}$}
\newunicodechar{¬}{$\neg$}
\newunicodechar{±}{$\pm$}
\newunicodechar{×}{$\times$}
\newunicodechar{÷}{$\div$}
\newunicodechar{…}{$\dots$}
\newunicodechar{ℕ}{$\mathbb{N}$}
\newunicodechar{ℚ}{$\mathbb{Q}$}
\newunicodechar{ℝ}{$\mathbb{R}$}
\newunicodechar{ℤ}{$\mathbb{Z}$}
\newunicodechar{←}{$\leftarrow$}
\newunicodechar{↑}{$\uparrow$}
\newunicodechar{→}{$\rightarrow$}
\newunicodechar{🡪}{$\rightarrow$}
\newunicodechar{↓}{$\downarrow$}
\newunicodechar{↔}{$\leftrightarrow$}
\newunicodechar{⇒}{$\Rightarrow$}
\newunicodechar{⇐}{$\Leftarrow$}
\newunicodechar{⇔}{$\Leftrightarrow$}
\newunicodechar{∀}{$\forall$}
\newunicodechar{∃}{$\exists$}
\newunicodechar{∅}{$\emptyset$}
\newunicodechar{∈}{$\in$}
\newunicodechar{∉}{$\notin$}
\newunicodechar{∋}{$\ni$}
\newunicodechar{∎}{$\blacksquare$}
\newunicodechar{∑}{$\sum$}
\newunicodechar{∓}{$\mp$}
\newunicodechar{∗}{$\ast$}
\newunicodechar{∘}{$\circ$}
\newunicodechar{∙}{$\bullet$}
\newunicodechar{∝}{$\propto$}
\newunicodechar{∞}{$\infty$}
\newunicodechar{∥}{$\parallel$}
\newunicodechar{∧}{$\land$}
\newunicodechar{∨}{$\lor$}
\newunicodechar{∩}{$\cap$}
\newunicodechar{∪}{$\cup$}
\newunicodechar{∴}{$\therefore$}
\newunicodechar{∵}{$\because$}
\newunicodechar{≈}{$\approx$}
\newunicodechar{≠}{$\neq$}
\newunicodechar{≡}{$\equiv$}
\newunicodechar{≤}{$\leq$}
\newunicodechar{≥}{$\geq$}
\newunicodechar{⊂}{$\subset$}
\newunicodechar{⊃}{$\supset$}
\newunicodechar{⊆}{$\subseteq$}
\newunicodechar{⊇}{$\supseteq$}
\newunicodechar{⊢}{$\vdash$}
\newunicodechar{⊤}{$\top$}
\newunicodechar{⊥}{$\bot$}
\newunicodechar{⊨}{$\vDash$}
\newunicodechar{⋅}{$\cdot$}
\newunicodechar{⋮}{$\vdots$}
\newunicodechar{⋯}{$\cdots$}
\newunicodechar{α}{$\alpha$}
\newunicodechar{Α}{$\Alpha$}
\newunicodechar{β}{$\beta$}
\newunicodechar{Β}{$\Beta$}
\newunicodechar{γ}{$\gamma$}
\newunicodechar{Γ}{$\Gamma$}
\newunicodechar{δ}{$\delta$}
\newunicodechar{Δ}{$\Delta$}
\newunicodechar{ε}{$\varepsilon$}
\newunicodechar{Ε}{$\Epsilon$}
\newunicodechar{ζ}{$\zeta$}
\newunicodechar{Ζ}{$\Zeta$}
\newunicodechar{η}{$\eta$}
\newunicodechar{Η}{$\Eta$}
\newunicodechar{θ}{$\theta$}
\newunicodechar{Θ}{$\Theta$}
\newunicodechar{ι}{$\iota$}
\newunicodechar{Ι}{$\Iota$}
\newunicodechar{κ}{$\kappa$}
\newunicodechar{Κ}{$\Kappa$}
\newunicodechar{λ}{$\lambda$}
\newunicodechar{Λ}{$\Lambda$}
\newunicodechar{μ}{$\mu$}
\newunicodechar{Μ}{$\Mu$}
\newunicodechar{∇}{$\nabla$}
\newunicodechar{ν}{$\nu$}
\newunicodechar{Ν}{$N$}
\newunicodechar{ξ}{$\xi$}
\newunicodechar{Ξ}{$\Xi$}
\newunicodechar{ο}{$\omicron$}
\newunicodechar{Ο}{$\Omicron$}
\newunicodechar{π}{$\pi$}
\newunicodechar{Π}{$\Pi$}
\newunicodechar{ρ}{$\rho$}
\newunicodechar{ϱ}{$\varrho$}
\newunicodechar{Ρ}{$\Rho$}
\newunicodechar{σ}{$\sigma$}
\newunicodechar{ς}{$\varsigma$}
\newunicodechar{Σ}{$\Sigma$}
\newunicodechar{τ}{$\tau$}
\newunicodechar{Τ}{$\Tau$}
\newunicodechar{υ}{$\upsilon$}
\newunicodechar{Υ}{$\Upsilon$}
\newunicodechar{φ}{$\varphi$}
\newunicodechar{ϕ}{$\phi$}
\newunicodechar{Φ}{$\Phi$}
\newunicodechar{χ}{$\chi$}
\newunicodechar{Χ}{$\Chi$}
\newunicodechar{ψ}{$\psi$}
\newunicodechar{Ψ}{$\Psi$}
\newunicodechar{ω}{$\omega$}
\newunicodechar{Ω}{$\Omega$}
\newunicodechar{°}{$^{\circ}$}
\newunicodechar{💙}{$\heartsuit$}
\newunicodechar{₀}{~0~}
\newunicodechar{₁}{~1~}
\newunicodechar{₂}{~2~}
\newunicodechar{₃}{~3~}
\newunicodechar{₄}{~4~}
\newunicodechar{₅}{~5~}
\newunicodechar{₆}{~6~}
\newunicodechar{₇}{~7~}
\newunicodechar{₈}{~8~}
\newunicodechar{₉}{~9~}

% Anatomia
\definecolor{ossa}{HTML}{BEAE84}
\newcommand{\mus}[1]{\colorbox{Salmon}{\textcolor{white}{\textsc{#1}}}}
\newcommand{\oss}[1]{\colorbox{ossa}{\textcolor{white}{\textsc{#1}}}}
\newcommand{\ven}[1]{\colorbox{RoyalBlue}{\textcolor{white}{\textsc{#1}}}}
\newcommand{\art}[1]{\colorbox{RedOrange}{\textcolor{white}{\textsc{#1}}}}
\newcommand{\tol}[1]{\colorbox{Aquamarine}{\textcolor{white}{\textsc{#1}}}}
\newcommand{\ner}[1]{\colorbox{Dandelion}{\textcolor{white}{\textsc{#1}}}}
\newcommand{\lin}[1]{\colorbox{PineGreen}{\textcolor{white}{\textsc{#1}}}}
\newcommand{\far}[1]{\fbox{\textsc{#1}} } % Farmaco (principio attivo)
\newcommand{\farf}[1]{\fbox{\fbox{\textsc{#1}}} } % Famiglia di farmaci
\newcommand{\pat}[1]{\colorbox{black}{\textcolor{white}{\textsc{#1}}}}
\renewcommand{\a}[1]{\underline{\textsc{#1}}}

% Sistema nervoso
\newcommand{\nere}[1]{\colorbox{Dandelion}{\textcolor{Maroon}{\textsc{#1}}}} % fibre nervose efferenti
\newcommand{\nera}[1]{\colorbox{Dandelion}{\textcolor{NavyBlue}{\textsc{#1}}}} % fibre nervose afferenti
\newcommand{\nerm}[1]{\colorbox{Dandelion}{\textcolor{Purple}{\textsc{#1}}}} % fibre nervose miste
\newcommand{\nerdisc}[1]{\colorbox{Dandelion}{\textcolor{Maroon}{\textsc{#1}}}} % fibre nervose efferenti
\newcommand{\nerasc}[1]{\colorbox{Dandelion}{\textcolor{NavyBlue}{\textsc{#1}}}} % fibre nervose afferenti
\newcommand{\nermist}[1]{\colorbox{Dandelion}{\textcolor{Purple}{\textsc{#1}}}} % fibre nervose miste
\newcommand{\nerorto}[1]{\colorbox{Dandelion}{\textcolor{Red}{\textsc{#1}}}} % SN ortosimpatico o simpatico
\newcommand{\nerpara}[1]{\colorbox{Dandelion}{\textcolor{ForestGreen}{\textsc{#1}}}} % SN parasimpatico
\newcommand{\nerent}[1]{\colorbox{Dandelion}{\textcolor{Blue}{\textsc{#1}}}} % SN neurotenterico

% Riferimenti a libri
\newcommand{\gray}[1]{\textsf{ADG, pag. #1}}
\newcommand{\adg}[1]{\textsf{ADG, pag. #1}}
\newcommand{\prom}[1]{\textcolor{NavyBlue}{\textsf{Prometheus, pag. #1}}}
\newcommand{\netter}[1]{ \fbox{\textsf{Netter, tav. #1}} }
\newcommand{\nnetter}[1]{ \fbox{\textsf{Netter 2019, Plate #1}} }

% Tcolorbox
\tcbuselibrary{breakable}
\newcommand{\casoclinico}[3]{\begin{tcolorbox}[title=Caso clinico: #1,colback=cyan!5,colframe=cyan!75!cyan, coltitle=black, list inside=clinic]#2 \tcblower #3 \end{tcolorbox}}
\newcommand{\greenbox}[2]{\begin{tcolorbox}[title=#1,colback=green!5,colframe=green!35!black]#2\end{tcolorbox}}
\newcommand{\normalbox}[2]{\begin{tcolorbox}[title=#1]#2\end{tcolorbox}}
\newcommand{\redbox}[2]{\begin{tcolorbox}[title=#1,colback=red!5,colframe=red!75!black]#2\end{tcolorbox}}
\newcommand{\simplebox}[2]{\begin{tcolorbox}[title=#1]#2\end{tcolorbox}}
\newcommand{\warningbox}[2]{\begin{tcolorbox}[title=#1,colback=yellow!5,colframe=yellow!75!red, coltitle=black]#2\end{tcolorbox}}
\newcommand{\yellowbox}[2]{\begin{tcolorbox}[title=#1,colback=yellow!5,colframe=yellow!75!red, coltitle=black]#2\end{tcolorbox}}

% Altra roba
\newcommand{\att}[0]{ $\oplus$ }                                        % Attivazione o regolazione positiva
\newcommand{\down}{$\downarrow$}
\newcommand{\fig}[1]{\textsf{\textbf{Figura \ref{#1}}}}
\newcommand{\goldstandard}{\textcircled{$\star$} }                      % Gold standard (*)
\newcommand{\ini}[0]{ $\otimes$ }                                       % Inibizione o regolazione negativa
\newcommand{\q}[1]{\textcolor{blue}{\textbf{\textsf{Q: #1?}}}}          % Domandona
\newcommand{\sint}[1]{\textsf{#1}}                                      % Segno o sintomo
\newcommand{\sos}[1]{\textsf{#1}}                                       % segno o sintomo
\newcommand{\up}{$\uparrow$}
\newcommand{\TODO}[1]{\textcolor{red}{\textsf{\footnotesize{TODO #1}}}} % TODO

% Up to 9 level of depth in list
\setlistdepth{9}
\renewlist{itemize}{itemize}{9}
\setlist[itemize,1]{label=\textbullet}
\setlist[itemize,2]{label=\textendash}
\setlist[itemize,3]{label=$\circ$}
\setlist[itemize,4]{label=\textasteriskcentered}
\setlist[itemize,5]{label=$\diamond$}
\setlist[itemize,6]{label=\textperiodcentered}
\setlist[itemize,7]{label=\textperiodcentered}
\setlist[itemize,8]{label=\textperiodcentered}
\setlist[itemize,9]{label=\textperiodcentered}
\renewlist{enumerate}{enumerate}{9}
\setlist[enumerate,1]{label=$\arabic*.$}
\setlist[enumerate,2]{label=$\alph*.$}
\setlist[enumerate,3]{label=$\roman*.$}
\setlist[enumerate,4]{label=$\arabic*.$}
\setlist[enumerate,5]{label=$\alpha*$}
\setlist[enumerate,6]{label=$\roman*.$}
\setlist[enumerate,7]{label=$\arabic*.$}
\setlist[enumerate,8]{label=$\alph*.$}
\setlist[enumerate,9]{label=$\roman*.$}
